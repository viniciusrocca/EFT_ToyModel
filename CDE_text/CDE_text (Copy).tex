\documentclass[10pt,a4paper]{report}


\usepackage[utf8]{inputenc}
\usepackage[T1]{fontenc}
\usepackage[english]{babel}
\usepackage{amsmath,mathtools}
\usepackage{amsfonts}
\usepackage{amssymb}
\usepackage{graphicx}
\usepackage{indentfirst}
\numberwithin{equation}{section}
\usepackage{overpic}
\usepackage{geometry}
\usepackage{url}
\usepackage{color}
\usepackage{tikz-feynman}
\usepackage{physics}
\usepackage{array}
\usepackage[colorlinks,linkcolor=black,citecolor=blue,urlcolor=blue,bookmarks=false,hypertexnames=true]{hyperref}
\usepackage{caption}
\usepackage{subcaption}
\usepackage{multirow}
\usepackage{pbox}
\usepackage{booktabs}
\usepackage{bm}
\usepackage{slashed}
\usepackage{simpler-wick}
\usepackage{titlesec}
\usepackage{dsfont}
\numberwithin{equation}{section} % Number equations within sections 
%Título do capítulo centralizado
%\titleformat{\chapter}{}{\bf\huge\centering}{0em}{\bf\huge\centering}


%Título do capítulo a esquerda
%\titleformat{\chapter}{}{}{}{\bf\huge\centering}


%\usepackage{csvsimple} \csvautotabular{Quadrimomentos.csv} tabela csv para latex
%\usepackage{cite}


\definecolor{gray}{gray}{0.95}

\geometry{a4paper, left = 2.5cm, right = 2.5 cm, top = 1.75 cm, bottom = 1.75 cm}




\newcommand{\unps}[2]{\prescript{}{#1}{#2}}

%Defining some commands

\newcommand{\ccj}{\mathcal{C}}
\newcommand{\id}{\mathds{1}}
\newcommand{\lf}{\mathcal{I}}





\author{Vinícius Rocca}
\title{Monografia}





\title{CDE Method Notes}
\author{Vinicius Rocca}
\date{\today}

\begin{document}
	
	
	\maketitle
	
	\tableofcontents
	
	\newpage
	
	\addcontentsline{toc}{chapter}{Abstract}
	\chapter*{Abstract}
	
	
	\textcolor{red}{Space for Abstract}
	
	

	
	
	
	
	
	
	
	
	
	
	
	
	
	
	
	
	
	\addcontentsline{toc}{chapter}{Introduction}
	\chapter*{Introduction}
	
	\textcolor{red}{Standard Model is solid but we have evidences that is not the definite theory. It is possible that there is new physics at energy scales not achievable for us now and we can only see indirect effects of this new physics. Assuming that this new physics occur in an energy scale much bigger than the SM masses, what we observe can be described by an Effective Field Theory, which comes from integrating out the heavy particles of a complete UV theory that consists in an extension for SM. }
	
	
	
	\chapter{Covariant Derivative Expansion Method (CDE)}
	
	%For modern physics, the ultimate goal is to understand every observed phenomena in terms of some fundamental dynamics among the basic constituents of nature. The so-called theory of everything, would unify all the different kinds of interactions. However, even if we had this theory, a quantitative analysis at the most elementary level would be of little use to providing a comprehensive and practical description of nature at all physical scales. Even only with the knowledge that we have today, we already face similar situations. For instance, in order to understand, in a simple way, the most relevant physics in the atomic structure, we use a simplified description in terms of non-relativistic electrons orbiting around the nuclear Coulomb potential instead of starting with the fundamental Quantum Electrodynamics (QED) to describe the interactions of the quarks composing the proton and the electron. In a first approximation, the atomic structure or even the chemical bond between atoms can be understood in therms of the electron mass $m_e$ and the fine structure constant $\alpha \approx 1/137$, while the proton mass $m_p$ is needed only to estimate the dominant corrections. However, if we try to provide a useful understanding of condensed matter phenomenas in terms of this simplifiead approach to atomic structure, we are going to fail because it becomes very complex.
	
	
	
	
	For modern physics, the ultimate goal is to understand every observed phenomenon in terms of fundamental dynamics among the basic constituents of nature. The so-called "theory of everything" would unify all known interactions. However, even if such a theory were available, performing quantitative analyses directly at the most fundamental level would be of limited practical use for describing nature across all physical scales. In fact, even with the knowledge we currently possess, we often encounter similar challenges.
	
	For example, to understand the essential features of atomic behavior in a simple way, we typically use a simplified description involving non-relativistic electrons moving in the Coulomb potential of a nucleus. This approach is far more practical than starting from the fundamental interactions of quarks inside the proton and the electron, which are described by Quantum Electrodynamics (QED). In a first approximation, atomic behavior, can be described in terms of the electron mass $m_e$ and the fine-structure constant $\alpha\approx 1/137$, with the proton mass $m_p$ being necessary only to estimate leading corrections. However, if we attempt to extend this simplified atomic model to obtain a useful understanding of condensed matter phenomena, we quickly run into difficulties. The complexity increases dramatically, requiring another layer of effective description.
	
	To analyze a particular physical system embedded in a complex environment, it is essential to isolate the relevant parameters, enabling a simplified description without requiring a complete understanding of the entire system. In situations involving widely separated energy scales, it is often possible to study the low-energy dynamics independently of the high-energy details. The key idea is to identify parameters that are very large or very small compared to the characteristic energy scale of interest and to take appropriate limits. This yields a well-defined approximation that can be systematically improved by including corrections arising from the neglected scales, treated as perturbations.
	
	In the context of elementary particle physics described by Quantum Field Theory (QFT), such low-energy approximations, with respect to some energy scale $\Lambda$, are formalized through the framework of Effective Field Theory (EFT). An EFT retains only the degrees of freedom relevant at energies well below $\Lambda$ and must reproduce the same physical observables of the UV theory in within this low-energy regime. To illustrate this, consider a UV theory containing heavy and light particles associated with fields of masses $m\ll\Lambda$ and $M= \Lambda$, respectively. In physical processes occurring at energy scales much smaller than $M$, heavy particles do not appear in the initial or final states, as their on-shell production is energetically forbidden. Therefore, at such energy scales, the light fields constitute the relevant degrees of freedom, and the EFT is constructed entirely in terms of them. However, heavy particle still can contribute virtually in the processes trough loops, and their effects must be encoded in the coefficients of the resulting EFT Lagrangian.
	
	Given a UV theory, an EFT can be systematically derived by integrating out the irrelevant (heavy or high-energy) degrees of freedom using the path integral formalism, followed by the imposition of matching conditions to ensure that the EFT reproduces the low-energy behavior of the full theory. In most cases, this procedure cannot be carried out exactly, and so it is implemented perturbatively. The result is an EFT Lagrangian expressed as an infinite series of higher-dimensional, non-renormalizable operators:\footnote{Although EFTs generally include an infinite number of terms, this does not obstruct renormalization. At any given order in the low-energy expansion, only a finite number of operators contribute, enabling a consistent order-by-order renormalization.} these are built from the light fields and are organized as an expansion in powers of $E/\Lambda$, where $E$ is the characteristic energy of the process under consideration.
	
	In this chapter, we present a method known as the Covariant Derivative Expansion (CDE) for computing the one-loop Effective Field Theory (EFT) Lagrangian starting from a given ultraviolet (UV) theory, while explicitly preserving covariance in the final result. The discussion is primarily based on Refs.\cite{Zhang_2017,Kr_mer_2020} with additional insights drawn from Refs. \cite{Fuentes_Mart_n_2016,henning2016oneloopmatchingrunningcovariant,henning2015usestandardmodeleffective,Cohen_2021}. 
	
	We begin the chapter with a brief review of essential elements of the path integral formalism that form the foundation for the derivation. We then introduce the core ideas behind the CDE approach, focusing on the procedure for integrating out heavy real scalar fields and the concept of matching. To illustrate the method and reinforce the main concepts, we apply it to a simple toy model that allows for a transparent, step-by-step computation.
	
	In the fourth section, we extend the formalism to include Dirac fields thereby generalizing the computation of the EFT Lagrangian, up to one-loop order, to theories involving fermions. Finally, we apply the full framework to a more realistic setting by constructing an EFT for a simple extension of the Standard Model
	
	
	
	
	
	
	
	
%	In any physical process at energy scales much smaller than $M$, the heavy particles does not appear in the initial or final, since their on-shell states are energetically forbidden. Therefore, light fields are the relevant degrees of freedom and an EFT will be entirely built with them.
	
%	we can equivalently calculate any observablean EFT will only contain light fields, since they are the only relevant degrees of freedom.
	
%	 The result is a low-energy theory with an infinite sum of non-renormalizable operators\footnote{Although EFTs generically contain an infinite number of terms, this does not pose a problem for renormalization once at any given order in the energy expansion, only a finite number of operators contribute. This allows for a consistent order-by-order renormalization procedure.} built from the light fields and organized as an expansion in powers of $E/\Lambda$, where $E$ is the characteristic energy of the physical problem under consideration.
	
%	The effects of the heavy particles are encoded in the coefficients of the resulting EFT Lagrangian in such a way that any observable  
	
%	In this chapter we are interest in how to compute these coefficients given a UV theory. Therefore, we are going to discuss a systematic matching m
	
	
	
	
	
	
	%Effective field theories (EFT’s) provides a crucial framework in modern physics by allowing a
	%simplified yet accurate description of complex systems. Instead of consider all possible degrees of freedom, EFT’s focus only on those that are relevant at a given energy scale or under specific conditions. By systematically integrating out high-energy or short-distance effects, EFT’s allow us to derive low-energy approximations that retain the essential physics of the system. This approach is particularly powerful in quantum field theory, where interactions can span a wide range of scales, enabling an effective separation of phenomena based on their characteristic energy regimes.
	
	%\textcolor{red}{This space is designed to the introduction of this section. Here i must discuss: whats an effective theory, why to use it and how to obtain, i.e, present the need of a matching method. Furthermore, it must contain a description of the following sections}

	
	\section{Brief review on essential elements of path integral formalism}
	
	Theoretical predictions of physical observables play a crucial role in physics, as they provide the primary means of experimentally testing the validity of a theory. In the context of Quantum Field Theory, observable quantities such as scattering cross sections and decay rates are directly related to the elements of the $S$-matrix, which can be computed from the $n$-point correlation functions of the theory using the LSZ reduction procedure \cite{itzykson2006quantum}. Within the path integral formalism, all such $n$-point correlation functions are encoded in the generating functional\cite{Peskin:1995ev,Ryder:1985wq},
	
	
	\begin{align}
		Z[J] = \int D\phi e^{i\int d^d x \left[\mathcal{L}(\phi) + J_r(x)\phi^r(x) \right]},
	\end{align}
	
	and they can be obtained by taking functional derivatives with respect to the sources $J_r(x)$:
	
	\begin{align}
		G^{(n)}(x_1,...,x_n) =\frac{\int D\phi \ \phi^r(x_1)...\phi^s(x_n) e^{i\int d^d x \mathcal{L}(\phi)}}{\int D\phi  \ e^{i\int d^d x \mathcal{L}(\phi)} } = \frac{(-i)^n}{Z[0]} \frac{\delta^n}{\delta J_r(x_1)...\delta J_s(x_n)} Z[J]\bigg\vert_{J = 0}
	\end{align}
	
	\noindent where is $\mathcal{L}(\phi)$ is the Lagrangian density of the theory and $\phi^r(x)$ are the fields. These fields are not necessarily scalars, they might even be Dirac fields.
	
	The generating functional corresponds to the sum of all vacuum-vacuum amplitudes in the presence of the source $J$. In terms of Feynman diagrams, this sum includes contributions from both connected and disconnected diagrams. However, diagrams that differ only by permutations of vertices within the same or across different connected subdiagrams are not counted as distinct. For a generic disconnected diagram composed of $N$ connected components, its contribution to $Z[J]$ is given by the product of the contributions from each connected component, divided by the number of permutations of the connected components, $N!$. As a result, the sum of all diagrams simplifies to\cite{Peskin:1995ev}:
	
	\begin{align}
		Z[J] = \sum_{N=0}^{\infty}\frac{1}{N!}(iW[J])^N = e^{iW[J]} \implies W[J] = -i\ln{Z[J]} \label{eq:ZW}
	\end{align}
	
	\noindent where $W[J]$ is the generating functional for all fully connected correlation functions. In other words, $W[J]$ represents the sum of all fully connected Feynman diagrams contributing to the vacuum-to-vacuum amplitude.
	
	For many applications, such as renormalization and matching, it is more convenient go one step further and to work with the sum of all connected one-particle-irreducible (1PI) diagrams. These diagrams are connected Feynman diagrams that cannot be separated into disconnected parts by cutting a single internal propagator. To proceed and designate the expression for the generating functional of the 1PI correlation functions, we first introduce the vacuum expectation value (VEV) of the field $\phi^r(x)$ in the presence of the external source $J_r(x)$. This VEV, commonly referred to as the classical field, is defined as the first functional derivative of $W[J]$ with respect to the source:
	

	
	
	\begin{align}
		\phi_c^r(x) \equiv \frac{\delta W[J]}{J_r(x)} = -i\frac{\delta\ln{Z[J]}}{\delta J_r(x)} \label{eq:phic}
	\end{align}
	
	
	Using the classical field $\phi^r_c(x)$, we define the generating functional for 1PI correlation functions, also known as the 1PI quantum action, through a Legendre transformation of $W[J]$, with $J_r(x)$ and $\phi^r_c(x)$ treated as conjugate variables\cite{Weinberg:1996kr,Peskin:1995ev}:
	
	
	%The generating functional for 1PI correlation functions, also referred to as the 1PI quantum action, is defined as a Legendre transform of $W[J]$ with $J_r(x)$ and \cite{Peskin:1995ev, Weinberg:1996kr}:
	
	
	%For many purposes, such as renormalization and matching, its useful to go one step further, and instead of working with $W[J]$ we use the sum of all connected one-particle-irreducible (1PI) diagrams, which correspond to connected Feynman diagrams that cannot be separated into disconnected parts by cutting a single internal propagator. The generating functional for the 1PI correlation functions, also known as 1PI quantum action, is defined as a Legendre transformation of $W[J] = -i\ln{Z[J]}$\cite{Peskin:1995ev, Weinberg:1995mt}:
	
	\begin{align}
		\Gamma[\phi_c] = W[J] - \int d^d x \phi^r_c(x) J_r(x) = -i\ln{Z[J]}- \int d^d x \phi^r_c(x) J_r(x)\label{eq:1PIGF}
	\end{align}
	
	
	\subsection{Background field method}
	
	Although we previously defined the one-particle irreducible (1PI) generating functional using the vacuum expectation value of the field, this is not the only viable approach for computing it. In particular, to derive the EFT Lagrangian via matching in the upcoming sections, we will employ the Background Field Method (see \cite{itzykson2006quantum,Abbott:1981ke,PhysRevD.9.1686}) to evaluate the 1PI quantum effective action.
	
	In the Background Field Method, the field is split into a classical background field $\phi_b$ and a quantum fluctuation $\phi'$ component, as follows:
	
	
	\begin{align}
		\phi^r(x) = \phi^r_b(x) + \phi'^r(x)
	\end{align}
	
	\noindent where background field is defined to satisfy the classical equation of motion in the presence of the external source: 
	
	\begin{align}
		\frac{\delta\mathcal{L}}{\delta \phi^r}\bigg\vert_{\phi^r = \phi^r_b} + J_r(x) = 0 
	\end{align}
	
	
	We now expand the Lagrangian, including the source term, around the background field: 
	
	\begin{align}
		\mathcal{L}[\phi] + J_r\phi^r = \mathcal{L}[\phi_b] + J_r\phi_b^r - \frac{1}{2}\phi'^r \frac{\delta^2\mathcal{L}}{\delta\phi^r\delta\phi^l}\bigg\vert_{\phi = \phi_b}\phi'^l + ...
	\end{align}
	
	\noindent where there is no linear term in $\phi'$ because the background field satisfies the equation of motion. Substituting this expansion into the generating functional yields: 
	

	\begin{align}
		Z[J] = e^{i\int d^dx\left(\mathcal{L}[\phi_b] + J_r\phi_b^r\right)} \int D\phi \exp\left\{-\frac{i}{2}\int d^dx \phi'^r \frac{\delta^2\mathcal{L}}{\delta\phi^r\delta\phi^l}\bigg\vert_{\phi = \phi_b}\phi'^l + ... \right\}
	\end{align}
	
	In this work, we are interested only in the one-loop contributions, which correspond to the terms quadratic in the quantum fluctuations. This can be understood diagrammatically: the background field enters as tree-level lines, while the quantum fluctuation propagates internally and forms loops. Thus, terms higher than quadratic in $\phi'$ contribute only at two-loop order or beyond. Truncating at one-loop order, we obtain: 
	
	\begin{align}
		Z[J] &\approx e^{i\int d^dx\left(\mathcal{L}[\phi_b] + J_r\phi_b^r\right)} \int D\phi \exp\left\{-\frac{i}{2}\int d^dx \phi'^r \frac{\delta^2\mathcal{L}}{\delta\phi^r\delta\phi^l}\bigg\vert_{\phi = \phi_b}\phi'^l \right\}\nonumber\\
		&= N e^{i\int d^dx\left(\mathcal{L}[\phi_b] + J_r\phi_b^r\right)} \left[\det\left(-\frac{\delta^2\mathcal{L}}{\delta\phi^r\delta\phi^l}\bigg\vert_{\phi = \phi_b}\right)\right]^{-c_s}
	\end{align}
	
	\noindent where we have used the standard result for a Gaussian path integral \cite{Peskin:1995ev}. Here, $c_s = \frac{1}{2}$ for bosonic fields and $c_s = -1$ for fermionic fields, and $N$ is an irrelevant normalization constant.
	
	With this result we can calculate the generating functional of all connected $n$-point correlation functions using Eq. (\ref{eq:ZW}):
	
	\begin{align}
		W[J] = \int d^dx\left(\mathcal{L}[\phi_b] + J_r\phi_b^r\right) + i c_s \ln\det\left(-\frac{\delta^2\mathcal{L}}{\delta\phi^r\delta\phi^l}\bigg\vert_{\phi = \phi_b}\right)\label{eq:WJ}
	\end{align}
	
	Subtracting $\int d^dx J_r\phi_b^r$ at each side and defining:
	
	\begin{align}
		\Gamma[\phi_b] \equiv W[J] - \int d^dx J_r\phi_b^r = \int d^dx \mathcal{L}[\phi_b] + i c_s \ln\det\left(-\frac{\delta^2\mathcal{L}}{\delta\phi^r\delta\phi^l}\bigg\vert_{\phi = \phi_b}\right)\label{eq:1PIB}
	\end{align}
	
	This object is equivalent, up to one-loop order, to the $1PI$ generating functional defined by Eq. (\ref{eq:1PIGF}). To see this, consider the relationship between the vacuum expectation value $\phi^r_c$ the background field $\phi^r_b$, obtained by functional differentiating Eq. (\ref{eq:WJ}) with respect to the source:
	
	
	\begin{align}
		%\implies \frac{\delta W[J]}{\delta J_r} = \frac{\delta \phi^l_b}{\delta J_r} \frac{\delta}{\delta \phi^l_b} \int d^dx\mathcal{L}[\phi_b] +  \phi_b^r + i c_s \frac{\delta}{\delta J}\ln\det\left(-\frac{\delta^2\mathcal{L}}{\delta\phi^r\delta\phi^l}\bigg\vert_{\phi = \phi_b}\right)
		 \frac{\delta W[J]}{\delta J_r} =  \phi_b^r + i c_s \frac{\delta}{\delta J}\ln\det\left(-\frac{\delta^2\mathcal{L}}{\delta\phi^r\delta\phi^l}\bigg\vert_{\phi = \phi_b}\right)
	\end{align}
	
	By Eq.~(\ref{eq:phic}), the left-hand side equals $\phi_c^r$, leading to: 
	
	\begin{align}
		\phi^r_c = \phi^r_b + (\text{quantum corrections})
	\end{align}
	
	This result shows that the vacuum expectation value $\phi^r_c$ differs from the background field $\phi^r_b$ by quantum corrections that start at one-loop order. Therefore, to leading order, we can identify $\phi^r_b \approx \phi^r_c$. Moreover, since the background field satisfies the classical equation of motion, the action $S[\phi_b] = \int d^d x, \mathcal{L}[\phi_b]$ is stationary at $\phi_b$. As discussed in \cite{itzykson2006quantum}, this implies: 
	
	%This result show us that the vacuum expectation value is equal to the background field plus quantum corrections of at least one-loop order. Therefore, we can say that $\phi_b^r$ is equal to $\phi_c^r$ at leading order. Moreover, since $\phi_b$ is the solution of the equation of motion, i.e, the action $S[\phi_b] = \int d^dx \mathcal{L}[\phi_b]$ is stationary at $\phi_b$, we have that\cite{itzykson2006quantum}:
	
	\begin{align}
		S[\phi_c] + J_r\phi^r_c - S[\phi_b] - J_r\phi_b^r = \text{Second order loop corrections}
	\end{align}
	
	As a result, since we are only interested in the 1PI quantum action up to one-loop order, functional $\Gamma[\phi_b]$ defined in Eq.(\ref{eq:1PIB}), is equivalent to the $1PI$ generating functional defined in Eq. (\ref{eq:1PIGF}). Consequently, we can use the Background Field Method to compute the $1PI$ generating functional to this order.
	
	
	
	
	
	
	
	
	
	
	
	
	
	
	
	
	
	
	
	
	
	
	
	
	
	%\newpage
	
	%\begin{align}
	%	\mathcal{L}_R = \mathcal{L} + \delta\mathcal{L}
	%\end{align}
	
	%\begin{align}
	%	\frac{\delta\mathcal{L}}{\delta \phi^r}\bigg\vert_{\phi^r = \phi^r_b} + J_r(x) = 0 
	%\end{align}
	
	%\begin{align}
	%	\frac{\delta\mathcal{L}}{\delta \phi^r}\bigg\vert_{\phi^r = \phi^r_b} + J'_{r}(x) = 0 
	%\end{align}
	
	
	%\begin{align}
	%	J_r(x) = J_r^1(x) + \delta J_r(x)
	%\end{align}
	
	%\begin{align}
	%	Z[J] = \int D\phi e^{i\int d^dx \left(\mathcal{L}[\phi] + J^1_r\phi^r\right)} e^{i\int d^dx \left(\delta\mathcal{L}[\phi] + \delta J_r\phi^r\right)} 
	%\end{align}
	
	%\begin{align}
	%	\phi^r(x) = \phi^r_b(x) + \phi'^r(x)
	%\end{align}
	
	
	
	
	
	
	
	

	%\noindent where $\phi_b(x)$, the so-called classical or background field, is a solution to the classical equations of motion, i.e, $\frac{\delta\mathcal{L}}{\delta \phi^r}\big\vert_{\phi^r = \phi_b^r} + J = 0$
	
	%begin{align}
	%	\phi_b^r(x) = \frac{\delta\ln{Z[J]}}{\delta J_r(x)}
	%\end{align}
	
	
	
	
	
	
	
	
	
	%A scalar field theory is chosen for this discussion due to its simplicity. However, the generalization to other types of fields follows a similar structure with only minor modifications.
	
	
	%Functional differentiating $n$ times $Z[J]$ with respect to the source we obtain the $n$-point correlation function:
	
	%\begin{align}
	%	G^{(n)}(x_1,...,x_n) =\frac{\int D\phi \ \phi(x_1)...\phi(x_n) e^{i\int d^d x \mathcal{L}(\phi)}}{\int D\phi  \ e^{i\int d^d x \mathcal{L}(\phi)} } = \frac{(-i)^n}{Z[0]} \frac{\delta^n}{\delta J(x_1)...\delta J(x_n)} Z[J]\bigg\vert_{J = 0}
	%\end{align}
	
	
	
	
	
	%To compute these correlation functions in practice, it is convenient to use the diagrammatic approach provided by Feynman diagrams and Feynman rules \cite{Peskin:1995ev}. In the above definition, the $n$-point correlation function is expressed as a sum of both connected and disconnected diagrams. However, any disconnected diagram can be expressed as a product of fully connected diagrams associated with $n'$-point correlation functions, where $n' < n$. Consequently, disconnected diagrams do not represent new physical processes and do not require explicit calculation. For this reason, it is more practical to focus only on connected $n$-point correlation functions.
	
	%All fully connected $n$-point correlation functions can also be encoded in the generating functional $W[J]$, defined as \cite{Srednicki:2007qs}:
	
	%\begin{align}
	%	W[J] = -i\ln{Z[J]}
	%\end{align}
	
	
	
	%\begin{equation}
	%	\begin{tikzpicture}[baseline=-0.09cm]
	%		\begin{feynman}[every blob={/tikz/fill=gray!500,/tikz/inner sep=2pt}]
				
	%			\vertex[blob] (a) {D};
	%			\vertex[above left=1.5cm of a] (b){};
		%		\vertex[above right=1.5cm of a] (c) {};
		%		\vertex[below right=1.5cm of a] (d) {};
		%		\vertex[below left=1.5cm of a] (e) {};
		%		\diagram* {
		%		(a) -- (b), 
		%		(a) -- (c), 
		%		(a) -- (d), 
		%		(a) -- (e),  
		%		};
		%	\end{feynman}
		%\end{tikzpicture} 
	%\end{equation}
	
	 
		
	
	
	
	
	
	
	
	
	
	
	
	
	
	
	
	
	
	
	
	
	
	
	
	
	
	
	
	
	
	
	
	
	
	
	
	
	
	
	
	
	
	\section{Matching for real scalar fields}

		
	%(First version) In this section we will discuss a systematic method to obtain the Lagrangian of the EFT, up to one loop order, from the complete UV theory ($\mathcal{L}_{UV}$). For now, we will restrict our discussion to scalar fields and extend the method to fermions and Gauge bosons only at the end. Thus, consider an UV Lagrangian $\mathcal{L}_{UV}[\phi,\Phi]$ for a set of heavy scalar fields $\Phi$ of masses $\{M_i\}$ and a set of light fields $\phi$ of masses $\{m_{j}\}<<\{M_i\}$. As we discussed early, for energy scales much smaller than the heavy field masses, i.e, $\Lambda << \{M_i\}$, only the light fields are degrees of freedom.
	
	%(Second version) Consider a UV-complete theory with a set of light fields $\phi$ and heavy fields $\Phi$ with Lagrangian $\mathcal{L}_{UV}(\phi,\Phi)$. We are interest in the low energy EFT where $\Phi$ are integrated out. The EFT is valid at energy scales smaller than the heavy fields mass $M_i$ and the interactions of the light fields, at these scale, can be described by a local Lagrangian $\mathcal{L}_{EFT}(\phi)$. Thus, our objective in this section is to discuss a systematic method to integrate out heavy fields using the path integral formalism elements that we presented in the previous section. Although we are restricting ourselves to scalar fields, we will extend this method to fermions and Gauge bosons later.
	
	Consider a UV-complete theory with a set of light ($\phi$) and heavy ($\Phi$) scalar fields whose interactions are described by the $\mathcal{L}_{UV}$ Lagrangian density. As discussed in the previous section, every correlation function of this theory are encoded in generating functional:
	
	\begin{align}
		Z_{UV}[J_\Phi,J_\phi] =  \int D\Phi D\phi e^{i\int d^d x \left[\mathcal{L}_{UV}(\phi,\Phi) + J_\phi \phi + J_\Phi \Phi\right]}.
	\end{align}
	
	At energy scales below the heavy fields mass $M$, the light fields are the only relevant degrees of freedom. Consequently, all physical processes at these scales must be described by correlation functions involving only light external fields, which are generated by setting $J_\Phi = 0$ in the UV generating functional, i.e.,
	
	\begin{align} 
		Z_{UV}[J_\Phi = 0, J_\phi] = \int D\Phi D\phi e^{i\int d^d x \left[\mathcal{L}_{UV}(\phi,\Phi) + J_\phi \phi \right]}. 
	\end{align}
	
	On the other hand, at low energies, the same physics can be described using an EFT, where the heavy fields $\Phi$ are integrated out. In this approach, the interactions of the light fields are described by a local Lagrangian density $\mathcal{L}_{EFT}(\phi)$, with the corresponding EFT generating functional given by:
	
	\begin{align}
		Z_{EFT}[J_\phi] = \int  D\phi e^{i\int d^d x \left[\mathcal{L}_{EFT}(\phi) + J_\phi \phi \right]}.
	\end{align}
	
	Since both the UV theory and its EFT describe the same low-energy physics, all observable predictions must agree. A natural first attempt at ensuring this agreement is to impose the matching condition:
	
	\begin{align}
		Z_{EFT}[J_\phi] = Z_{UV}[J_\Phi = 0,J_\phi]
	\end{align}

	However, this condition is stronger than necessary, as it requires equality not only for on-shell but also for off-shell correlation functions. In practice, we only need to ensure that the $S$-matrix elements match between the two theories, meaning that on-shell correlation functions must be identical. Furthermore, these generating functionals are highly complex objects, making direct matching impractical.
	
	%However, this statement is somewhat stronger than we need, as it ensures the equality of on-shell as well off-shell correlation functions and what we really need is the equality of the $S$-matrix elements in the UV and EFT theories, i.e, the on-shell correlation functions. Furthermore, those generating functionals are highly complex objects. Therefore, for practical reasons, the matching condition is imposed over the one-light-particle-irreducible (1LPI) generating functional in the UV-complete theory and the 1PI generating functional of the EFT:
	
	To address these issues, the matching condition is instead imposed on the one-light-particle-irreducible (1LPI) generating functional in the UV theory and the 1PI generating functional of the EFT\cite{Zhang_2017}:
	
	\begin{align}
		\Gamma_{EFT}[\phi_b] = \Gamma_{L,UV}[\phi_b]\label{eq:MT_cd}
	\end{align}
	
	%\noindent where $\phi_b$ is the classical background field. However, in practice, it is generally not possible to find an effective Lagrangian $\mathcal{L}_{EFT}(\phi)$ that exactly satisfies this equation. Instead, we typically impose this equality only up to a prescribed order in the loop expansion and/or in the Taylor expansion in powers of $\frac{1}{M}$. In particular, throughout this chapter, we will restrict the matching condition to one-loop order.
	
	
	\noindent where $\phi_b$ is the classical background field. However, in practice, it is generally not possible to find an effective Lagrangian $\mathcal{L}_{EFT}(\phi)$ that exactly satisfies this equation. Instead, we impose the matching condition perturbatively, order-by-order in the loop expansion. In particular, throughout this chapter, we will restrict the matching condition to one-loop order, which leads to the conditions:
	
	\begin{align}
		\Gamma^{\text{tree}}_{EFT}[\phi_b] &= \Gamma^{\text{tree}}_{L,UV}[\phi_b],\label{eq:MT_C0}\\ \Gamma^{\text{1-loop}}_{EFT}[\phi_b] &= \Gamma^{\text{1-loop}}_{L,UV}[\phi_b]  \label{eq:MT_C1}
	\end{align}
	
	\noindent where this is imposed for a renormalization scale $\mu = M$.
	
	
	
	

	
	\subsection{Calculating the 1LPI generating functional}
	
	To compute $\Gamma_{L,UV}[\phi_b]$, we begin with the generating functional of the UV-complete theory, setting the source for the heavy field to zero:
	
	
	\begin{align} 
		Z_{UV}[J_\Phi = 0, J_\phi] = \int D\Phi D\phi e^{i\int d^d x \left[\mathcal{L}_{UV}(\phi,\Phi) + J_\phi \phi \right]}. \label{eq:Z_L}
	\end{align}

	
	Next, we decompose each field into a classical background configuration, labeled by a subscript ``b'', and a quantum fluctuation, labeled by a prime. That is, we write:
	
	
	\begin{align}
		\phi = \phi_b + \phi', \ \ \ \Phi = \Phi_b + \Phi'
	\end{align}
	
	The background fields satisfy the classical equations of motion (EOM) in the presence of the sources, which are given by \cite{Zhang_2017}:
	
	\begin{align}
		\frac{\delta \mathcal{L}_{UV}(\phi,\Phi)}{\delta \phi}\bigg\vert_{\phi = \phi_b} + J_{\phi} = 0, \ \ \ \frac{\delta \mathcal{L}_{UV}(\phi,\Phi)}{\delta \Phi}\bigg\vert_{\Phi = \Phi_b} = 0\label{eq:EOM}
	\end{align}
	
	\noindent where we already set $J_\Phi = 0$ due our interest in energy scales smaller than the heavy fields mass.
	
	
	%Diagrammatically, the background part corresponds to tree lines in Feynman graphs while lines inside loops arises from quantum fluctuation fields, this means that terms higher than quadratic in the quantum fields yield vertices that can only appear in the diagrams at higher loop orders. Therefore, in order to achieve the matching up to 1-loop order, we can Taylor expand the UV Lagrangian plus the light source term around $\phi = \phi_b$, i.e, $\phi' = 0$,  and consider only up to terms quadratic in the quantum fluctuations fields:
	
	%Diagrammatically, the background fields correspond to tree lines in Feynman diagrams, while internal lines within loops arise from quantum fluctuations. As a result, terms higher than quadratic in the quantum fields generate interaction vertices that contribute only at higher-loop orders. To perform the matching up to one-loop order, we decompose the fields and expand the UV Lagrangian, along with the light field source term, around $\phi = \phi_b$, retaining only terms up to quadratic order in the quantum fluctuations.
	

	Diagrammatically, the background fields correspond to tree lines in Feynman diagrams, while internal lines within loops arise from quantum fluctuations. As a result, terms higher than quadratic in the quantum fields generate interaction vertices that contribute only at higher-loop orders. Since we aim to perform the matching up to a specific order in the loop expansion, we can isolate the relevant contributions to $Z_{UV}[J_\Phi = 0, J_\phi]$ by counting the powers the quantum fluctuations fields in the argument of the exponential. 
	
	
	A systematic approach to achieving this is to Taylor expand $\mathcal{L}_{UV}[\Phi,\phi]+ J_\phi \phi$ around $\phi = \phi_b$, decompose the fields into background and fluctuation components, and truncate the expansion at the desired order in the quantum fluctuations, or equivalently in the desired loop order. Following this procedure, we obtain:

	

	\begin{align}
		\mathcal{L}_{UV}[\Phi,\phi]+ J_\phi \phi  = \mathcal{L}_{UV}[\Phi_b,\phi_b] + J_\phi \phi_b  - \frac{1}{2} 
		\begin{pmatrix}
			\Phi'^T & \phi'^T
		\end{pmatrix}
		\mathcal{Q}_{UV}[\Phi_b,\phi_b]
		\begin{pmatrix}
			\Phi' \\ \phi'
		\end{pmatrix} + ...\label{eq:T_exp}
	\end{align}
	
	\noindent where the absence of a linear term in $\phi'$ or $\Phi'$ follows from the fact that the background fields satisfy the equations of motion (Eq. (\ref{eq:EOM})). Furthermore, the quadratic operator $\mathcal{Q}_{UV}[\Phi_b,\phi_b]$, also know as fluctuation operator, is defined as:
	
	
	\begin{align}
		\mathcal{Q}_{UV}[\Phi_b,\phi_b] = 
		\begin{pmatrix}
			\Delta_H &X_{HL}\\
			X_{LH} &\Delta_L
		\end{pmatrix} = 
		\begin{pmatrix}
			-\frac{\delta^2 \mathcal{L}_{UV}}{\Phi'^2}[\Phi_b,\phi_b] &-\frac{\delta^2 \mathcal{L}_{UV}}{\Phi'\phi'}[\Phi_b,\phi_b]\\
			-\frac{\delta^2 \mathcal{L}_{UV}}{\phi'\Phi'}[\Phi_b,\phi_b] &-\frac{\delta^2 \mathcal{L}_{UV}}{\phi'^2}[\Phi_b,\phi_b]
		\end{pmatrix}  \label{eq:Q_UV}
	\end{align}

	At zeroth order, the expansion depends only on the background fields and contributes to the tree-level generating functional. Therefore, truncating the series in the first term and substituting in Eq. (\ref{eq:Z_L}) give us:
	
	\begin{align}
		Z^{\text{tree}}_{UV}[J_\Phi = 0, J_\phi] =  N e^{i\int d^d x \left[\mathcal{L}_{UV}[\Phi_b,\phi_b] + J_\phi \phi_b \right]}
	\end{align}

	\noindent where $N$ is some renormalization constant which is irrelevant in practical applications. 
	
	To include one-loop effects, we must retain the quadratic term in the fluctuation fields. Substituting Eq. (\ref{eq:T_exp}) into Eq. (\ref{eq:Z_L}) and truncating at quadratic order, we obtain:
	
	
	\begin{align}
		Z_{UV}[J_\Phi = 0, J_\phi] =  Z^{\text{tree}}_{UV}[J_\Phi = 0, J_\phi]\int D\phi'D\Phi'\exp{-\frac{i}{2}\int d^d x \left[
			\begin{pmatrix}
				\Phi'^T & \phi'^T
			\end{pmatrix}
			\mathcal{Q}_{UV}[\Phi_b,\phi_b]
			\begin{pmatrix}
				\Phi' \\ \phi'
			\end{pmatrix} \right]}\label{eq:ZZ}
	\end{align}
	
	%To proceed, we could formally evaluate the integral via Gaussian integration. However, the presence of mixing terms with heavy-light quantum fields in $\mathcal{Q}_{UV}[\Phi_b,\phi_b]$ (equivalently, one-loop diagrams containing both heavy and light lines), makes it necessary to first rewrite the fluctuation operator in Eq. (\ref{eq:Q_UV}) in an equivalent block-diagonal form, since the fields can have different statistics. A possibility of doing it is to perform shifts, with unit Jacobian determinant, in the quantum fields. We can choose a field transformation that transfers the information in the mixing to the heavy fields block ($\Delta_H$), while leaving $\Delta_L$ untouched. This has the advantageous that all heavy particle effects in the one-loop order comes from the evaluation of the path integral over the heavy fields and, as we will see later, the contribution of the pure light loops cancels out when imposing the matching condition, once they are the same in UV and EFT theories.
	
	Although the path integral could, in principle, be evaluated using Gaussian integration, it is useful to first bring the fluctuation operator into a block-diagonal form, decoupling the heavy and light path integrals. This can be achieved by performing shifts, with unit Jacobian determinant, in the quantum fields. Specifically, we can choose a field transformation which transfers the information in the mixed components\footnote{Equivalently, one-loop diagrams containing both heavy and light lines.} ($X_{LR}$ and $X_{RL}$) to the heavy fields block ($\Delta_H$), while leaving $\Delta_L$ untouched. This approach simplifies the matching procedure by ensuring that all heavy particle effects at one-loop order are encapsulated within the result of the heavy field integral. Additionally, since pure light loops contribution must trivially appears in both the UV theory and the EFT, it cancels out when imposing the matching conditions. Furthermore, this decoupling is particularly crucial when the heavy and light fields obey different statistics, ensuring a consistent treatment of their fluctuations.

	

	

	%Although the path integral could, in principle, be evaluated using Gaussian integration, the presence of mixing terms between heavy and light quantum fields in $\mathcal{Q}_{UV}[\Phi_b,\phi_b]$ introduces a complication. These terms correspond to one-loop diagrams involving both heavy and light fields, and since the fields may obey different statistics, it is convenient to first bring the fluctuation operator into a block-diagonal form. This can be achieved by performing shifts, with unit Jacobian determinant, in the quantum fields. We can choose a field transformation which transfers the information in the mixing to the heavy fields block ($\Delta_H$), while leaving $\Delta_L$ untouched. This has the advantageous that all heavy particle effects in the one-loop order comes from the evaluation of the path integral over the heavy fields. Furthermore, as we will see later, the contributions of the pure light loops cancels out when imposing the matching condition, once they are the same in UV and EFT theories.
	
	
   A suitable choice for the fields transformation that brings the fluctuation operator into the desired block-diagonal form is:
	
	\begin{align}
		V = \begin{pmatrix}
			\mathds{1} &0\\
			-\Delta_L^{-1} X_{LH} &\mathds{1}
		\end{pmatrix} 
	\end{align} 

	Applying this transformation to the fluctuation operator, we obtain:


	\begin{align}
		V^\dagger \mathcal{Q}_{UV}[\hat{\Phi}_c[\phi_b],\phi_b] V = \begin{pmatrix}
			\Delta_H - X_{HL}\Delta_L^{-1}X_{LH} &0\\
			0 & \Delta_L
		\end{pmatrix}
	\end{align}
	
	The functional integrals in Eq. (\ref{eq:ZZ}) can now be carried out easily\cite{Zhang_2017}:
	
	\begin{align}
		Z_{UV}[J_\Phi = 0, J_\phi] &=   Z^{\text{tree}}_{UV}[J_\Phi = 0, J_\phi]\int D\Phi' \exp{-\frac{i}{2}\int d^d x \Phi'^T \left(\Delta_H - X_{HL}\Delta_L^{-1}X_{LH}\right)\Phi'} \nonumber\\ &\hspace{8cm}\int D\phi' \exp{-\frac{i}{2}\int d^d x \phi'^T \Delta_L \phi'}\nonumber\\
		&= N' Z^{\text{tree}}_{UV}[J_\Phi = 0, J_\phi] \left[\det \left(\Delta_H - X_{HL}\Delta_L^{-1}X_{LH}\right)\right]^{-\frac{1}{2}} \left[\det \Delta_L\right]^{-\frac{1}{2}}\label{eq:GI}
	\end{align}
	
	%\noindent with $c_h$ and $c_l$ equals to $1/2$ or $-1$ depending on the bosonic or fermionic nature of the heavy and light fields. Furthermore $N'$ is another renormalization constant with irrelevant physical effects.


	\noindent where $N'$ is another renormalization constant with irrelevant physical effects.


	From $Z_{UV}[J_\Phi = 0, J_\phi]$, we obtain the generating functional of connected correlation functions, $W_{UV}[J_\Phi = 0, J_\phi]$, by taking its natural logarithm and multiplying by $-i$. Finally, we obtain the 1LPI quantum action by performing a Legendre transform on $W_{UV}$, treating $\phi_b$ and $J_\phi$ as conjugate variables. Up to an irrelevant constant, this yields:
	
	
	
	\begin{align}
		\Gamma_{L,UV}[\phi_b] &= -i\ln{Z_{UV}[J_\Phi = 0,J_\phi]} - \int d^d x J_\phi(x) \phi_b(x) \nonumber\\
		&= -i\ln{Z^{\text{tree}}_{UV}[J_\Phi = 0,J_\phi]} - \int d^d x J_\phi(x) \phi_b(x)+ \frac{i}{2}  \ln\det \left(\Delta_H - X_{HL}\Delta_L^{-1}X_{LH}\right) + \frac{i}{2} \ln\det \Delta_L
		%&= \int d^d x \mathcal{L}_{UV}[\Phi_b,\phi_b] + \frac{i}{2} \ln{\text{det}\mathcal{Q}_{UV}[\Phi_b,\phi_b]}\label{eq:G_1LPI}
	\end{align}
	
	To explicitly separate contributions at different loop orders, we define:
	
	\begin{align}
		\Gamma_{L,UV}[\phi_b] \equiv \Gamma^{\text{tree}}_{L,UV}[\phi_b] + \Gamma^{\text{1-loop}}_{L,UV}[\phi_b]
	\end{align}

	\noindent where:
	
	\begin{align}
		\Gamma^{\text{tree}}_{L,UV}[\phi_b] &= -i\ln{Z^{\text{tree}}_{UV}[J_\Phi = 0,J_\phi]} - \int d^d x J_\phi(x) \phi_b(x) = \int d^d x \mathcal{L}_{UV}[\Phi_b,\phi_b],\label{eq:G_UV_tree}\\
		\Gamma^{\text{1-loop}}_{L,UV}[\phi_b] &= \frac{i}{2} \ln\det \left(\Delta_H - X_{HL}\Delta_L^{-1}X_{LH}\right) + \frac{i}{2} \ln\det \Delta_L\label{eq:G_UV_one}\
	\end{align}


	Notice that on the right-hand side of the 1LPI equation, the functional dependence is explicitly in $\phi_b$. whereas on the left-handed side, it appears to depend also on $\Phi_b$. However, since we have set the source for the heavy field to zero, $\Phi_b$ satisfies the equation of motion given by Eq. (\ref{eq:EOM}), making it a function of $\phi_b$ alone. This allows us to eliminate the explicit dependence on $\Phi_b$ in the 1LPI quantum action by solving its equation of motion with $\phi_b$ treated as the background.

	
	%This substitution must be handled carefully, as the solution $\Phi_b\vert_{J_\Phi = 0}$ may be non-local due to the presence of inverse differential operators and a Lagrangian density is, by definition a sum of local operators. Therefore, we expand this solution in powers of the inverse heavy mass and truncate at a finite order. Furthermore, to distinguish the heavy field from its solution in the absence of the source, we define:
	
	
	
	The solution $\Phi_b\vert_{J_\Phi = 0}$ may be non-local due to the presence of inverse differential operators, whereas the Lagrangian density must consist in a sum of local operators. Therefore, this substitution must be handled carefully. When the solution is non-local, we expand it in inverse powers of the heavy mass and truncate at a finite order to ensure locality. Additionally, to clearly distinguish the heavy field from its solution in the absence of the source, we define:
	 
	%Notice that in the right side of the 1LPI quation we are stating that it is a functional of $\phi_b$ and the left side shows dependence on $\Phi_b$. However,since we have set the source for the heavy field to zero, the background heavy field \(\Phi_b\) satisfies the equation of motion (Eq. (\ref{eq:EOM})), it becomes a function of $\phi_b$ alone. This allows us to eliminate the dependence on the background heavy fields in the 1LPI effective action by solving their equations of motion with the light fields treated as background. This substitution must be carried out carefully, as the solution $\Phi_b\vert_{J_\Phi = 0}$ may be non-local due to the presence of inverse differential operators. In such cases, we perform an expansion in powers of the inverse heavy mass and truncate at a finite order to ensure that the Lagrangian density remains a local object.  
	
	
	
	\begin{align}  
		\Phi_c[\phi_b] \equiv \Phi_b\vert_{J_\Phi = 0}.  
	\end{align}  
	
	After performing the expansion in $1/M$, we denote the resulting local approximation of $\Phi_c[\phi_b]$ with a hat to emphasize its locality. Therefore, using the EOM of the heavy fields in Eqs. (\ref{eq:G_UV_tree}) and (\ref{eq:G_UV_one}):
	
	\begin{align}
		\Gamma^{\text{tree}}_{L,UV}[\phi_b]  &= \int d^d x \mathcal{L}_{UV}[\hat{\Phi}_c[\phi_b],\phi_b], \label{eq:Gamma_LUV_0}\\
		\Gamma^{\text{1-loop}}_{L,UV}[\phi_b] &= \frac{i}{2} \ln\det \left(\Delta_H - X_{HL}\Delta_L^{-1}X_{LH}\right) + \frac{i}{2} \ln\det \Delta_L\ \label{eq:Gamma_LUV_1}
	\end{align}

	\noindent where we omitted the arguments $[\hat{\Phi}_c[\phi_b],\phi_b]$ of the operators in the second equation in order to keep a clear notation.
	%\begin{align}
	%	\Gamma_{L,UV}[\phi_b] =  \int d^d x \mathcal{L}_{UV}[\hat{\Phi}_c[\phi_b],\phi_b] + \frac{i}{2} \ln{\text{det}\mathcal{Q}_{UV}[\hat{\Phi}_c[\phi_b],\phi_b]}\label{eq:1LPI}
	%\end{align}


	\subsection{Calculating the 1PI generating functional of the EFT}
	
	%The procedure is very similar of what we done in the last section, however the effective theory only has the light fields as degrees of freedom. Thus we must decompose $\phi$ into the background and quantum fluctuation components, Taylor expand the EFT Lagrangian plus the source term up to one-loop order, substitute into the EFT generating functional and perform the Legendre transform it to obtain the 1PI quantum action.

	%Having obtained the 1LPI quantum action of the UV theory $\Gamma_{L,UV}[\phi_b]$ up to one-loop order, we now proceed to the calculation of the 1PI generating functional of the EFT in order to impose the matching condition. The procedure closely follows the steps taken in the previous section, so we start with the generating functional of the EFT:
	
	Having obtained the 1LPI quantum action of the UV theory, $\Gamma_{L,UV}[\phi_b]$, up to one-loop order, we now turn to the calculation of the 1PI generating functional of the EFT to impose the matching condition. The procedure closely follows the steps from the previous section. We begin with the generating functional of the EFT:
	
	\begin{align}
		Z_{EFT}[J_\phi] = \int D\phi e^{i\int d^d x \left[\mathcal{L}_{EFT}[\phi]  + J_\phi \phi\right]}\label{eq:Z_EFT}
	\end{align}

	%Here we are already considering only the relevant degrees of freedom for the energy scale of our interest. Therefore, the Lagrangian density depends only of the light field $\phi$. Since we actually want to find $\mathcal{L}_{EFT}[\phi]$ that satisfies the matching condition (Eq. (\ref{eq:MT_cd})) up to some order in loop expansion, we write the Lagrangian density distinguishing each loop order contribution:
	
	
	Here, we are working within the EFT framework and consequently only the light fields $\phi$ remains as relevant degrees of freedom. Our goal is to determine $\mathcal{L}_{EFT}[\phi]$ such that the matching condition (Eq. (\ref{eq:MT_cd})) holds up to a given loop order. To systematically organize the contributions, we write the EFT Lagrangian distinguishing each loop order:

	\begin{align}
		\mathcal{L}_{EFT}[\phi] = \mathcal{L}^{\text{tree}}_{EFT}[\phi] + \mathcal{L}^{\text{1-loop}}_{EFT}[\phi] + ...
	\end{align}

	\noindent where $\mathcal{L}^{\text{tree}}_{EFT}[\phi]$ and $\mathcal{L}^{\text{1-loop}}_{EFT}[\phi]$ contain effective operators at tree and one-loop level, respectively. Furthermore, we also split the fields $\phi$ into background and quantum fluctuations, Taylor expand the Lagrangian plus source around $\phi = \phi_b$, truncate the series at quadratic order in the quantum fluctuations, and substitute in the generating functional (Eq. (\ref{eq:Z_EFT})). Up to one-loop order, we obtain:
	

	
	\begin{align}
		Z_{EFT}[J_\phi] &= e^{i\int d^d x \left[\mathcal{L}^{\text{tree}}_{EFT}[\phi_b] + \mathcal{L}^{\text{1-loop}}_{EFT}[\phi_b] + J_\phi \phi_b\right]}\int D\phi'\exp{-\frac{i}{2}\int d^d x \left[\phi'^T \mathcal{Q}_{EFT}[\phi_b] \phi'\right]}\nonumber\\
		&=e^{i\int d^d x \left[\mathcal{L}^{\text{tree}}_{EFT}[\phi_b] + \mathcal{L}^{\text{1-loop}}_{EFT}[\phi_b] + J_\phi \phi_b\right]}\left(\det\mathcal{Q}_{EFT}[\phi_b] \right)^{-\frac{1}{2}} \label{eq:Z_EFT1}
	\end{align}


	\noindent where the quadratic operator $\mathcal{Q}_{EFT}[\phi_b]$ is:
	
	\begin{align}
		\mathcal{Q}^{\text{tree}}_{EFT}[\phi_b] \equiv -\frac{\delta^2\mathcal{L}^{\text{tree}}_{EFT}[\phi]}{\delta \phi^2}\big\vert_{\phi=\phi_b} \label{eq:Q_EFT}
	\end{align}
	
	Observe that, up to one-loop order, only $\mathcal{L}^{\text{tree}}_{EFT}[\phi]$ is considered in the computation of the quadratic operator, as the terms from $\mathcal{L}^{\text{1-loop}}_{EFT}[\phi]$ enter the calculation at two-loop order after performing the Gaussian integration. 


	Given the generating functional of the EFT, the 1PI quantum action can be calculated by the same Legendre transform of the UV case, resulting in:
	
	
	\begin{align}
		\Gamma_{EFT}[\phi_b] = \int d^d x \mathcal{L}^{tree}_{EFT}[\phi_b] + \int d^d x \mathcal{L}^{1-loop}_{EFT}[\phi_b] + \frac{i}{2} \ln{\text{det}\mathcal{Q}_{EFT}[\phi_b]}
	\end{align}
	
	Since we have carefully kept track of the contributions at each loop order, we can now explicitly identify the tree-level and one-loop terms in the 1PI quantum action:
	
	\begin{align}
		&\Gamma^{\text{tree}}_{EFT}[\phi_b] = \int d^d x \mathcal{L}^{tree}_{EFT}[\phi_b],\label{eq:Gamma_EFT_0}\\ &\Gamma^{\text{1-loop}}_{EFT}[\phi_b] = \int d^d x \mathcal{L}^{1-loop}_{EFT}[\phi_b] + \frac{i}{2} \ln{\text{det}\mathcal{Q}_{EFT}[\phi_b]} \label{eq:Gamma_EFT_1}
	\end{align}

	Notice that the one-loop contribution to the 1PI quantum action in the EFT consists of two terms, whereas in the UV theory, there is only one. This distinction arises because the EFT includes effective operators that are already of one-loop order and contribute at tree level (first term), in addition to the one-loop corrections generated from the tree-level EFT operators (second term).
	
	
	\subsection{Matching $\Gamma_{L,UV}[\phi_b]$ and $\Gamma_{EFT}[\phi_b]$}
	
	With the tree and one-loop contributions to the 1LPI generating functional of the UV theory and the 1PI quantum action of the EFT at hand, we can now impose the matching condition order by order in the loop expansion. At tree level, substituting Eqs. (\ref{eq:Gamma_LUV_0}) and (\ref{eq:Gamma_EFT_0}) into Eq. (\ref{eq:MT_C0}) yields:
	
	
	\begin{align}
		%\int d^d x \mathcal{L}^{tree}_{EFT}[\phi_b] = \int d^d x \mathcal{L}_{UV}[\hat{\Phi}_c[\phi_b],\phi_b]  \implies 
		\mathcal{L}^{tree}_{EFT}[\phi_b] = \mathcal{L}_{UV}[\hat{\Phi}_c[\phi_b],\phi_b] \label{eq:L_EFT_tree}
	\end{align}
	
	Thus, the tree-level EFT Lagrangian is simply the UV Lagrangian evaluated with the light fields set to their classical background values and the heavy fields replaced by the local expansion of the EOM solution in the absence of a heavy source.
	
	At one-loop level, substituting Eqs. (\ref{eq:Gamma_LUV_1}) and (\ref{eq:Gamma_EFT_1}) into Eq. (\ref{eq:MT_C1}) and rearranging, we obtain:

	
	\begin{align}
		\int d^d x \mathcal{L}^{1-loop}_{EFT}[\phi_b] = \frac{i}{2} \ln\det \left(\Delta_H - X_{HL}\Delta_L^{-1}X_{LH}\right) + \frac{i}{2} \ln\det \Delta_L - \frac{i}{2} \ln{\text{det}\mathcal{Q}_{EFT}[\phi_b]} \label{eq:L_1-loop_1}
	\end{align}
	

	%To proceed with the calculation of $\mathcal{L}^{1-loop}_{EFT}[\phi_b]$, we must handle both logarithmic determinant terms. Starting with the first term in the right-hand side, we can rewrite $\mathcal{Q}_{UV}$ in a block-diagonal form. Since we only need the determinant of the quadratic operator, and the determinant of a matrix product equals the product of the individual determinants, we introduce a transformation matrix $V$ with unit determinant that diagonalizes $\mathcal{Q}_{UV}$. A suitable choice for $V$ and its effect on the quadratic operator is given by:
	
	
	%\begin{align}
	%	\det(V) = \mathds{1} \implies \det\left(V^\dagger \mathcal{Q}_{UV}[\hat{\Phi}_c[\phi_b],\phi_b] V\right) = \det\left(\mathcal{Q}_{UV}[\hat{\Phi}_c[\phi_b],\phi_b]\right)
	%\end{align}
	
	
	
	
	%Thus, using this result and product properties of determinants and logarithm:
	
	

	%\begin{align}
	%	\det \mathcal{Q}_{UV}[\hat{\Phi}_c[\phi_b],\phi_b] = \det\left(	\Delta_H - X_{HL}\Delta_L^{-1}X_{LH}\right)\det\left(\Delta_L\right)
	%\end{align}
	
	%\begin{align}
	%	\frac{i}{2}\ln\det \mathcal{Q}_{UV}[\hat{\Phi}_c[\phi_b],\phi_b] =  \frac{i}{2}\ln\det V^\dagger \mathcal{Q}_{UV}[\hat{\Phi}_c[\phi_b],\phi_b] V =\frac{i}{2}\ln\det\left(	\Delta_H - X_{HL}\Delta_L^{-1}X_{LH}\right) + \frac{i}{2}\ln\det\Delta_L \label{eq:ln_Q_UV}
	%\end{align}
	
	To proceed, we must rewrite the quadratic operator of the EFT in term of elements in the first and second term in the equation above. Recall that only the tree-level part of the EFT Lagrangian contributes to $\mathcal{Q}^{\text{tree}}_{EFT}[\phi_b]$ at one-loop. Therefore, substituting Eq.(\ref{eq:L_EFT_tree}) into (\ref{eq:Q_EFT}) and applying the chain rule:

	
	%For the quadratic operator of the EFT the procedure is different. Remember that only $\mathcal{L}^{\text{tree}}_{EFT}[\phi_b]$ contributes to $\mathcal{Q}^{\text{tree}}_{EFT}[\phi_b]$ and we shown the relation between the EFT tree-level and the UV Lagrangian. This allow us to use the chain rule when substituting Eq. (\ref{eq:L_EFT_tree}) into (\ref{eq:Q_EFT}):
	
	\begin{align}
			\mathcal{Q}^{\text{tree}}_{EFT}[\phi_b] = -\left[\frac{\delta}{\delta \phi}\left(\frac{\delta \mathcal{L}_{UV}[\hat{\Phi}_c[\phi],\phi] }{\delta \phi}\right)\right]_{\phi=\phi_b} = -\left[\frac{\delta}{\delta \phi}\left(\frac{\delta \mathcal{L}_{UV}[\hat{\Phi}_c[\phi],\phi] }{\delta \phi} + \frac{\delta \hat{\Phi}_c[\phi]}{\delta \phi} \frac{\delta \mathcal{L}_{UV}[\hat{\Phi}_c[\phi],\phi]}{\delta \Phi}\right)\right]_{\phi=\phi_b}
	\end{align}
	
	Since $\Phi_c[\phi]$ satisfies the heavy-field equation of motion (Eq. (\ref{eq:EOM})), the second term on the right-hand side vanishes. Applying the chain rule again to the first term, we obtain:
	
	%Notice that the second term in the right-hand side is zero, once $\Phi_c[\phi]$ solves the heavy fields EOM (Eq. (\ref{eq:EOM})). Thus, using the chain rule again in the first term:
	
	
	\begin{align}
		\mathcal{Q}^{\text{tree}}_{EFT}[\phi_b] = -\left[\frac{\delta^2 \mathcal{L}_{UV}[\hat{\Phi}_c[\phi],\phi] }{\delta \phi^2} + \frac{\delta \hat{\Phi}_c[\phi]}{\delta \phi} \frac{\delta^2 \mathcal{L}_{UV}[\hat{\Phi}_c[\phi],\phi]}{\delta \Phi\phi'}\right]_{\phi=\phi_b} = \Delta_L + \frac{\delta \hat{\Phi}_c[\phi]}{\delta \phi}\bigg\vert_{\phi = \phi_b} X_{HL}\label{eq:Q_EFT1}
	\end{align}
	
	To further simplify, we use again the equation of motion for the heavy fields:
	
	\begin{align}
		0 &= \left[\frac{\delta}{\delta \phi} \left(\frac{\delta \mathcal{L}_{UV}[\hat{\Phi}_c[\phi]}{\delta \Phi}\right)\right]_{\phi = \phi_b} = \left[\frac{\delta^2 \mathcal{L}_{UV}[\hat{\Phi}_c[\phi],\phi] }{\delta \phi\delta \Phi} + \frac{\delta \hat{\Phi}_c[\phi]}{\delta \phi} \frac{\delta^2 \mathcal{L}_{UV}[\hat{\Phi}_c[\phi],\phi]}{\delta \Phi^2}\right]_{\phi=\phi_b} =  -X_{LH} - \frac{\delta \hat{\Phi}_c[\phi]}{\delta \phi}\bigg\vert_{\phi = \phi_b} \Delta_H 
	\end{align}
	
	Solving for $	\frac{\delta \hat{\Phi}_c[\phi]}{\delta \phi}\bigg\vert_{\phi = \phi_b}$:
	
	\begin{align}
		\frac{\delta \hat{\Phi}_c[\phi]}{\delta \phi}\bigg\vert_{\phi = \phi_b} = -X_{LH}\hat{\Delta}_H^{-1}
	\end{align}
	
	\noindent where $\hat{\Delta}_H^{-1}$ represents the local operator expansion of $\Delta_H^{-1}$. Substituting this result into Eq. (\ref{eq:Q_EFT1}), we obtain:
	
	\begin{align}
		&\mathcal{Q}^{\text{tree}}_{EFT}[\phi_b] = \Delta_L - X_{LH}\hat{\Delta}_H^{-1} X_{HL}
	\end{align}
	
	Taking the natural logarithm of the determinant:
	\begin{align}
		 - i  \ln\det\mathcal{Q}^{\text{tree}}_{EFT}[\phi_b] &= -\frac{i}{2}\ln\det\left(\Delta_L - X_{LH}\hat{\Delta}_H^{-1} X_{HL}\right)\nonumber\\
		& = - i  \left[\ln\det\Delta_L + \ln\det\left(\mathds{1} - \Delta_L^{-1}X_{LH}\hat{\Delta}_H^{-1} X_{HL}\right)\right]\nonumber\\
		& = - i \left[\ln\det\Delta_L + \ln\det\left(\mathds{1} - \hat{\Delta}_H^{-1}X_{HL}\Delta_L^{-1} X_{LH}\right)\right]\nonumber\\
		& = - i \left[\ln\det\Delta_L  - \ln\det\hat{\Delta}_H+ \ln\det\left( \hat{\Delta}_H- X_{HL}\Delta_L^{-1} X_{LH}\right)\right]\label{eq:ln_Q_EFT}
	\end{align}
	
	Now, substituting Eq. (\ref{eq:ln_Q_EFT}) into (\ref{eq:L_1-loop_1}), we obtain the one-loop contribution to the effective Lagrangian:
	
	\begin{align}
		\int d^d x \mathcal{L}^{1-loop}_{EFT}[\phi_b] = \frac{i}{2}\ln\det\hat{\Delta}_H + \frac{i}{2}\left[\ln\det\left(	\Delta_H - X_{HL}\Delta_L^{-1}X_{LH}\right) - \ln\det\left( \hat{\Delta}_H- X_{HL}\Delta_L^{-1} X_{LH}\right)\right]\label{eq:L_1_loop_1}
	\end{align}
	
	Notice that the contributions from pure light loops, i.e., $\ln\det\Delta_L$, cancels out. This is expected, as the one-loop effective operators should only encode the indirect contributions from mixed and pure heavy field loops.\textcolor{red}{(This is what i understood, but im not sure that is right)}
	
	%Notice that the contribution of pure light loops, i.e, $\ln\det\Delta_L$ cancels out. This is expected once the one-loop effective operators should contain the indirect contribution from mixed and pure heavy loops.
	
	\subsection{Application of the expansion by regions method}
	
	
	
	
	
	%Although we can compute the one-loop Lagrangian of the EFT using Eq. (\ref{eq:L_1_loop_1}), it is not practical. The natural logarithms of the determinant in the expression contain $d$-dimensional loop integrals and can be hard or even impossible to directly evaluate them exactly when there is different scales from masses and kinematical parameters. However, the full result in an expanded form can be reproduced by using the expansion by region method\cite{Smirnov:2021dkb,Jantzen_2011,Beneke_1998}, which consists in the following steps:
	
	Although Eq. (\ref{eq:L_1_loop_1}) provides a formal expression for the one-loop effective Lagrangian, directly computing it is often impractical. The presence of natural logarithms of determinants leads to $d$-dimensional loop integrals, which become difficult to evaluate exactly using standard dimensional regularization, especially when multiple mass and kinematic scales are involved. A more effective approach is the expansion by regions method \cite{Smirnov:2021dkb,Jantzen_2011,Beneke_1998}, which reconstructs the loop integral result in an expanded form. This method proceeds as follows:
	

	
	\begin{enumerate}
		\item Divide the space of the loop momenta into regions according with the masses or kinematical scales and, in every region expand the integrand into a Taylor series with respect to the parameters that are small there.
		\item Use dimensional regularization to integrate the integrand, expanded in the appropriate way in every region, over the whole integration domain of the loop momenta.
		\item Set to zero any scaleless integral.
		\item Sum the result of the integral in each region.
	\end{enumerate}
	
	
	When integrating out heavy fields of a UV theory, we have two distinct mass scales: the light fields mass $m$ and the heavy fields mass $M$. It is therefore natural to assume that the two regions of the loop momenta space relevant to this problem are
	
	\begin{itemize}
		\item the hard region, where $q\sim M$,
		\item and the soft region where $q\sim m$.
	\end{itemize}
	
	Here, $q\sim M$ mean that all components of the loop momentum $q$ are of the order of the mass $M$, and similarly for $q\sim m$. As a consequence of the expansion by regions method, any ``$\ln\det$'' can be decomposed as\cite{Jantzen_2011}:
	
		\begin{align}
		\ln \text{det} X = \ln \text{det} X\vert_{\text{hard}} + \ln \text{det} X\vert_{\text{soft}},
		\end{align}
	
	\noindent where hard and the soft region contributions are obtained by first expanding the integrand for $\abs{q^2}\sim M^2 \gg m^2$ and $\abs{q^2}\sim m^2 \ll M^2$, respectively, and then integrating over the full loop momentum space, as specified by the method steps. Applying this decomposition to the logarithms of determinants appearing in Eq. (\ref{eq:L_1_loop_1}), we obtain:
	
	\begin{align}
		\begin{cases}
			&\ln\det\left(	\Delta_H - X_{HL}\Delta_L^{-1}X_{LH}\right) = \ln\det\left(	\Delta_H - X_{HL}\Delta_L^{-1}X_{LH}\right)_{\text{hard}} + \ln\det\left(	\Delta_H - X_{HL}\Delta_L^{-1}X_{LH}\right)_{\text{soft}}\\
			&\ln\det\left( \hat{\Delta}_H- X_{HL}\Delta_L^{-1} X_{LH}\right) = \ln\det\left( \hat{\Delta}_H- X_{HL}\Delta_L^{-1} X_{LH}\right)_{\text{soft}}\\
			&\ln\det\det\hat{\Delta}_H = \ln\det\hat{\Delta}_H\vert_{\text{soft}} = 0
		\end{cases}
	\end{align}
	
	In the second and third equation there is no hard region contribution once the replacement of $\Delta_H$ by its local expansion $\hat{\Delta}_H$ in powers of $1/M$ is equivalent to isolating only the soft region. Furthermore, $\ln\det\hat{\Delta}_H\vert_{\text{soft}}$ vanishes because, for pure heavy loops, expanding in the soft region produces scaleless integrals ($M^{-c} \int d^d q, q^n = 0$), which are set to zero according to the third step of the expansion by regions method.
	
	Substituting these results into the one-loop EFT Lagrangian expression (Eq. (\ref{eq:L_1_loop_1})), we obtain:
	
	\begin{align}
		\int d^dx \mathcal{L}^{1-loop}_{EFT}[\phi_b] &= \frac{i}{2}\ln{\text{det}\left(\Delta_H - X_{HL}\Delta^{-1}_L X_{LH}\right)}\vert_{\text{hard}} \label{eq:L_EFT_2}
	\end{align}
	
	%Notice that since pure light loops does not receive contribution from the hard region, i.e, $\ln\det\Delta_L\vert_{\text{hard}} = 0$, then the right-hand side of the equation above is the hard contribution of the 1LPI generating functional of the UV theory:
	
	Since pure light loops do not receive contributions from the hard region, i.e., $\ln\det\Delta_L\vert_{\text{hard}} = 0$, the right-hand side of the equation above corresponds to the hard contribution of the one-loop 1LPI generating functional of the UV theory:
	
	\begin{align}
		\int d^dx \mathcal{L}^{1-loop}_{EFT}[\phi_b] &= \Gamma^{\text{1-loop}}_{L,UV}\vert_{\text{hard}}
	\end{align}
	
	This result demonstrates that the one-loop EFT Lagrangian is entirely determined by the hard contribution, i.e., by highly virtual loops with momenta outside the EFT regime ($q \sim M \gg m$). Physically, this means that the effective operators in $\mathcal{L}^{\text{1-loop}}_{\text{EFT}}[\phi_b]$ encode the short-distance behavior of the full theory, while the long-distance effects, governed by the soft region, are already captured by the tree-level EFT Lagrangian.
	
	
	
	
	%In the hard region, all the low-energy scales are expanded out and only $M$ remains in the propagator. The resulting integrand yields local contributions in form of a polynomial in the low-energy momenta and masses, with factors of $1/M$ to adjust the dimensions. This part is therefore fully determined by the short-distance behavior
	
	
	\subsection{Evaluating functional determinants}\label{subsec:Func_determinants}
	
	To evaluate the one-loop EFT Lagrangian via Eq. (\ref{eq:L_EFT_2}), it is convenient to transform the determinant into a trace using a standard linear algebra identity\cite{Peskin:1995ev}:
	
	\begin{align} \ln\det X = \Tr\ln X. \end{align}
	
	Applying this identity, we obtain:
	
	\begin{align} \int d^d x, \mathcal{L}^{\text{1-loop}}_{\text{EFT}}[\phi_b] &= \frac{i}{2} \Tr \ln \left( \Delta_H - X_{HL} \Delta_L^{-1} X_{LH} \right) \Big|_{\text{hard}}, \label{eq:L_EFT_3}
	\end{align}
	
	\noindent where $\Tr$ denotes the trace over both the function space in which the operator acts and all relevant internal indices (e.g., spin, color, and flavor) of the quantum fields involved.
	
	%Unfortunately, we cannot diagonalize $\ln \left( \Delta_H - X_{HL} \Delta_L^{-1} X_{LH} \right)$ and compute its spectrum once it depends, in general, of arbitrary functions, i.e, the background fields. However, we can still develop a perturbative approximation of the trace.
	
	%The computation of the trace requires that we choose a basis and, consequently, a representation for the operator. The trivial choice would be the eigenstate basis of $\ln \left( \Delta_H - X_{HL} \Delta_L^{-1} X_{LH} \right)$, which we call $\ln\Delta$ to simplify the notation. However, we cannot diagonalize this operator and compute its spectrum once, in general, it depends of arbitrary functions (background fields). The second option is the momentum or position eigenstate basis because $\Delta$ derives from the UV Lagrangian density and it is often represented in one of those spaces, i.e, we know $\ln\Delta(x,i\partial_x)$ or $\ln\Delta(-i\partial_q,q)$ such\cite{henning2015usestandardmodeleffective}:
	
	%The computation of this trace requires choosing a basis, which determines a specific representation for the operator. A natural choice would be the eigenstate basis of $\ln \left( \Delta_H - X_{HL} \Delta_L^{-1} X_{LH} \right)$, which we simplify as $\ln\Delta$. However, in general, this operator depends on arbitrary background fields, preventing its diagonalization and direct computation of its spectrum. Instead, a practical alternative is to work in the position or momentum eigenstate basis, since $\Delta$ derives from the UV Lagrangian density, which is typically represented in one of these spaces, i.e, we know $\ln\Delta(x,i\partial_x)$ or $\ln\Delta(-i\partial_q,q)$ such\cite{henning2015usestandardmodeleffective}:
	
	The computation of this trace requires choosing a basis, which determines a specific representation for the operator. A natural choice would be the eigenstate basis of $\ln \left( \Delta_H - X_{HL} \Delta_L^{-1} X_{LH} \right)$, which we simplify as $\ln\Delta$. However, in general, this operator depends on arbitrary background fields, preventing its diagonalization and direct computation of its spectrum. Instead, a practical alternative is to work in the position or momentum eigenstate basis, where we can express
	
	
	\begin{align}
		\bra{x}\ln\Delta = \ln\Delta(x,i\partial_x)\bra{x} \ \ \ or \ \ \ \bra{q}\ln\Delta =  \ln\Delta(-i\partial_q,q)\bra{q},
	\end{align}
	
	\noindent once $\Delta$ derives from the UV Lagrangian density, which is typically represented in one of these spaces. For convenience, we start computing the trace in momentum eigenstate basis and insert an identity in position representation:
	
	%For notational simplicity we define $\Delta = \Delta_H - X_{HL} \Delta_L^{-1} X_{LH} $ and since $\Delta$ derives from a Lagrangian density, is therefore a function of position and momentum operator $\Delta(\hat{x},\hat{p})$. Using the momentum eigenstate basis to compute the spacetime part of the trace and inserting an identity in position space representation:
	%(\hat{x},\hat{p})
	

	\begin{align}
		\Tr \ln\Delta &= \int \frac{d^dq}{(2\pi)^d} \bra{q}\tr\ln\Delta\ket{q} = \int d^dx\int \frac{d^dq}{(2\pi)^d} \bra{q}\ket{x}\bra{x}\tr \ln\Delta\ket{q} = \int d^dx\int \frac{d^dq}{(2\pi)^d} e^{iq\cdot x} \tr \ln\Delta(x,i\partial_x)e^{-iq\cdot x}
	\end{align}
	
	\noindent where we used that $\bra{q}\ket{x} = e^{iq\cdot x}$, and ``$\tr$'' represent the trace over internal indices only. The derivatives in $\Delta(x,i\partial_x)$ yields factors of $q$ upon acting on the exponentials, leading to:
	
	\begin{align}
		&\Tr \ln\Delta = \int d^dx\int \frac{d^dq}{(2\pi)^d} \tr\ln\Delta(x, i\partial_x + q)\mathds{1}
	\end{align}
	
	Notice that even if $\Delta(x,i\partial_x)$ contains transpose derivatives $\partial_x^T$ acting on the left, it can be replaced by $-\partial_x$ using integration by parts, since the difference is a total derivative. Therefore, the result above remains general. Furthermore, the identity $\mathds{1}$ is a reminder that any derivative at rightmost disappear after acting on the exponential.
	
	Since the integral runs over the entire momentum space, it remains invariant over the variable change $q \to -q$, which gives:
	
	\begin{align}
		&\Tr \ln\Delta = \int d^dx\int \frac{d^dq}{(2\pi)^d}\tr\ln\Delta(x, i\partial_x - q)
	\end{align}
	
	Substituting this result into Eq. (\ref{eq:L_EFT_3}):
	
	\begin{align}
		\mathcal{L}^{1-loop}_{EFT}[\phi_b] = \frac{i}{2}\int \frac{d^dq}{(2\pi)^d}\text{tr}\ln{\left(\Delta_H - X_{HL}\Delta^{-1}_L X_{LH}\right)}_{\text{hard}}\vert_{P \to P-q}\label{eq:L_EFT_4}
	\end{align}
	
	
	\noindent where $P$ is the momentum representation of the covariant derivative and $q$ is the shift coming from the trace calculation that we discussed along this section.
	
	Some references \cite{henning2015usestandardmodeleffective, Drozd_2016} apply an additional transformation to Eq. (\ref{eq:L_EFT_4}), which we do not adopt here. However, it is worth briefly discussing. The idea is to introduce $e^{P\cdot\partial_q}$ and $e^{-P\cdot\partial_q}$, sandwiching the integrand of Eq. (\ref{eq:L_EFT_4})\footnote{This is equivalent to inserting identities, given that the derivatives $\partial_q$ in the exponentials act on unity to the right (for $e^{-P\cdot\partial_q}$) and, by integration by parts, can be made to act on unity to the left (for $e^{P\cdot\partial_q}$).} and use the Baker-Campbell-Hausdorff formula to put all covariant derivatives into commutators. Although, is advantageous to deal with $P_\mu$ in intermediate steps or in the final result through commutators, the use of Baker-Campbell-Hausdorff formula generate a lot of terms with a notation that is hard to follow and rearranging them.
	
	
	
	

	
	
	%At this point, there is one additional transformation that can be made \cite{henning2015usestandardmodeleffective, Drozd_2016}, but we are not going to do it. The idea behind would be introduce $e^{P\cdot\partial_q}$ and $e^{-P\cdot\partial_q}$ sandwiching the integrand of Eq. (\ref{eq:L_EFT_4})\footnote{This is equivalent to insert identities once is understood that the derivatives $\partial_q$ contained in the exponential act on unity to the right (for $e^{-P\cdot\partial_q}$) and, by integration by parts, can be made to act on unity to the left (for $e^{P\cdot\partial_q}$)} and use the Baker-Campbell-Hausdorff formula to put all covariant derivatives into commutators. This is conveni 
	
	
	\subsection{Covariant derivative expansion (CDE)}\label{subsec:CDE_Scalar}
	
	Unfortunately, it is not possible to compute exactly the right-handed side of Eq. (\ref{eq:L_EFT_4}). However, we can still develop a perturbative expansion. In this work, we employ the Covariant Derivative Expansion (CDE), a method that, as the name suggests, involves a power series expansion in the covariant derivative ($P_\mu$ or $D_\mu$), rather than splitting it into the partial derivative and gauge fields.
	
	In order to obtain a CDE for the one-loop EFT Lagrangian, we start by splitting the kinetic and interaction contributions in $\Delta_H$ assuming that it takes the generic form:
	
	
	\begin{align}
		\Delta_H = -P^2 + M^2 + X_H(\phi_b,P)
	\end{align}
	 
	\noindent with $X_H(\phi,P)$ containing the interaction contributions. Substituting this into Eq. (\ref{eq:L_EFT_4}) and already performing the shift $P\to P-q$ in $\Delta_H$:
	
	
	\begin{align}
		\mathcal{L}^{1-loop}_{EFT}[\phi_b] &= \frac{i}{2}\int \frac{d^dq}{(2\pi)^d}\text{tr}\ln{\left( -P^2 +2q\cdot P - q^2 + \left(X_H- X_{HL}\Delta^{-1}_L X_{LH}\right)_{P \to P-q}\right)}_{\text{hard}}\nonumber\\
		&= \frac{i}{2}\int \frac{d^dq}{(2\pi)^d}\tr\left\{\ln{[M^2 - q^2]} + \ln{\left[1 - \frac{\left(2q\cdot P - P^2 + \left(X_H- X_{HL}\Delta^{-1}_L X_{LH}\right)_{P \to P-q}\right)}{q^2 - M^2}\right]}\right\}_{\text{hard}}\label{eq:L_EFT_6}
	\end{align}
	
	The first term on the right-hand side is independent of $\phi_b$ or any gauge field. Consequently, it can be dropped, as the Lagrangian is invariant under constant shifts. Thus:
	
	
	\begin{align}
		\mathcal{L}^{1-loop}_{EFT}[\phi_b] =  \frac{i}{2}\int \frac{d^dq}{(2\pi)^d}\tr\ln{\left[1 - \frac{\left(2q\cdot P - P^2 + \left(X_H- X_{HL}\Delta^{-1}_L X_{LH}\right)_{P \to P-q}\right)}{q^2 - M^2}\right]_{\text{hard}}}\label{eq:L_EFT_7}
	\end{align}
	
	
	In the hard region, the fraction inside the logarithm is small, allowing us to expand it as a Taylor series. To see why, let us analyze each term. The denominator, $q^2-M^2$, is of order $O(M^{2})$ since $q\sim M$ in this region. Meanwhile, in the numerator, $X_H$ arises from interaction terms in the UV Lagrangian that involve at least three fields. Since we are dealing with bosons, it may include dimension-4 operators containing either a dimensionful parameter of order $M$ or a derivative, $X_H$ is at most of order $\mathcal{O}(M)$. The terms in $X_{HL}\Delta^{-1}_L X_{LH}$ appear from the product of two interaction terms and a light-field propagator and hence they generate terms of order $\mathcal{O}(M^0)$\cite{Fuentes_Mart_n_2016}. 
	
	The remaining two terms contain covariant derivatives that acts on $\phi_b$ or generates field strength tensors. Decomposing the covariant derivatives using its definition implies that those terms are composed by background light fields (gauge and others) and ordinary derivatives of them, or equivalently momentum powers ($p$) related to this fields. Therefore, since the EFT validity is for energy scales $\Lambda \ll M$, these momentum should satisfy $p\ll M$ which also means that the fields in the EFT need to be slowly varying on distance scales of order $M^{-1}$\cite{henning2015usestandardmodeleffective}. Trivially, these conditions extends to the whole covariant derivative, i.e, $P \ll M$, which also implies that $2q\cdot P$ is $O(M)$, once $q\sim M$ in the hard region.
	
	With all these considerations we can generally state that the numerator is at most $O(M)$ while the denominator is $O(M^2)$ in the hard region. Therefore:
	
	\begin{align}
		\frac{2q\cdot P - P^2 + \left(X_H- X_{HL}\Delta^{-1}_L X_{LH}\right)_{P \to P-q}}{q^2 - M^2} << 1
	\end{align}
	
	\noindent and we can Taylor expand the logarithm in Eq. \ref{eq:L_EFT_7}:
	
	\begin{align}
		\mathcal{L}^{1-loop}_{EFT}[\phi_b] = -\frac{i}{2}\sum_{n=1}^{+\infty}\frac{1}{n}\int \frac{d^dq}{(2\pi)^d} \left[\frac{\left(2q\cdot P - P^2 + \left(X_H- X_{HL}\Delta^{-1}_L X_{LH}\right)_{P \to P-q}\right)}{q^2 - M^2}\right]^n_{\text{hard}}
	\end{align}
	
	This result demonstrates that, for the case of real scalars, the one-loop effective Lagrangian is an infinite series of local operators generated by evaluating momentum integrals. However, in practice, we typically truncate this series at a specific operator dimension or, equivalently, a given power of $1/M$, based on the desired level of accuracy.
	
	%Furthermore, this result together with the tree level EFT Lagrangian expression that we found, does not give us the full simplified EFT Lagrangian up to one-loop order. We can still perform some simplification.
		
	
	After computing the EFT Lagrangian up to one-loop order and to the desired operator dimension, there is often room for further simplification by identifying and eliminating redundant operators. These are operators that can be rewritten in terms of others without affecting the $S$-matrix elements and, consequently, any physical observables. The objective is to construct a minimal, non-redundant set of operators that accurately captures the same physics as the unsimplified Lagrangian. This process is analogous to finding a basis in linear algebra, but in this context, it applies to the space of higher-dimensional operators.
	
	Simplification procedure is decomposed in two steps. First, we apply exact identities, such as integration by parts, group identities, and commutation relations, to relate operators linearly and reduce the Lagrangian to what is known as a Green's basis. In this context, the Lagrangian can be seen as an element of a vector space spanned by all possible operators consistent with the symmetries of the theory. The exact identities serve as constraints that reduce this space, eliminating linear dependencies and leaving a smaller, but physically equivalent, basis.
	
	
	The second step involves field redefinitions, where we rewrite the field variables as nonlinear but local functions of a new set of field variables to eliminate redundant operators. According to the equivalence theorem \cite{CHISHOLM1961469,Coleman.177.2239,Kallosh:1972ap}, such field redefinitions do not alter the on-shell $S$-matrix elements, ensuring that the simplified Lagrangian is physically equivalent to the original.
	
	\textcolor{red}{Should i add something discussing the relation between using the EOM of the field and the field redefinition? For the sake of generality we are going to use only field redefinitions (see example).}
	
	
	%Importantly, this step connects to the use of the classical equations of motion to eliminate redundant operators. Applying the equations of motion to trade one operator for another is equivalent to performing a field redefinition in a nonlinear way\footnote{As demonstrated in Ref. \cite{Arzt_1995}, shifting higher-dimensional operators by a term proportional to the equations of motion does not affect the $S$-matrix elements, being consistent with the equivalence theorem for field redefinitions.}. Typically, the kinetic part of the Lagrangian generates a term of the form $D_\mu D^\mu \phi$ in the equation of motion. Thus, any interaction operator containing at least one occurrence of $D_\mu D^\mu \phi$ can be systematically eliminated from the Lagrangian using the equations of motion. 
	
	\textcolor{red}{Does worth to mention that MACHETE has these simplification steps automated?}
	
	
	%The second step involves field redefinitions, where we express the change variable as nonlinear but local functions of a new set of field variables. As a consequence to the equivalence theorem \cite{CHISHOLM1961469,Coleman.177.2239,Kallosh:1972ap}, these field redefinitions do not affect the on-shell $S$-matrix elements, ensuring that the simplified Lagrangian is on-shell equivalent to the original. A significant aspect of this step is that applying the equations of motion to trade one interaction operator for another is the same as redefining the fields in the Lagrangian in a non-linear way. Typically, the kinetic part of the Lagrangian contributes with a $D_\mu D^\mu\phi$ term to the equation of motion. Therefore, interaction operators with at least one occurrence of $D_\mu D^\mu\phi$ can be removed of the Lagrangian by using the EOM. As demonstrated in Ref. \cite{Arzt_1995}, shifting higher-dimensional operators by a term proportional to the classical EOM does not alter the $S$-matrix elements, as expected from the relation to the field redefinitions and the equivalence theorem. As a consequence, using the equation of motion to eliminate redundant operator is valid as a systematic approach.
	
	%The simplification procedure can be approached in two steps. First, we use exact identities, such as integration by parts, group identities or commutation relations, to linearly relate operators and taking the Lagrangian to what we call Green's basis. Then, we use field redefinitions that rewrites the field variable as a nonlinear but a local function of another set of field variables to eliminate the rest of redundant operators. As a consequence of the equivalence theorem \cite{CHISHOLM1961469,Coleman.177.2239,Kallosh:1972ap}, these field redefinitions does not change the on-shell $S$-matrix elements and, therefore, they will produce a Lagrangian with on-shell equivalence with the unsimplified one. An interesting point of field redefinition is that trading one interaction term for another using the equations of motion is the same as redefining fields in a non-linear way. Therefore, every interaction operator with at least one occurrence of $D_\mu D^\mu\phi$ can be removed from the Lagrangian using the equation of motion of the field since it always contains a piece $D_\mu D^\mu\phi$ coming from the kinetic part. As shown in Ref.\cite{Arzt_1995}, shifting the higher-dimensional operators by a term proportional to the classical equations of motion does not change the $S$-matrix elements, as we expect when interpret this as a field redefinition.

	
	
	%The simplification procedure can be approached in two steps. First, we use exact identities, such as integration by parts, group ident redundancies. In this step, we can think of the unsimplified Lagrangian $\mathcal{L}_{EFT}$ as an element of a vector space $\mathcal{V}$, whose basis ${\mathcal{V}_a}$ consists of all possible gauge and Lorentz-invariant monomials of fields, their covariant derivatives, and relevant coefficients. Initially, this basis is overcomplete because it does not account for the exact identities that relate the operators. These identities can be viewed as vectors in $\mathcal{V}$ that are equivalent to zero, forming a subspace $I \subseteq \mathcal{V}$. The process of simplifying $\mathcal{L}_{EFT}$ then corresponds to identifying a convenient basis for the coset space $\mathcal{V}/I$ and finding the appropriate representative of $\mathcal{L}{EFT}$ within this space.ities, and commutation relations, to eliminate
	
	%After computing the EFT Lagrangian up to one-loop order and to the desired operator dimension, it is often possible to simplify the result further by identifying and eliminating redundant operators. These are operators that can be rewritten in terms of others without affecting the $S$-matrix elements and, consequently, any physical observables. Basically, we need to find a minimal set of operator that describes the same physics as the unsimplified Lagrangian, i.e, it is analogous of the basis concept from linear algebra but for the higher dimensional operators.
	
	%We can split the simplification process into two parts, in the first we are going to use exact identities, such as integration-by-parts, group identities or commutation relations, to eliminate redundant operators. In this step, we can think in $\mathcal{L}_{EFT}$ as an element of a vector space $\mathcal{V}$ equipped with a basis $\{\mathcal{V}_a\}$ consisting of all operators in the absence of any exact identities. That is, the element of this basis span the complete set of gauge and Lorentz-invariant monomials of the fields, their covariant derivatives, and CG coefficients (including Dirac matrices). This vector space is redundant once the exact identities, relating
	%the basis operators, are accounted for. Each identity relation can be represented as a vector
	%that is equivalent to 0. Together, the identity vectors span a subspace $I \subseteq \mathcal{V}$, and we can identify the coset $\mathcal{V}/I$ with a set of Green’s basis Lagrangians. Simplifying $\mathcal{L}_{EFT}$ then comes down to finding a convenient basis for $\mathcal{V}/I$ and determining the representative element of the
	%equivalence class $\mathcal{L}_{EFT}$ defined by the coset.
	
	%eliminate redundant operators, which are operators that can be rewritten in terms of another operators without change the on-shell S-matrix elements ans, consequently, observable quantities. There are two types of simplification that we can do: using exact identities or field redefinition. When we say exact identities we
	
	%When computing an EFT Lagrangian we will often find results that contains redundant operators, which are those that who we can re-expressed in terms of another operators without affect the on-shell S-matrix and consequently observables. Therefore, eliminating those redundant operators simplifies the Lagrangian expression. We can distinguish between two types of simplification, the first is to use exact identities, such integration-by-parts, group identities or commutation relations and the second is to use field redefinitions that satisfies the on-shell equivalence. For instance, we already use this first type of simplification when replaced $\phi^2 \partial_\mu \phi \partial^\mu \phi$ by $-\frac{1}{3}\phi^3 \Box \phi$ alleging the use of integration-by-parts.
	
	%Another very common simplification, which belongs to the second type, is the shift of some operators by using the classical equation of motion of the field. It was proven in Ref. \cite{Arzt_1995} that shifting operators by a term proportional to the EOM does not cgange the S-matrix elements, even at loop level. The point is that trading one interaction term for another using the equations
	%of motion is the same as redefining the fields in the Lagrangian in a non-linear way.
	
	
	\section{Example: Integrating out a real scalar in a toy model}
	
	In the previous section, we established a systematic approach to perform matching, up to one-loop order, between a UV theory and its corresponding EFT, where a multiplet of heavy and real scalar fields are integrated out. To illustrate this method concretely, we will now apply it to a simple toy model consisting of two scalar fields: one heavy and one light. This example will provide a clear and practical demonstration of the matching procedure, highlighting the essential steps and techniques discussed earlier.
	
	Consider a renormalizable UV model with a $\mathcal{Z}_2$ symmetry, described by the following Lagrangian:
	

	
		\begin{align}
		\mathcal{L}_{UV}[\phi,\Phi] = \frac{1}{2}\left[\partial_\mu \phi\partial^\mu \phi - m^2\phi^2 + \partial_\mu \Phi\partial^\mu \Phi - M^2\Phi^2\right] -\frac{\lambda_0}{4!} \phi^4 - \frac{\lambda_1}{4} \phi^2 \Phi^2\label{eq:UV_ex}
		\end{align}
		
	\noindent where $\phi$ and $\Phi$ are light and heavy real scalar fields with mass $m$ and $M\gg M$, respectively.

	
	For energy scales $\Lambda \ll M$, we wish to derive the EFT obtained by integrating out the heavy field $\Phi$. Applying the systematic procedure discussed earlier, the resulting EFT Lagrangian will be composed of local effective operators constructed solely from the light field $\phi$. These operators can be separated into kinetic, mass and interaction terms, which are organized as an infinite series of operators with increasing dimension. We expect that the general structure of the EFT Lagrangian is given by:
	
	\begin{align} 
		\mathcal{L}_{EFT}[\phi] = \frac{1}{2}\partial^\mu\phi \partial_\mu \phi - \frac{m_L^2}{2}\phi^2 - \sum_{\mathcal{D}=4}^{\infty} \frac{C_\mathcal{D}}{M^{\mathcal{D}-4}} O_\mathcal{D} 
	\end{align}
	
	\noindent where $O_\mathcal{D}$ are operators of dimension $\mathcal{D}$, $C_\mathcal{D}$ are dimensionless Wilson coefficients, and $m_L$ is the effective mass of the light field, potentially modified by virtual heavy field contributions. For clarity, we have omitted the subscript $b$ from the background fields of the EFT to simplify the notation.
	
	To make the calculation more tractable while still capturing the essential features of the matching procedure, we truncate the series at dimension six, corresponding to terms suppressed by $1/M^2$. We now proceed with the explicit matching calculation, beginning with the tree-level contribution before moving on to one-loop corrections.
	
	
	
	\subsection{Tree level}
	
	At tree level, the EFT Lagrangian is obtained by integrating out the heavy field using its classical equation of motion. From our earlier discussion, this is given by Eq. (\ref{eq:L_EFT_tree}):
	
	\begin{align}
		\mathcal{L}_{EFT}^{\text{tree}}[\phi] = \mathcal{L}_{UV}[\phi,\hat{\Phi}_c[\phi]]
	\end{align} 
	
	The first step in evaluating this expression is to find the solution to the classical equation of motion for the heavy field $\Phi$:

	
	\begin{align}
		\frac{\delta \mathcal{L}_{UV}[\phi,\Phi]}{\delta \Phi}\bigg\vert_{\Phi = \Phi_c} = 0 \implies \left(\partial^2 +M^2 +\frac{\lambda_1}{2} \phi^2 \right)\Phi \bigg\vert_{\Phi = \Phi_c}= 0 \implies \Phi_c[\phi] =  0
	\end{align}
	
	Since the solution is already local, there is no need to perform an expansion. Substituting this result into the UV Lagrangian (\ref{eq:UV_ex}) and then into the tree-level EFT expression, we obtain:
	
	
	
	\begin{align}
		\mathcal{L}_{EFT}^{\text{tree}}[\phi] = \frac{1}{2} \partial_\mu \phi\partial^\mu \phi -\frac{1}{2} m^2\phi^2  -\frac{\lambda_0}{4!} \phi^4 \label{eq:tree}
	\end{align}
	
	As we can see, the tree-level EFT Lagrangian does not contain higher-dimensional operators. 
	
	\subsection{One-loop level}
	
	Having obtained the tree-level EFT Lagrangian, we now move on to the one-loop level matching. As derived in the previous section, the resulting one-loop contribution to the EFT Lagrangian is given by (Eq. (\ref{eq:1_loop_EFT})):
	
	
	
	
	\begin{align}
		\mathcal{L}^{1-loop}_{EFT}[\phi_b] = -\frac{i}{2}\sum_{n=1}^{+\infty}\frac{1}{n}\int \frac{d^dq}{(2\pi)^d} \left[\frac{\left(2q\cdot P - P^2 + \left(X_H- X_{HL}\Delta^{-1}_L X_{LH}\right)_{P \to P-q}\right)}{q^2 - M^2}\right]^n_{\text{hard}}\label{eq:1_loop_EFT}
	\end{align}
	
	\noindent where $X_H$, $X_{LH}$, $X_{HL}$ and $\Delta_L$ are components of the fluctuation operator of the UV theory. Using the example UV Lagrangian expression into the definition of theses components (Eq. (\ref{eq:Q_UV})) we find:
	
	
	\begin{align}
			&X_H = -\frac{\delta^2 \mathcal{L}_{UV}}{\Phi'^2}\bigg\vert_{\Phi = 0} = \frac{\lambda_1}{2}\phi^2\\
			&X_{LH} = -\frac{\delta^2 \mathcal{L}_{UV}}{\phi'\Phi'}\bigg\vert_{\Phi = 0} = 0\\
			&X_{HL} = -\frac{\delta^2 \mathcal{L}_{UV}}{\Phi'\phi'}\bigg\vert_{\Phi = 0} = 0\\
			&\Delta_L = -\frac{\delta^2 \mathcal{L}_{UV}}{\phi'^2}\bigg\vert_{\Phi = 0} = \left(\partial^2 + m^2 + \frac{\lambda_0}{2}\phi\right)\phi
		\end{align}
	
	 The absence of heavy-light mixed terms ($X_{LH}$ and $X_{HL}$) arises due to two factors: the $\mathcal{Z}_2$ symmetry of the theory, which prohibits linear couplings in at least one of the fields, and the fact that the classical solution for the heavy field is zero.
	
	\begin{align}
		\mathcal{L}^{1-loop}_{EFT}[\phi] = -\frac{i}{2}\sum_{n=1}^{+\infty}\frac{1}{n}\int \frac{d^dq}{(2\pi)^d} \left[\frac{\left(2q\cdot P - P^2 + \frac{\lambda_1}{2}\phi^2\right)}{q^2 - M^2}\right]^n_{\text{hard}}
	\end{align}
	
	In the above expression, the subscript "hard" denotes that the loop momenta are in the high-energy regime, $q \sim M$. Normally, this would allow us to Taylor expand small parameters in the integrand, i.e, low-energy scales such as light field mass or derivatives acting on the light field with the appropriate factors of $1/M$ to adjust dimension. However, since the integrand already is a polynomial in low-energy scales, the hard subscript becomes redundant and can be omitted. Consequently, we have:
	
	
	\begin{align}
		\mathcal{L}^{1-loop}_{EFT}[\phi] = -\frac{i}{2}\sum_{n=1}^{+\infty}\frac{1}{n}\int \frac{d^dq}{(2\pi)^d} \left[\frac{\left(2q\cdot P - P^2 + \frac{\lambda_1}{2}\phi^2\right)}{q^2 - M^2}\right]^n
	\end{align}
	
	The infinite series above generates effective operators of all possible dimensions. However, since we aim to retain only operators up to dimension six, we must extract the relevant contributions accordingly. This can be accomplished by analyzing the initial terms of the series, selecting those that contain the appropriate number of light fields and derivatives to match the target dimensionality. Alternatively, we can filter contributions by focusing on terms suppressed by factors up to $\mathcal{O}(1/M^2)$.
	
	Since heavy mass contributions always arise from the loop integral, it is more practical in this scenario to count the fields and derivatives directly to determine the operator's dimension. Recall that in four-dimensional field theory, a scalar field $\phi$ and each derivative $\partial_\mu$ have mass dimension 1.
	
	We begin by examining the term with $n=1$:
	
	\begin{align}
		\mathcal{L}^{1-loop}_{EFT}[\phi]&\supset -\frac{i}{2}\int \frac{d^dq}{(2\pi)^d} \left[\frac{\left(2q\cdot P - P^2 + \frac{\lambda_1}{2}\phi^2\right)}{q^2 - M^2}\right]
	\end{align}
	
	In this simple UV theory without Gauge symmetries, the covariant derivative reduces to the ordinary derivative. Since derivatives act on fields placed to their left, they do not contribute here, simplifying the analysis. Thus, the only surviving term corresponds to a dimension-2 operator:
	
	\begin{align}
		\mathcal{L}^{1-loop}_{EFT}[\phi]&\supset -\frac{i}{2}\frac{\lambda_1}{2}\phi^2\int \frac{d^dq}{(2\pi)^d} \frac{1}{q^2 - M^2}
	\end{align}
	
	The loop integral is evaluated using dimensional regularization\cite{Zhang_2017}:
	
	\begin{align}
		\int \frac{d^dq}{(2\pi)^d} \frac{1}{q^2 - M^2} = \frac{i}{16\pi^2}M^2 \left[\frac{2}{\bar{\epsilon}} + \ln\left(\frac{\mu^2}{M^2}\right) + 1\right]
	\end{align}
	
	\noindent where $\frac{2}{\bar{\epsilon}} = \frac{2}{\epsilon} - \gamma_E + \ln 4\pi$. In the $\overline{MS}$ scheme, is understood that $\frac{2}{\bar{\epsilon}} + \ln\left(\frac{\mu^2}{M^2}\right)$ is replaced by $ln\left(\frac{\mu^2}{M^2}\right)$ in the final result. Applying this, we obtain:
	
	
	\begin{align}
		\mathcal{L}^{1-loop}_{EFT}[\phi]\supset \frac{\lambda_1 M^2}{32\pi^2} \left[1 + \ln\left(\frac{\mu^2}{M^2}\right)\right]\frac{\phi^2}{2}\label{eq:n1}
	\end{align}
	
	
	We now examine the contribution from the $n=2$ term:
	
	\begin{align}
		\mathcal{L}^{1-loop}_{EFT}[\phi]&\supset -\frac{i}{4}\int \frac{d^dq}{(2\pi)^d} \left[\frac{\left(2q\cdot P - P^2 + \frac{\lambda_1}{2}\phi^2\right)}{q^2 - M^2}\right]\left[\frac{\left(2q\cdot P - P^2 + \frac{\lambda_1}{2}\phi^2\right)}{q^2 - M^2}\right]\nonumber\\
		&\supset-\frac{i}{4}\int \frac{d^dq}{(2\pi)^d}\frac{\left(2q\cdot P - P^2 + \frac{\lambda_1}{2}\phi^2\right)\left( \frac{\lambda_1}{2}\phi^2\right)}{\left(q^2 - M^2\right)^2}
	\end{align}
	
	In the above expression, the first two terms in the numerator involve total derivatives. Since total derivative terms in the Lagrangian do not contribute to the physical dynamics, they can be discarded. As a result, the surviving contribution is a dimension-four operator:

	
	\begin{align}
		\mathcal{L}^{1-loop}_{EFT}[\phi]\supset -\frac{i}{4}\frac{\lambda_1^2}{4} \phi^4 \int \frac{d^dq}{(2\pi)^d} \frac{1}{\left(q^2 - M^2\right)^2 }
	\end{align}
	
	To evaluate the loop integral, we use dimensional regularization\cite{Zhang_2017}:
	
	\begin{align}
		\int \frac{d^dq}{(2\pi)^d} \frac{1}{\left(q^2 - M^2\right)^2 } = \frac{i}{16\pi^2}\left[\frac{2}{\bar{\epsilon}} + \ln\left(\frac{\mu^2}{M^2}\right)\right]
	\end{align}
	
	In the $\overline{\text{MS}}$ scheme, where $\frac{2}{\bar{\epsilon}}$ is subtracted, we obtain:
	
	\begin{align}
		\mathcal{L}^{1-loop}_{EFT}[\phi]\supset \frac{3\lambda_1^2}{32\pi^2}\ln\left(\frac{\mu^2}{M^2}\right) \frac{\phi^4}{4!} \label{eq:n2}
	\end{align}
	
	Let's examine the contribution from the $n=3$ term in the series:
	
	\begin{align}
		\mathcal{L}^{1-loop}_{EFT}[\phi]&\supset -\frac{i}{6}\int \frac{d^dq}{(2\pi)^d} \left[\frac{\left(2q\cdot P - P^2 + \frac{\lambda_1}{2}\phi^2\right)}{q^2 - M^2}\right]^3\nonumber\\
		&\supset-\frac{i}{6}\int \frac{d^dq}{(2\pi)^d}\frac{\left(2q\cdot P - P^2 + \frac{\lambda_1}{2}\phi^2\right)\left(2q\cdot P - P^2 + \frac{\lambda_1}{2}\phi^2\right)\left( \frac{\lambda_1}{2}\phi^2\right)}{\left(q^2 - M^2\right)^3}
	\end{align}
	
	To understand the possible effective operators generated, we analyze the operator dimensions. The minimal dimension possible is four. However:
	
	\begin{itemize}
		\item The only dimension-four operator with two derivatives and two scalar fields is a total derivative. Since total derivatives do not affect the physical dynamics, these terms can be ignored.
		\item Dimension-five operators, containing three derivatives and two scalar fields, are forbidden by the $\mathcal{Z}_2$ symmetry.
		\item Thus, the leading non-trivial contributions arise from dimension-six operators. These can take the form of either six scalar fields or four scalar fields with two derivatives.
	\end{itemize}
	
	Therefore, the relevant operators emerging from this term are those with six scalar fields or four fields with two derivatives:
		
	\begin{align}
		\mathcal{L}^{1-loop}_{EFT}[\phi]\supset-\frac{i}{6}\left[\frac{\lambda_1^3}{8}\phi^6 - \frac{\lambda_1^2}{4}\left(\phi^2P^2\phi^2 + P^2\phi^4\right)\right]\int \frac{d^dq}{(2\pi)^d} \frac{1}{\left(q^2 - M^2\right)^3 }%\nonumber\\
		% - \frac{i}{6}\frac{\lambda_1^2}{2}\left(\phi^2P_\mu\phi^2+
 		%P_\mu\phi^4\right)\int \frac{d^dq}{(2\pi)^d} \frac{q^\mu}{\left(q^2 - M^2\right)^3 }	
	\end{align}
	
	Applying $P_\mu = i\partial_\mu$ to make the derivatives explicit, we have:


	\begin{align}
		\mathcal{L}^{1-loop}_{EFT}[\phi]&\supset-\frac{i}{6}\left[\frac{\lambda_1^3}{8}\phi^6 + \frac{\lambda_1^2}{4}\left(\phi^2\Box\phi^2 + \Box\phi^4\right)\right]\int \frac{d^dq}{(2\pi)^d} \frac{1}{\left(q^2 - M^2\right)^3 }
	\end{align}
	
	To further simplify, we use the identities:
	\begin{align}
		\Box\phi^2 = 2(\partial^\mu\phi\partial_\mu\phi + \phi\Box\phi), \ \ \ \Box\phi^4 = 12 \phi^2\partial^\mu\phi\partial_\mu\phi + 4\phi^3\Box \phi
	\end{align}
	
	Substituting these, the Lagrangian becomes:
	
	\begin{align}
		\mathcal{L}^{1-loop}_{EFT}[\phi]&\supset -\frac{i}{6}\left[\frac{\lambda_1^3}{8}\phi^6 + \frac{\lambda_1^2}{2}\left(3\phi^3\Box\phi + 7\phi^2 \partial^\mu\phi\partial_\mu\phi\right)\right]\int \frac{d^dq}{(2\pi)^d} \frac{1}{\left(q^2 - M^2\right)^3 }
	\end{align}
	
	
	Using integration by parts to eliminate the redundant derivative term, we can replace $\phi^2 \partial^\mu \phi \partial_\mu \phi$ with $-\frac{1}{3}\phi^3 \Box \phi$, yielding:
	
	
	\begin{align}
		\mathcal{L}^{1-loop}_{EFT}[\phi]&\supset -\frac{i}{6}\left[\frac{\lambda_1^3}{8}\phi^6 + \frac{\lambda_1^2}{3}\phi^3\Box\phi \right]\int \frac{d^dq}{(2\pi)^d} \frac{1}{\left(q^2 - M^2\right)^3 }
	\end{align}
	
	
	Using dimensional regularization, the momentum integral evaluates to\cite{Zhang_2017}:
	
	\begin{align}
		\int \frac{d^dq}{(2\pi)^d} \frac{1}{\left(q^2 - M^2\right)^3 }= -\frac{i}{32\pi^2 M^2}
	\end{align}
	
	Thus, the final contribution from the $n=3$ term is:
	
	\begin{align}
		\mathcal{L}^{1-loop}_{EFT}[\phi]&\supset -\frac{15 \lambda_1^3}{32\pi^2} \frac{\phi^6}{6!M^2} - \frac{\lambda_1^2}{24 \pi^2}\frac{\phi^3\Box\phi}{4! M^2}\label{eq:n3}
	\end{align}
	
	Term with $n=4$:
	
	
	\begin{align}
		\mathcal{L}^{1-loop}_{EFT}[\phi]&\supset -\frac{i}{8}\int \frac{d^dq}{(2\pi)^d} \left[\frac{\left(2q\cdot P - P^2 + \frac{\lambda_1}{2}\phi^2\right)}{q^2 - M^2}\right]^4\nonumber\\
		&\supset-\frac{i}{8}\int \frac{d^dq}{(2\pi)^d}\frac{\left(2q\cdot P - P^2 + \frac{\lambda_1}{2}\phi^2\right)\left(2q\cdot P - P^2 + \frac{\lambda_1}{2}\phi^2\right)\left(2q\cdot P - P^2 + \frac{\lambda_1}{2}\phi^2\right)\left( \frac{\lambda_1}{2}\phi^2\right)}{\left(q^2 - M^2\right)^4}
	\end{align}
	
	
	Here the minimal operator dimension is five, which are those with three derivative and two fields. The relevant operator to us will be only those with dimension six, which in these case are those involving four fields and two derivatives. The rest of the contributions are of dimension higher than six, which does not interest us. Therefore, the $n=4$ will contribute only with dimension six operators:
	
	
	\begin{align}
		\mathcal{L}^{1-loop}_{EFT}[\phi]&\supset -\frac{i}{8}\lambda_1^2\left(\phi^2P_\mu P_\nu\phi^2 + P_\mu P_\nu \phi^4 + P_\mu\phi^2  P_\nu\phi^2\right)\int \frac{d^dq}{(2\pi)^d} \frac{q^\mu q^\nu}{\left(q^2 - M^2\right)^4 }\nonumber\\
		&\supset \frac{i}{8}\lambda_1^2\left(\phi^2\partial_\mu \partial_\nu\phi^2 + \partial_\mu \partial_\nu \phi^4 + \partial_\mu\phi^2  \partial_\nu\phi^2\right)\int \frac{d^dq}{(2\pi)^d} \frac{q^\mu q^\nu}{\left(q^2 - M^2\right)^4 }
	\end{align}
	
	
	We can simplify a little bit by using the following identities:
	\begin{align}
		\begin{cases}
			\phi^2\partial_\mu \partial_\nu\phi^2  = 2\phi^2 \partial_\mu \phi \partial_\nu\phi + 2\phi^3\partial_\mu \partial_\nu\phi \\
			\partial_\mu \partial_\nu \phi^4 = 12\phi^2\partial_\mu\phi\partial_\nu\phi + 4\phi^3\partial_\mu\partial_\nu\phi\\
			\partial_\mu\phi^2  \partial_\nu\phi^2 = 6\phi^2\partial_\mu\phi\partial_\nu\phi + 2\phi^3\partial_\mu\partial_\nu\phi
		\end{cases}
	\end{align}
	
	Substituting into the Lagrangian:
	
	
	
	
	
	\begin{align}
		\mathcal{L}^{1-loop}_{EFT}[\phi]&\supset \frac{i}{2}\lambda_1^2\left(2\phi^3\partial_\mu \partial_\nu\phi + 5\phi^2 \partial_\mu \phi \partial_\nu\phi\right)\int \frac{d^dq}{(2\pi)^d} \frac{q^\mu q^\nu}{\left(q^2 - M^2\right)^4 }
	\end{align}
	
	Performing the loop integral with dimensional regularization and also discarding the $\frac{2}{\bar{\epsilon}}$ term because we are using the $\overline{MS}$ scheme:
	
	\begin{align}
		\int \frac{d^dq}{(2\pi)^d} \frac{q^\mu q^\nu}{\left(q^2 - M^2\right)^4 } = -\frac{i}{192\pi^2 M^2}g^{\mu\nu}
	\end{align}
	
	Substituting this into the Lagrangian expression:
	
	\begin{align}
		\mathcal{L}^{1-loop}_{EFT}[\phi]&\supset \frac{\lambda_1^2}{384\pi^2 M^2}\left(2\phi^3\Box\phi +5 \phi^2 \partial_\mu \phi \partial^\mu\phi\right)
	\end{align}
	
	Using the integration by parts to replace $\phi^2\partial_\mu \phi \partial^\mu$ by $-\frac{1}{3}\phi^3\Box\phi$:
'	
	
	\begin{align}
		\mathcal{L}^{1-loop}_{EFT}[\phi]&\supset \frac{\lambda_1^2}{1152\pi^2 M^2}\phi^3\Box\phi  \label{eq:n4}
	\end{align}
	
	The other terms in $n> 4$ contributes from operator of higher dimension, or equivalently to order $1/M^4$. Therefore they are not interesting to us.
	
	Now we sum Eq. (\ref{eq:n1}), (\ref{eq:n2}), \ref{eq:n3}), \ref{eq:n4}) and \ref{eq:tree}) to obtain the EFT Lagrangian up to one-loop order and up to dimension six operators:
	
	  \begin{align}
	  	\mathcal{L}_{EFT}^{\text{tree}}[\phi] + \mathcal{L}^{1-loop}_{EFT}[\phi]\supset \frac{1}{2}\partial^\mu\phi\partial_\mu\phi - m_L^2 \frac{\phi^2}{2} - c_4 \frac{\phi^4}{4!}
	  	- c_6 \frac{\phi^6}{6!} - c_6'\frac{\phi^3\Box\phi}{6!} \label{eq:ex_EFT}
	  \end{align}
	
	
	%\begin{align}
	%	\mathcal{L}_{EFT}^{\text{tree}}[\phi] + \mathcal{L}^{1-loop}_{EFT}[\phi]\supset \frac{1}{2}\partial^\mu\phi\partial_\mu\phi + \left[m^2 - \frac{\lambda_1M^2}{32\pi^2}\left(\ln\left(\frac{\mu^2}{M^2}\right) + 1\right)\right] \frac{\phi^2}{2} + \left[\lambda_0 - \frac{3\lambda_1^2}{32\pi^2}\ln\left(\frac{\mu^2}{M^2}\right)\right]\frac{\phi^4}{4!} \nonumber\\
	%	- \frac{15\lambda_1^3}{32\pi^2M^2} \frac{\phi^6}{6!} - \frac{5\lambda_1^2}{8\pi^2 M^2}\frac{\phi^3\Box\phi}{6!} \label{eq:ex_EFT}
	%\end{align}
	
	\noindent where:
	
	\begin{align}
		&m_L^2 = m^2 - \frac{\lambda_1M^2}{32\pi^2}\left(\ln\left(\frac{\mu^2}{M^2}\right) + 1\right)\\
		&c_4 = \lambda_0 - \frac{3\lambda_1^2}{32\pi^2}\ln\left(\frac{\mu^2}{M^2}\right)\\
		&c_6 = \frac{15\lambda_1^3}{32\pi^2M^2} \\
		&c_6' = \frac{5\lambda_1^2}{8\pi^2 M^2}
	\end{align}
	
	As discussed in the previous section, we can simplify this result by eliminating redundant operators. Actually, we already implemented the first step of simplification using exact identities when we used integration by parts to replace $\phi^2 \partial^\mu \phi \partial_\mu \phi$ by $-\frac{1}{3}\phi^3 \Box \phi$. Therefore, we can jump to the second step and use the following field redefinition:
	
	
	\begin{align}
		\phi\to \phi \left(1 - c_6' \frac{\phi^2}{6!}\right)
	\end{align}
	
	This will eliminate the $\phi^3\Box\phi$ operator by adding contributions to $\phi^4$ and $\phi^6$ operators:
	
	\begin{align}
		\mathcal{L}_{EFT}^{\text{tree}}[\phi] + \mathcal{L}^{1-loop}_{EFT}[\phi]\supset \frac{1}{2}\partial^\mu\phi\partial_\mu\phi - m_L^2 \frac{\phi^2}{2} - \left[c_4 - \frac{m_L^2}{30}c_6'\right] \frac{\phi^4}{4!}
		- \left[c_6 + \frac{m_L^2}{2} \frac{c_6'^2}{6!} - \frac{c_4c_6'}{6} \right] \frac{\phi^6}{6!} 
	\end{align}
	
	 Recall that $m_L^2$ and $c_4$ have leading order contributions from tree-level matching, along with quantum corrections from one-loop matching, whereas $c_6$ and $c_6'$ arises purely form quantum effects. Consequently, the products $m_L^2 c_6'$ and $c_4 c_6'$ consist of two types of terms: one involving a leading-order contribution multiplied by a quantum correction, and another resulting from the product of two quantum corrections. Since we are working within a perturbative framework, quantum corrections are expected to be small, and terms involving the product of two such corrections are even smaller. Thus, retain only the leading-order times quantum correction contributions and neglecting the subleading terms is a good approximation:

	\begin{align}
		m_L^2c_6'\approx m^2 c_6', \ \ \ c_4c_6'\approx \lambda_0 c_6'
	\end{align}
	
	Furthermore, since $c_6$ and $c_6'$ are of order $1/M^2$ and $M^2 \gg m_L^2$ we can neglect the quadratic contribution of $m_L c_6'$ that appears in the $\phi^6$ coefficient:  
	
	\begin{align}
		c_6  - \frac{c_4c_6'}{6} \gg \frac{m_L^2}{2} \frac{c_6'^2}{6!} \implies c_6 + \frac{m_L^2}{2} \frac{c_6'^2}{6!} - \frac{c_4c_6'}{6} \approx c_6  - \frac{c_4c_6'}{6} 
	\end{align}  


	Taking all these approximations into account and redefining the coefficients, we can rewrite the EFT Lagrangian up to one-loop order and up to dimension six as
	
	
	
	\begin{align}
		\mathcal{L}_{EFT}^{\text{tree}}[\phi] + \mathcal{L}^{1-loop}_{EFT}[\phi]\supset  \frac{1}{2}\partial^\mu\phi \partial_\mu \phi - m_L^2 \frac{\phi}{2} - C_4 \frac{\phi^4}{4!} - \frac{C_6}{M^2} \frac{\phi^6}{6!} 
	\end{align}
	
	
	
	\noindent where the effective coefficients are given by:
	
	\begin{align}
		&m_L^2 = m^2 - \frac{\lambda_1M^2}{32\pi^2}\left(\ln\left(\frac{\mu^2}{M^2}\right) + 1\right)\\
		&C_4 = c_4 - \frac{m_L^2}{30}c_6' \approx \lambda_0 - \frac{3\lambda_1^2}{32\pi^2}\ln\left(\frac{\mu^2}{M^2}\right) - \frac{\lambda_1^2m^2}{48\pi^2 M^2}\\
		&C_6 = c_6 + \frac{m_L^2}{2} \frac{c_6'^2}{6!} - \frac{c_4c_6'}{6} \approx \frac{15\lambda_1^2}{32\pi^2M^2} - \frac{5\lambda_0\lambda_1^2}{48\pi^2}
	\end{align}
	
	Despite its simplicity, this example allowed us to apply the matching method developed in the previous section, practice identifying relevant operators up to the desired dimension, and develop a better understanding of simplification techniques used to recognize and eliminate redundant operators.
	
	%\begin{align}
	%	\partial_\mu\partial_\nu\phi^4 = 4\partial_\mu(\phi^3\partial_\nu\phi) = 12\phi^2\partial_\mu\phi\partial_\nu\phi + 4\phi^3\partial_\mu\partial_\nu\phi
	%\end{align}
	
	%\begin{align}
	%	\partial_\mu(\phi^2\partial_\nu\phi^2) = 2\partial_\mu (\phi^3\partial_\nu\phi) = 6\phi^2\partial_\mu\phi\partial_\nu\phi + 2\phi^3\partial_\mu\partial_\nu\phi
	%\end{align}
	
	%Calculation of the d'lambertian
	%\begin{align}
	%	\Box\phi^2 = \partial^\mu \partial_\mu\phi^2 = 2\partial^\mu(\phi\partial_\mu\phi) = 2(\partial^\mu\phi\partial_\mu\phi + \phi\partial^2\phi)
	%\end{align}
	
	%\begin{align}
	%	\Box\phi^4 = \partial^\mu \partial_\mu\phi^2 = 4\partial^\mu(\phi^3\partial_\mu\phi) =12 \phi^2\partial^\mu\phi\partial_\mu\phi + 4\phi^3\Box \phi
	%\end{align}
	
	
	%\begin{align}
	%	\mathcal{L}_{UV} = \frac{1}{2}\left[\partial_\mu \phi\partial^\mu \phi - m^2\phi^2 + \partial_\mu \Phi\partial^\mu \Phi - M^2\Phi^2\right] -\frac{\lambda_0}{4!} \phi^4 - \frac{\lambda_1}{3!} \phi^3 \Phi
	%\end{align}
	
	%\begin{align}
	%	\frac{\delta \mathcal{L}_{UV}}{\delta \Phi}\bigg\vert_{\Phi = \Phi_c} = 0 \implies \Phi_c(\phi) = -\frac{\lambda_1}{3!}\frac{1}{\partial^2  +M^2}\phi^3
	%\end{align}
	
	%Local expansion up to $1/M^2$ order:
	
	%\begin{align}
	%	\hat{\Phi}_c(\phi) = -\frac{\lambda_1}{6M^2}\phi^3 + O(M^{-4})
	%\end{align}
	
	
	
	%\begin{align}
	%	\mathcal{L}_{EFT}^{\text{tree}}[\phi] = \frac{1}{2}\left[\partial_\mu \phi\partial^\mu \phi - m^2\phi^2 + \partial_\mu \Phi\partial^\mu \Phi - M^2\Phi^2\right] -\frac{\lambda_0}{4!} \phi^4 - \frac{\lambda_1}{3!} \phi^3 \Phi
	%\end{align}
	
	\section{Generalization of the matching method for fermions}
	
	So far, we have applied the matching procedure to derive an effective theory in the simplest case: integrating out a heavy real scalar from a UV theory composed solely of real scalar fields. In this section, we extend this framework to UV theories that include both real and complex scalars, as well as fermions. While the fundamental principles of matching in the path integral formalism remain unchanged, new features arise due to the distinct statistical properties of bosons and fermions. In particular, scalar fields are ordinary commuting variables, whereas fermionic fields are Grassmann variables, leading to key differences in their integration and the structure of the resulting effective operators.
	
	Consider a UV theory containing both heavy and light scalar and fermionic fields, organized into multiplets as shown in Table \ref{tab:table1}. Our goal is to derive the tree-level and one-loop effective Lagrangian by integrating out the heavy fields. Since the tree-level matching procedure applied to the real scalar case did not rely on the field statistics, the same steps can be followed here. As a result, the tree-level EFT Lagrangian is obtained simply by replacing the heavy fields in the UV Lagrangian with their local classical equations of motion solution, expressed in terms of the background light fields:
	
	
	\begin{align}
		\mathcal{L}_{EFT}^{\text{tree}}[\psi,\phi] = \mathcal{L}_{UV}[\hat{\Psi}_c[\psi,\phi], \hat{\Phi}_c[\psi,\phi],\psi,\phi]
	\end{align}
	
    \noindent where, for simplicity, we omit the subscript $b$, assuming that the light fields are treated as background fields.
    
    
		\begin{table}[h!]
			\centering
			\begin{tabular}{cll}
				\toprule
				Multiplet & Components & Description \\
				\midrule
				$\Psi$ & $\left(\Omega, \Omega^C, \Xi\right)^T$ & \pbox{20cm}{$\Omega$, $\Omega^C$: heavy Dirac fermions \\
					$\Xi$: heavy Majorana fermions}  \\
				\midrule
				$\Phi$ & $\big(\Sigma, \Sigma^*, \Theta\big)^T$ & \pbox{20cm}{$\Sigma$, $\Sigma^*$: heavy complex scalars \\
					$\Theta$: heavy real scalars}  \\
				\midrule
				$\psi$ & $\big(\omega, \omega^C, \xi\big)^T$ & \pbox{20cm}{$\omega$, $\omega^C$: light Dirac fermions \\
					$\xi$: light Majorana fermions} \\
				\midrule
				$\phi$ & $\big(\sigma, \sigma^*, \theta\big)^T$ & \pbox{20cm}{$\sigma$, $\sigma^*$: light complex scalars \\
					$\theta$: light real scalars} \\
				\bottomrule
			\end{tabular}
			\caption{Composition of the multiplets involved in the calculation. The superscript on the Dirac fields indicates charge conjugation, i.e., $\omega^C = \mathcal{C} \bar{\Omega}^T$, where $\mathcal{C}$ is the charge conjugation operator.}
			\label{tab:table1}
		\end{table}
		
		At the one-loop level, the first and most significant difference when integrating out fields with different statistics, compared to the purely scalar case, appears in the evaluation of the Gaussian integral over quantum fluctuations. This step occurs when computing the one-loop perturbative interaction (1LPI) and one-particle-irreducible (1PI) generating functionals for the UV and EFT theories, respectively. The key distinction lies in how functional determinants arise based on field statistics. Specifically, bosonic fields lead to determinants in the denominator\footnote{\textcolor{red}{I'm not entirely sure, but from my understanding, when treating a complex field and its conjugate as independent degrees of freedom, the result is similar to the real scalar case. The functional integral still yields a determinant in the denominator, but with an additional square root factor, leading to $(\det\mathcal{O})^{\frac{1}{2}}$ instead of $(\det\mathcal{O})^{-1}$}}, while fermionic fields, being Grassmann variables, contribute determinants in the numerator:
		
		
		
		
		\begin{align}
			&\int D\Phi \exp\left\{ \int d^4x \Phi^\dagger \mathcal{O} \Phi\right\} = N \left(\det\mathcal{O}\right)^{-\frac{1}{2}} \ \ \ & \text{(Bosonic scalar multiplet)}\nonumber\\
			%&\int D\Sigma D\Sigma^* \exp\left\{ \int d^4x \Sigma^* \mathcal{O} \Sigma\right\} = N \left(\det\mathcal{O}\right)^{-1} \ \ \ & \text{(Complex scalar field)}\nonumber\\
			&\int D\Psi D\bar{\Psi} \exp\left\{ \int d^4x \bar{\Psi} \mathcal{O} \Psi\right\} = N \det\mathcal{O}  \ \ \ & \text{(Fermionic  multiplet)}\nonumber
		\end{align}
		
		
		
		Because the Gaussian integral depends on field statistics, and typical UV theories contain both scalar and Dirac fields, the fluctuation operator often introduces mixed terms that couple fields with different statistics. As a result, the Gaussian integral cannot be evaluated directly as in the purely scalar case. Instead, we first transform the fluctuation operator into a block-diagonal form to decouple bosonic and fermionic contributions.
		
		Rather than working directly with the fluctuation operator, we examine the second variation of the UV Lagrangian. This approach makes it easier to identify mixed terms, particularly given the presence of four distinct multiplets. Given the multiplet definitions in Table \ref{tab:table1}, the second variation of the UV Lagrangian is\footnote{The difference of signs in fermionic terms result from using ait-commutation relation between fermions and derivatives with respect to fermions}:
		

	\begin{align}
		\delta^2 \mathcal{L}_{UV} &= \delta^2 \mathcal{L}_\text{S} +\frac{1}{2}\Psi'^T \Delta_\Psi \Psi' - \frac{1}{2}  \Psi'^T \tilde{\mathbf{X}}_{\Psi \Phi}  \Phi'+\frac{1}{2} \Phi'^T \tilde{\mathbf{X}}_{\Phi \Psi} \Psi' -\frac{1}{2}  \Psi'^T \tilde{\mathbf{X}}_{\Psi \phi}  \phi'+\frac{1}{2}  \phi'^T \tilde{\mathbf{X}}_{\phi \Psi}  \Psi' \nonumber \\
		& \quad+\frac{1}{2}  \psi'^T \tilde{\mathbf{X}}_{\psi \Psi} \Psi' +\frac{1}{2} \Psi'^T \tilde{\mathbf{X}}_{\Psi \psi} \psi'+\frac{1}{2}\psi'^T \Delta_\psi \psi' -\frac{1}{2}  \psi'^T \tilde{\mathbf{X}}_{\psi \Phi}  \Phi' +\frac{1}{2}  \Phi'^T \tilde{\mathbf{X}}_{\Phi \psi} \psi' \nonumber \\ & \quad -\frac{1}{2} \psi'^T \tilde{\mathbf{X}}_{\psi \phi}  \phi'+\frac{1}{2}  \phi'^T \tilde{\mathbf{X}}_{\phi \psi}  \psi',
		\label{eq:variation-part1}
	\end{align}
	
	\noindent where the pure scalar part is given by:
	
	\begin{align}
		\delta^2\mathcal{L}_S = -\frac{1}{2}\Phi'\Delta_\Phi\Phi' - \frac{1}{2}\phi'\Delta_\phi\phi' - \frac{1}{2}\Phi'^T \tilde{\mathbf{X}}_{\Phi\phi}\phi' -\frac{1}{2}\phi'^T\tilde{\mathbf{X}}_{\phi\Phi}\Phi'\label{eq:pure_scalar_L}
	\end{align}
	
	To simplify the visualization of Eqs. (\ref{eq:variation-part1}) and (\ref{eq:pure_scalar_L}), we introduce some compact notation. For the mixed terms, we defined matrices $\tilde{\mathbf{X}}$ with elements:
	
		
	\begin{align}
		\left(X_{AB}\right)_{ij}\equiv -\frac{\delta^2 \mathcal{L}_{UV, \text{int}}}{\delta A_i \delta B_j}\bigg\vert_{A=A_b,B = B_b}
	\end{align}
	
	\noindent where $A$ and $B$ represent arbitrary scalar or fermionic fields, and $\mathcal{L}_{UV, \text{int}}$ denotes the interaction part of the UV Lagrangian. The indices $i$ and $j$ collectively account for all indices carried by the fields $A$ and $B$. Using that $\ccj$ as the charge conjugation operator, some representative examples of these matrices are:
	
	\begin{align}
		&\tilde{\mathbf{X}}_{\Psi\Phi} = \begin{pmatrix}
			X_{\Omega\Sigma} &X_{\Omega\Sigma^*} &X_{\Omega\Theta}\\
			\ccj X_{\bar{\Omega}\Sigma} & \ccj X_{\bar{\Omega}\Sigma^*} &\ccj X_{\bar{\Omega}\Theta}\\
			X_{\Xi\Sigma} &X_{\Xi\Sigma^*} &X_{\Xi\Theta}
		\end{pmatrix}, \ \ \ 
		&\tilde{\mathbf{X}}_{\Phi\Psi} = \begin{pmatrix}
			X_{\Sigma\omega} &X_{\Sigma\bar{\Omega}}\ccj^{-1} &X_{\Sigma\Xi}\\
			X_{\Sigma^*\Omega} &  X_{\Sigma^*\bar{\Omega}}\ccj^{-1} & X_{\Sigma^*\Xi}\\
			X_{\Theta\Omega} &X_{\Theta\bar{\Omega}}\ccj^{-1} &X_{\Theta\Xi}
		\end{pmatrix}, \\
		&\tilde{\mathbf{X}}_{\Psi\psi} = \begin{pmatrix}
			X_{\Omega\omega} &X_{\Omega\bar{\omega}}\ccj^{-1} &X_{\Omega\xi}\\
			\ccj X_{\bar{\Omega}\omega} &  \ccj X_{\bar{\Omega}\bar{\omega}}\ccj^{-1} & \ccj X_{\bar{\Omega}\xi}\\
			X_{\Xi\omega} &X_{\xi\bar{\omega}}\ccj^{-1} &X_{\Xi\xi}
		\end{pmatrix}, \ \ \
		&\tilde{\mathbf{X}}_{\Phi\phi} = \begin{pmatrix}
			X_{\Sigma\sigma} &X_{\Sigma\sigma^*} &X_{\Sigma\theta}\\
			X_{\Sigma^*\sigma} &  \ccj X_{\Sigma^*\sigma^*} & X_{\Sigma^*\theta}\\
			X_{\Theta\Sigma} &X_{\Theta\sigma^*} &X_{\Theta\theta}
		\end{pmatrix}, \\
	\end{align}
	
	
	\noindent The remaining $\tilde{\mathbf{X}}$ matrices follows similar definitions for substitutions $\Phi\to\phi$ and $\Psi\to\psi$. For the quadratic terms, we also introduce a compact notation:
	
	\begin{align}
		&\Delta_\Psi = \begin{pmatrix}
			X_{\Omega\Omega} & \mathcal{C}\left(\slashed{P}_{\Omega^C} - M_{\Omega} + \mathcal{C}^{-1} X_{\Omega\bar{\Omega}}\mathcal{C}^{-1}\right) &X_{\Omega\Xi}\\
			\ccj\left(\slashed{P}_\Omega- M_{\Omega} + X_{\bar{\Omega}\Omega}\right) &\ccj X_{\bar{\Omega}\bar{\Omega}}\ccj^{-1} &\ccj X_{\bar{\Omega}\Xi}\\
			X_{\Xi\Omega} &X_{\Xi\bar{\Omega}}\ccj^{-1} &\ccj \left(\slashed{P}_\Xi - M_{\Xi} + \mathcal{C}^{-1} X_{\Xi\Xi}\right)
		\end{pmatrix},\\
		&\Delta_{\Phi} = \begin{pmatrix}
			X_{\Sigma\Sigma} &-P^2_{\Sigma^*}+M^2_{\Sigma} + X_{\Sigma\Sigma^*} &X_{\Sigma\Theta}\\
			-P^2_{\Sigma} + M^2_{\Sigma} + X_{\Sigma^*\Sigma}  &X_{\Sigma^*\Sigma^*} &X_{\Sigma\Theta}\\
			X_{\Theta\Sigma}  &X_{\Theta\Sigma^*} &-P^2_{\Theta} + M_{\Theta}^2 + X_{\Theta\Theta}
		\end{pmatrix}, 
	\end{align}
	
	\noindent where $P^\mu_\Omega$ contains the generators $T_r^a$ of a representation $r$ of the group symmetry and $P^\mu_{\Omega^C}$ contains the generators $T_{\bar{r}}^a$ of a the conjugate representation $\bar{r}$. The same holds for the generators of the complex scalar fields.
	
	For a more intuitive form, we rewrite these as:
	
	\begin{align}
		\Delta_\Psi = \ccj \tilde{\id} \left(\slashed{P} - M_\Psi\right) + \tilde{\mathbf{X}}_{\Psi\Psi}, \ \ \ \Delta_{\Phi} = \tilde{\id}\left(-P^2 + M_\Phi^2\right) + \tilde{\mathbf{X}}_{\Phi\Phi},\label{eq:prop}
	\end{align}
	
	\noindent where:
	
	
	
	\begin{align}
		&\slashed{P} - M_\Psi = \begin{pmatrix}
			\slashed{P}_\Omega - M_\Omega &0 &0\\
			0 &\slashed{P}_{\Omega^C} - M_\Omega &0\\
			0 &0 &\slashed{P}_\Xi - M_\Xi
		\end{pmatrix}, \ \ \
		&\tilde{\mathbf{X}}_{\Psi\Psi} = \begin{pmatrix}
			X_{\omega\omega} &X_{\Omega\bar{\Omega}}\ccj^{-1} &X_{\Omega\Xi}\\
			\ccj X_{\bar{\Omega}\Omega} &\ccj X_{\bar{\omega}\omega}\ccj^{-1} &\ccj X_{\bar{\Omega}\Xi}\\
			X_{\Xi\Omega} &X_{\Xi\bar{\Omega}}\ccj^{-1} &X_{\Xi\Xi}
		\end{pmatrix},\\
		-&¨P^2 + M_\Phi^2 = \begin{pmatrix}
			-P^2_\Sigma + M_\Sigma^2 &0 &0\\
			0 &-P^2_{\Sigma^*} + M_\Sigma^2 &0\\
			0 &0 &-P^2_\Theta+M_\Theta^2
		\end{pmatrix}, \ \ \
		&\tilde{\mathbf{X}}_{\Phi\Phi} = \begin{pmatrix}
			X_{\Sigma\Sigma} &X_{\Sigma\Sigma^*} &X_{\Sigma\Theta}\\
			X_{\Sigma^*\Sigma} &X_{\Sigma^*\Sigma^*} &X_{\Sigma^*\Theta}\\
			X_{\Theta\Sigma} &X_{\Theta\Sigma^*} &X_{\Theta\Theta}
		\end{pmatrix},
	\end{align}
	
	Finally, for convenience, we introduced:
	
	\begin{align}
		&\tilde{\mathds{1}} = \begin{pmatrix}
			0 &\id &0\\
			\id &0 &0\\
			0 &0 &\id
		\end{pmatrix}
	\end{align}
	
	
	The quadratic terms for the light field multiplets follow the same structure and can be trivially obtained by substituting $\Psi\to\psi$ and $\Phi\to\phi$.
	
	
	
	%\begin{align}
	%	&\Delta_\Psi = \begin{pmatrix}
		%	X_{\Omega\Omega} & \mathcal{C}\left(\slashed{P}_{\Omega^C} - M_{\Omega} + \mathcal{C}^{-1} X_{\Omega\bar{\Omega}}\mathcal{C}^{-1}\right) &X_{\Omega\Xi}\\
		%	\ccj\left(\slashed{P}_\Omega- M_{\Omega} + X_{\bar{\Omega}\Omega}\right) &\ccj X_{\bar{\Omega}\bar{\Omega}}\ccj^{-1} &\ccj X_{\bar{\Omega}\Xi}\\
		%	X_{\Xi\Omega} &X_{\Xi\bar{\Omega}}\ccj^{-1} &\ccj \left(\slashed{P}_\Xi - M_{\Xi} + \mathcal{C}^{-1} X_{\Xi\Xi}\right)
		%\end{pmatrix},\\
	%	&\Delta_\psi = \begin{pmatrix}
	%		X_{\omega\omega} & \mathcal{C}\left(\slashed{P}_{\omega^C} - m_{\omega} + \mathcal{C}^{-1} X_{\omega\bar{\omega}}\mathcal{C}^{-1}\right) &X_{\omega\xi}\\
	%		\ccj\left(\slashed{P}_\omega- m_{\omega} + X_{\bar{\omega}\omega}\right) &\ccj X_{\bar{\omega}\bar{\omega}}\ccj^{-1} &\ccj X_{\bar{\omega}\xi}\\
	%		X_{\xi\omega} &X_{\xi\bar{\omega}}\ccj^{-1} &\ccj \left(\slashed{P}_\xi - m_{\xi} + \mathcal{C}^{-1} X_{\Xi\Xi}\right)
	%	\end{pmatrix},\\
	%	&\Delta_{\Phi} = \begin{pmatrix}
	%		X_{\Sigma\Sigma} &-P^2_{\Sigma^*}+M^2_{\Sigma} + X_{\Sigma\Sigma^*} &X_{\Sigma\Theta}\\
	%		-P^2_{\Sigma} + M^2_{\Sigma} + X_{\Sigma^*\Sigma}  &X_{\Sigma^*\Sigma^*} &X_{\Sigma\Theta}\\
	%		X_{\Theta\Sigma}  &X_{\Theta\Sigma^*} &-P^2_{\Theta} + M_{\Theta}^2 + X_{\Theta\Theta}
	%	\end{pmatrix}, \\
	%	&\Delta_{\phi} = \begin{pmatrix}
	%		X_{\sigma\sigma} &-P^2_{\sigma^*}+m^2_{\sigma} + X_{\sigma\sigma^*} &X_{\sigma\theta}\\
	%		-P^2_{\sigma} + m^2_{\sigma} + X_{\sigma^*\sigma}  &X_{\sigma^*\sigma^*} &X_{\sigma\theta}\\
	%		X_{\theta\sigma}  &X_{\theta\sigma^*} &-P^2_{\theta} + m_{\theta}^2 + X_{\theta\theta}
	%%	&\tilde{\mathbf{X}}_{\Psi\Phi} = \begin{pmatrix}
	%		X_{\Omega\Sigma} &X_{\Omega\Sigma^*} &X_{\Omega\Theta}\\
	%		\ccj X_{\bar{\Omega}\Sigma} & \ccj X_{\bar{\Omega}\Sigma^*} &\ccj X_{\bar{\Omega}\Theta}\\
	%		X_{\Xi\Sigma} &X_{\Xi\Sigma^*} &X_{\Xi\Theta}
	%	\end{pmatrix}, \\
	%	&\tilde{\mathbf{X}}_{\Phi\Psi} = \begin{pmatrix}
	%		X_{\Sigma\omega} &X_{\Sigma\bar{\Omega}}\ccj^{-1} &X_{\Sigma\Xi}\\
	%		 X_{\Sigma^*\Omega} &  X_{\Sigma^*\bar{\Omega}}\ccj^{-1} & X_{\Sigma^*\Xi}\\
	%		X_{\Theta\Omega} &X_{\Theta\bar{\Omega}}\ccj^{-1} &X_{\Theta\Xi}
	%	\end{pmatrix}, \\
	%	&\tilde{\mathbf{X}}_{\Psi\psi} = \begin{pmatrix}
	%		X_{\Omega\omega} &X_{\Omega\bar{\omega}}\ccj^{-1} &X_{\Omega\xi}\\
	%		\ccj X_{\bar{\Omega}\omega} &  \ccj X_{\bar{\Omega}\bar{\omega}}\ccj^{-1} & \ccj X_{\bar{\Omega}\xi}\\
	%		X_{\Xi\omega} &X_{\xi\bar{\omega}}\ccj^{-1} &X_{\Xi\xi}
	%	\end{pmatrix}, \\
	%	&\tilde{\mathbf{X}}_{\Phi\phi} = \begin{pmatrix}
	%		X_{\Sigma\sigma} &X_{\Sigma\sigma^*} &X_{\Sigma\theta}\\
	%		 X_{\Sigma^*\sigma} &  \ccj X_{\Sigma^*\sigma^*} & X_{\Sigma^*\theta}\\
	%		X_{\Theta\Sigma} &X_{\Theta\sigma^*} &X_{\Theta\theta}
	%	\end{pmatrix}. \\
	%\end{align}
 %Furthermore, the elements of these matrices are defined as:
	
%	\begin{align}
%		\left(X_{AB}\right)_{ij}\equiv -\frac{\delta^2 \mathcal{L}_{UV, \text{int}}}{\delta A_i \delta B_j}
%	\end{align}
	
%	\noindent where $A$ and $B$ represent arbitrary scalar or fermionic fields, and $\mathcal{L}_{UV, \text{int}}$ denotes the interaction part of the UV Lagrangian. The indices $i$ and $j$ collectively account for all indices carried by the fields $A$ and $B$. Additionally, if $P^\mu_\Omega$ contains generators $T_r^a$ of a representation $r$, then $P^\mu_{\Omega^C}$ contain generators of the conjugate representation $\bar{r}$, denoted by $T_{\bar{r}}^a$. The same correspondence holds between generators in $P^\mu_\Sigma$ and $P^\mu_{\Sigma^*}$.
	
	
%	\begin{align}
%		\tilde{\mathds{1}} = \begin{pmatrix}
%			0 &\id &0\\
%			\id &0 &0\\
%			0 &0 &\id
%		\end{pmatrix}
%	\end{align}
	
%	\begin{align}
%		\Delta_\Psi = \ccj \tilde{\id} \left(\slashed{P} - M_\Psi\right) + \tilde{\mathbf{X}}_{\Psi\Psi}
%	\end{align}
	
%	\begin{align}
%		&\slashed{P} - M_\Psi = \begin{pmatrix}
%			\slashed{P}_\Omega - M_\Omega &0 &0\\
%			0 &\slashed{P}_{\Omega^C} - M_\Omega &0\\
%			0 &0 &\slashed{P}_\Xi - M_\Xi
%		\end{pmatrix}\\
%		&\tilde{\mathbf{X}}_{\Psi\Psi} = \begin{pmatrix}
%			X_{\omega\omega} &X_{\Omega\bar{\Omega}}\ccj^{-1} &X_{\Omega\Xi}\\
%			\ccj X_{\bar{\Omega}\Omega} &\ccj X_{\bar{\omega}\omega}\ccj^{-1} &\ccj X_{\bar{\Omega}\Xi}\\
%			X_{\Xi\Omega} &X_{\Xi\bar{\Omega}}\ccj^{-1} &X_{\Xi\Xi}
%		\end{pmatrix}
%	\end{align}
	
%	\begin{align}
%		\Delta_{\Phi} = \tilde{\id}\left(-P^2 + M_\Phi^2\right) + \tilde{\mathbf{X}}_{\Phi\Phi}
%	\end{align}
	
	In order to write the fluctuation operator in a block diagonal form, our strategy will be to eliminate the mixed scalar and fermionic terms by redefining the fermionic fields. We start working in the terms of the second variation (Eq. (\ref{eq:variation-part1})) with at least one variation in the light fermionic multiplet:
	
	

	\begin{align}
		\delta^2 \mathcal{L}_\psi &=\frac{1}{2}  \psi'^T \tilde{\mathbf{X}}_{\psi \Psi} \Psi' +\frac{1}{2}\delta \Psi'^T \tilde{\mathbf{X}}_{\Psi \psi}  \psi'+\frac{1}{2} \psi'^T \Delta_\psi  \psi' -\frac{1}{2}  \psi'^T \tilde{\mathbf{X}}_{\psi \Phi}  \Phi' +\frac{1}{2}  \Phi'^T \tilde{\mathbf{X}}_{\Phi \psi} \psi' \nonumber \\ & \quad -\frac{1}{2}  \psi'^T \tilde{\mathbf{X}}_{\psi \phi} \phi'+\frac{1}{2}  \phi'^T \tilde{\mathbf{X}}_{\phi \psi}  \psi'
	\end{align}
	
	Defining the matrix-valued Green's function of $\Delta_\psi$
	
	
	\begin{align}
		\Delta_\psi(x) \Delta^{-1}_\psi(x,y) = \delta^{(d)}(x-y)
	\end{align}
	
	\noindent such it is understood that the product of these matrix contains an implicit integral:
	
	\begin{align}
		\Delta_\psi^{-1}f(x) \equiv \int d^d y \Delta^{-1}_\psi(x,y)f(y)
	\end{align}
	
	
	Similar to $\Delta_\psi^{-1}(x)$ we define $\overleftarrow{\Delta}_\psi^{-1}$ in such a way that:
	
	\begin{align}
		\int d^d y f(y) \overleftarrow{\Delta}_\psi^{-1}(x,y)\overleftarrow{\Delta}_\psi(x) = f(x)
	\end{align}
	
	\noindent where the left arrow indicates where the operator is acting. With these definitions of matrix-valued Green's function, the light fermionic part of the second variation can be expressed as:
	
	\begin{align}
		\delta^2 \mathcal{L}_\psi &= \frac{1}{2} \left[\psi'^T + \left(\Psi'^T \tilde{\mathbf{X}}_{\Psi\psi} + \Phi'^T\tilde{\mathbf{X}}_{\Phi\psi} + \phi'^T \tilde{\mathbf{X}}_{\phi\psi}\right)\overleftarrow{\Delta}^{-1}_\psi\right]\Delta_\psi\left[\psi' +\Delta^{-1}_\psi \left(\tilde{\mathbf{X}}_{\psi\Psi}\Psi' - \tilde{\mathbf{X}}_{\psi\phi}\Phi' - \tilde{\mathbf{X}}_{\psi\phi}\phi'\right)\right]\nonumber\\
		&\quad -\frac{1}{2}\left[\Psi'^T\tilde{\mathbf{X}}_{\Psi\psi} + \Phi'^T\tilde{\mathbf{X}}_{\Phi\psi} + \phi'^T \tilde{\mathbf{X}}_{\phi\psi}\right]\Delta^{-1}_\psi\left[\tilde{\mathbf{X}}_{\psi\Psi}\Psi' - \tilde{\mathbf{X}}_{\psi\Phi}\Phi' - \tilde{\mathbf{X}}_{\psi\phi}\phi'\right]
	\end{align}
	
	Next, we shift the light fermion multiplet as:
	
	\begin{align}
		&\psi'' = \psi' +\Delta^{-1}_\psi \left(\tilde{\mathbf{X}}_{\psi\Psi}\Psi' - \tilde{\mathbf{X}}_{\psi\phi}\Phi' - \tilde{\mathbf{X}}_{\psi\phi}\phi'\right)\\
		&\psi''^T = \psi'^T + \left(\Psi'^T \tilde{\mathbf{X}}_{\Psi\psi} + \Phi'^T\tilde{\mathbf{X}}_{\Phi\psi} + \phi'^T \tilde{\mathbf{X}}_{\phi\psi}\right)\overleftarrow{\Delta}^{-1}_\psi
	\end{align}
	
	
	\noindent under with the path integral measure is invariant. Since $\psi$ is a multiplet with Majorana-like spinors, the two shifts above are not independent. As proven in Ref. \cite{Kr_mer_2020}, it is required that:
	
	\begin{align}
		\left[\Delta_\psi^{-1}\left(\tilde{\mathbf{X}}_{\psi\Psi}\Psi' - \tilde{\mathbf{X}}_{\psi\Phi}\Phi' - \tilde{\mathbf{X}}_{\psi\phi}\phi'\right)\right]^T = \left(\Psi'^T \tilde{\mathbf{X}}_{\Psi\psi} + \Phi'^T\tilde{\mathbf{X}}_{\Phi\psi} + \phi'^T \tilde{\mathbf{X}}_{\phi\psi}\right)\overleftarrow{\Delta}^{-1}_\psi
	\end{align}
	
	After perform this shift in the light fermionic part of the second variation we obtain:
	
	
	\begin{align}
		\delta^2 \mathcal{L}_\psi &= \frac{1}{2}\psi''^T\Delta_\psi\psi'' - \frac{1}{2}\Psi'^T\tilde{\mathbf{X}}_{\Psi\psi}\Delta^{-1}_\psi \tilde{\mathbf{X}}_{\psi\Psi}\Psi' + \frac{1}{2}\Psi'^T\tilde{\mathbf{X}}_{\Psi\psi}\Delta^{-1}_\psi\tilde{\mathbf{X}}_{\psi\Phi}\Phi' + \frac{1}{2}\Psi'^T\tilde{\mathbf{X}}_{\Psi\psi}\Delta^{-1}_\psi\tilde{\mathbf{X}}_{\psi\phi}\phi'\nonumber\\
		&\quad - \frac{1}{2}\Delta\Phi^T\tilde{\mathbf{X}}_{\Phi\psi}\Delta^{-1}_\psi\tilde{\mathbf{X}}_{\psi\Psi}\Psi' + \frac{1}{2}\Delta\Phi^T\tilde{\mathbf{X}}_{\Phi\psi}\Delta^{-1}_\psi \tilde{\mathbf{X}}_{\psi\Phi}\Phi' + \frac{1}{2}\Delta\Phi^T\tilde{\mathbf{X}}_{\Phi\psi}\Delta^{-1}_\psi\tilde{\mathbf{X}}_{\psi\phi}\phi'\nonumber\\
		&\quad -\frac{1}{2}\phi'^T \tilde{\mathbf{X}}_{\phi\psi}\Delta^{-1}_\psi \tilde{\mathbf{X}}_{\psi\Psi}\Psi' + \frac{1}{2}\phi'^T \tilde{\mathbf{X}}_{\phi\psi}\Delta^{-1}_\psi\tilde{\mathbf{X}}_{\psi\Phi}\Phi' + \frac{1}{2}\phi'^T \tilde{\mathbf{X}}_{\phi\psi}\Delta^{-1}_\psi\tilde{\mathbf{X}}_{\psi\phi}\phi'
	\end{align}
	
	Now we can substitute this into Eq. (\ref{eq:variation-part1}), which will lead us to several terms. To keep the expression compact, we dfefine
	
	\begin{align}
		&\bar{\Delta}_A \equiv \Delta_{A} - \tilde{\mathbf{X}}_{A\psi}\Delta_\psi^{-1}\tilde{\mathbf{X}}_{\psi A}\\
		&\bar{\mathbf{X}}_{AB} = \tilde{\mathbf{X}}_{AB} - \tilde{\mathbf{X}}_{A\psi}\Delta^{-1}_\psi\tilde{\mathbf{X}}_{\psi B}
	\end{align}
	
	\noindent where $A$ and $B$ can be any fermionic of scalar multiplet of the theory. Using these definitions, the second variation of the UV Lagrangian takes the form:

	
	
	\begin{align}
		\delta^2\mathcal{L}_{UV} &= \delta^2\bar{\mathcal{L}}_S+ \frac{1}{2}\psi''^T\Delta_\psi \psi''+ \frac{1}{2}\Psi'^T\bar{\Delta}_{\Psi}\Psi' -\frac{1}{2} \Psi'^T\bar{\mathbf{X}}_{\Psi\Phi}\Phi' + \frac{1}{2}\Phi'^T\bar{\mathbf{X}}_{\Phi\Psi}\Psi' \nonumber\\
		&\quad - \frac{1}{2}\Psi'^T\bar{\mathbf{X}}_{\Psi\phi}\phi' + \frac{1}{2}\phi'^T\bar{\mathbf{X}}_{\phi\Psi}\Psi'
	\end{align}
	
	\noindent where:
	
	\begin{align}
		\delta^2\bar{\mathcal{L}}_S = -\frac{1}{2}\Phi'^T\bar{\Delta}_\Phi\Phi' - \frac{1}{2}\phi'\bar{\Delta}_\phi\phi' - \frac{1}{2}\Phi'^T \bar{\mathbf{X}}_{\Phi\phi}\phi' -\frac{1}{2}\phi'^T\bar{\mathbf{X}}_{\phi\Phi}\Phi'
	\end{align}
	
	
	Next, we proceed to eliminate the mixed scalar-heavy fermion terms. We first introduce $\Delta^{-1}_\Psi$ as the matrix-valued Green’s function, analogous to the light fermions. This allows us to rewrite the second variation of the UV Lagrangian as:
	
	
	
	\begin{align}
		\delta^2\mathcal{L}_{UV} &= \delta^2\bar{\mathcal{L}}_S + \frac{1}{2}\psi''^T\Delta_\psi \psi'' + \frac{1}{2}\left[\Psi'^T + \left(\Phi'^T\bar{\mathbf{X}}_{\Phi\Psi} + \phi'^T\bar{\mathbf{X}}_{\phi\Psi}\right)\overleftarrow{\bar{\Delta}}^{-1}_\Psi\right]\bar{\Delta}_\Psi\left[\Psi' - \bar{\Delta}^{-1}_\Psi\left(\bar{\mathbf{X}}_{\Psi\Phi}\Phi' + \bar{\mathbf{X}}_{\Psi\phi}\phi'\right)\right]\nonumber\\
		&\quad +\frac{1}{2}\left[\Phi'^T\bar{\mathbf{X}}_{\Phi\Psi} + \phi'^T\bar{\mathbf{X}}_{\phi\Psi}\right]\bar{\Delta}^{-1}_\Psi\left[\bar{\mathbf{X}}_{\Psi\Phi}\Phi' + \bar{\mathbf{X}}_{\Psi\phi}\phi'\right]
	\end{align}
	
	From this expression, we identify an appropriate shift for the heavy fermionic multiplet:\footnote{These shifts are not independent and satisfy a relation analogous to that for the light fermionic multiplet.}
	
	
	\begin{align}
		&\Psi'' = \Psi' + \bar{\Delta}^{-1}_\Psi\left(\bar{\mathbf{X}}_{\Psi\Phi}\Phi' + \bar{\mathbf{X}}_{\Psi\phi}\phi'\right)\\
		&\Psi''^T = \Psi'^T + \left(\Phi'^T\bar{\mathbf{X}}_{\Phi\Psi} + \phi'^T\bar{\mathbf{X}}_{\phi\Psi}\right)\overleftarrow{\bar{\Delta}}^{-1}_\Psi
	\end{align}
	
	Applying this shift to the second variation of the UV Lagrangian yields:
	
	\begin{align}
		\delta^2\mathcal{L}_{UV} &= \delta^2\bar{\mathcal{L}}_S + \frac{1}{2}\psi''^T\Delta_\psi \psi'' + \frac{1}{2}\Psi''\bar{\Delta}_\Psi\Psi'' +\Phi'^T\bar{\mathbf{X}}_{\Phi\Psi}\bar{\Delta}^{-1}_\Psi\bar{\mathbf{X}}_{\Psi\Phi}\Phi' + \Phi'^T\bar{\mathbf{X}}_{\Phi\Psi}\bar{\Delta}^{-1}_\Psi\bar{\mathbf{X}}_{\Psi\phi}\phi'\nonumber\\
		&\quad +\phi'^T\bar{\mathbf{X}}_{\phi\Psi}\bar{\Delta}^{-1}_\Psi\bar{\mathbf{X}}_{\Psi\Phi}\Phi' + \phi'^T\bar{\mathbf{X}}_{\phi\Psi}\bar{\Delta}^{-1}_\Psi\bar{\mathbf{X}}_{\Psi\phi}\phi'
	\end{align}
	
	Explicitly writing the scalar terms:
	
	\begin{align}
		\delta^2\mathcal{L}_{UV} &= -\frac{1}{2}\Phi'^T\left[\bar{\Delta}_\Phi 
		- \bar{\mathbf{X}}_{\Phi\Psi}\bar{\Delta}^{-1}_\Psi\bar{\mathbf{X}}_{\Psi\Phi}\right]\Phi' 
		- \frac{1}{2}\phi'\left[\bar{\Delta}_\phi - \bar{\mathbf{X}}_{\phi\Psi}\bar{\Delta}^{-1}_\Psi\bar{\mathbf{X}}_{\Psi\phi}\right]\phi' 
		- \frac{1}{2}\Phi'^T \left[\bar{\mathbf{X}}_{\Phi\phi} - \bar{\mathbf{X}}_{\Phi\Psi}\bar{\Delta}^{-1}_\Psi\bar{\mathbf{X}}_{\Psi\phi}\right]\phi'\nonumber\\
		&\quad -\frac{1}{2}\phi'^T\left[\bar{\mathbf{X}}_{\phi\Phi} - \bar{\mathbf{X}}_{\phi\Psi}\bar{\Delta}^{-1}_\Psi\bar{\mathbf{X}}_{\Psi\Phi}\right]\Phi'
		+ \frac{1}{2}\psi''^T\Delta_\psi \psi'' + \frac{1}{2}\Psi''\bar{\Delta}_\Psi\Psi''\nonumber\\
		&=-\frac{1}{2}\begin{pmatrix}
			\Phi'^T &\phi'^T
		\end{pmatrix}
		\begin{pmatrix}
			\bar{\Delta}_\Phi 
			- \bar{\mathbf{X}}_{\Phi\Psi}\bar{\Delta}^{-1}_\Psi\bar{\mathbf{X}}_{\Psi\Phi}
			&\bar{\mathbf{X}}_{\Phi\phi} - \bar{\mathbf{X}}_{\Phi\Psi}\bar{\Delta}^{-1}_\Psi\bar{\mathbf{X}}_{\Psi\phi}\\
			\bar{\mathbf{X}}_{\phi\Phi} - \bar{\mathbf{X}}_{\phi\Psi}\bar{\Delta}^{-1}_\Psi\bar{\mathbf{X}}_{\Psi\Phi}
			&\bar{\Delta}_\phi - \bar{\mathbf{X}}_{\phi\Psi}\bar{\Delta}^{-1}_\Psi\bar{\mathbf{X}}_{\Psi\phi}
		\end{pmatrix}
		\begin{pmatrix}
			\Phi'\\ \phi'
		\end{pmatrix}+ \frac{1}{2}\psi''^T\Delta_\psi \psi'' + \frac{1}{2}\Psi''\bar{\Delta}_\Psi\Psi''\nonumber\\
		&=-\frac{1}{2}\begin{pmatrix}
			\Phi'^T &\phi'^T
		\end{pmatrix}
		\mathcal{Q}_S
		\begin{pmatrix}
			\Phi'\\ \phi'
		\end{pmatrix}+ \frac{1}{2}\psi''^T\Delta_\psi \psi'' + \frac{1}{2}\Psi''\bar{\Delta}_\Psi\Psi''
	\end{align}
	
	\noindent where we introduce the folowing compact notation for the scalar fluctuation operator: 
	
	
	\begin{align}
		\mathcal{Q}_S \equiv \begin{pmatrix}
			\hat{\Delta}_\Phi &\hat{\mathbf{X}}_{\Phi\phi}\\
			\hat{\mathbf{X}}_{\phi\Phi} &\hat{\Delta}_{\phi}
		\end{pmatrix} \equiv \begin{pmatrix}
			\bar{\Delta}_\Phi 
			- \bar{\mathbf{X}}_{\Phi\Psi}\bar{\Delta}^{-1}_\Psi\bar{\mathbf{X}}_{\Psi\Phi}
			&\bar{\mathbf{X}}_{\Phi\phi} - \bar{\mathbf{X}}_{\Phi\Psi}\bar{\Delta}^{-1}_\Psi\bar{\mathbf{X}}_{\Psi\phi}\\
			\bar{\mathbf{X}}_{\phi\Phi} - \bar{\mathbf{X}}_{\phi\Psi}\bar{\Delta}^{-1}_\Psi\bar{\mathbf{X}}_{\Psi\Phi}
			&\bar{\Delta}_\phi - \bar{\mathbf{X}}_{\phi\Psi}\bar{\Delta}^{-1}_\Psi\bar{\mathbf{X}}_{\Psi\phi}
		\end{pmatrix}
	\end{align}
	
	
	Notice that all mixed fermionic-bosonic fluctuation terms have been eliminated. As a result, the fluctuation operator is now block-diagonal, allowing us to evaluate the path integrals separately according to the statistics of each field. The first term corresponds to the bosonic sector, while the second and third terms belong to the fermionic sector. Therefore, after splitting the fields into classical backgrounds and quantum fluctuations, and truncating the expansion at quadratic order in the fluctuations, the generating functional for the light degrees of freedom in the UV theory becomes:
	
	%\begin{align}
	%	\delta^2\mathcal{L}_{UV} &= \delta^2\mathcal{L}_{SF} + \delta^2\mathcal{L}_{F}
	%\end{align}
	
	%\noindent where:
	
	%\begin{align}
	%	&\delta^2\mathcal{L}_{F} = \frac{1}{2}\psi''^T\Delta_\psi \psi'' + \frac{1}{2}\Psi''\bar{\Delta}_\Psi\Psi''\\
	%	&\delta^2\mathcal{L}_{SF} = -\frac{1}{2}\begin{pmatrix}
	%		\Phi'^T &\phi'^T
	%	\end{pmatrix}\mathcal{Q}_S \begin{pmatrix}
	%		\Phi'\\ \phi'
	%	\end{pmatrix}
	%\end{align}
	
	
	
	
	\begin{align}
		Z_{L,UV}[J_\phi,J_\psi]&=Z_{UV}^{\text{tree}}[J_\phi,J_\psi] \int D\Phi'D\phi' \exp\left\{-\frac{i}{2}\int d^dx\begin{pmatrix}
				\Phi'^T &\phi'^T
			\end{pmatrix}\mathcal{Q}_S \begin{pmatrix}
				\Phi'\\ \phi'
		\end{pmatrix}\right\}\nonumber\\
		&\quad \int D\Psi'' \exp\left\{ \frac{i}{2}\int d^dx\psi''^T\Delta_\psi \psi'' \right\}\int D\Psi''\exp\left\{\frac{i}{2}\int d^dx\Psi''\bar{\Delta}_\Psi\Psi''\right\}\nonumber\\
		&\propto Z_{UV}^{\text{tree}}\left(\det\mathcal{Q_S}\right)^{-\frac{1}{2}}\det\Delta_\psi \det\bar{\Delta}_\Psi
	\end{align}
	
	Performing the Legendre transformation to obtain the $1LPI$ generating functional and already separating the one-loop contributions:
	
	\begin{align}
		\Gamma^{\text{1-loop}}_{L,UV}[\phi_b,\psi_b] = \frac{i}{2}\ln{\det Q_S´} -i\ln{\det \Delta_\psi} - i\ln{\det\bar{\Delta}_\Psi}
	\end{align}
	
	By the procedure used in the real scalar case, the next step is to compute the $1PI$ quantum action for the EFT and then perform the one-loop matching. However, recall that when we applied the expansion-by-regions method in one-loop matching, the role of the EFT was to cancel the non-vanishing soft contributions of the UV theory. This reflects the physical insight that the one-loop EFT Lagrangian captures the short-distance behavior of the full theory, while the long-distance effects are already encoded in the tree-level EFT. Therefore, by analogy with the real scalar case, the one-loop spacetime integral of the EFT Lagrangian must match the hard contribution of the one-loop $1PI$ action of the UV theory \cite{Fuentes_Mart_n_2016}:
	
	\begin{align}
		\int d^dx \mathcal{L}_{EFT}^{\text{1-loop}}[\phi_b,\psi_b] = \left[\frac{i}{2}\ln{\det Q_S´} -i\ln{\det \Delta_\psi} - i\ln{\det\bar{\Delta}_\Psi}\right]_{\text{hard}} = \left[\frac{i}{2}\ln{\det Q_S´}  - i\ln{\det\bar{\Delta}_\Psi}\right]_{\text{hard}}
	\end{align}
		
	\noindent where the contribution $\ln \det \Delta_\psi$ vanishes in the hard region, since it involves only light fields.
	
	To make the computation as transparent as possible, we separate the one-loop EFT Lagrangian into two distinct contributions: one from the bosonic path integral and another from the fermionic path integral,
	
	\begin{align}
		\mathcal{L}^{\text{1-loop}}_{EFT}[\phi_b,\psi_b]= \mathcal{L}^{\text{1-loop}}_{EFT,SF}[\phi_b,\psi_b] + \mathcal{L}^{\text{1-loop}}_{EFT,F}[\phi_b,\psi_b]
	\end{align}
	
	\noindent with:
	
	\begin{align}
			\int d^dx \mathcal{L}^{\text{1-loop}}_{EFT,SF}[\phi_b,\psi_b] = \frac{i}{2}\ln\det\mathcal{Q}_S\big\vert_{\text{hard}}\\
			\int d^dx\mathcal{L}^{\text{1-loop}}_{EFT,F}[\phi_b,\psi_b] = - i\ln{\det\bar{\Delta}_\Psi}\big\vert_{\text{hard}}
	\end{align}
	
	Let us first compute the bosonic contribution. Following the same steps as in the real scalar case, we begin by rewriting $\mathcal{Q}_S$ in block-diagonal form via field redefinitions:
	\begin{align}
		\mathcal{Q}'_S =\begin{pmatrix}
			\hat{\Delta}_\Phi - \hat{\mathbf{X}}_{\Phi\phi}\hat{\Delta}_\phi^{-1}\hat{\mathbf{X}}_{\phi\Phi}&0\\
			0&\hat{\Delta}_\phi
		\end{pmatrix}
	\end{align}
	
	Therefore, the bosonic one-loop contribution becomes:
	
	\begin{align}
		\int d^dx \mathcal{L}^{\text{1-loop}}_{EFT,SF}[\phi_b,\psi_b] = \frac{i}{2}\left[\ln\det\left(\hat{\Delta}_\Phi - \hat{\mathbf{X}}_{\Phi\phi}\hat{\Delta}_\phi^{-1}\hat{\mathbf{X}}_{\phi\Phi}\right) + \ln\det\hat{\Delta}_\phi\right]_{\text{hard}}
	\end{align}
	
	
	
	Following the exact same steps as in section~\ref{subsec:Func_determinants}, we write:
	
	\begin{align}
		\mathcal{L}^{\text{1-loop}}_{EFT,SF}[\phi_b,\psi_b] = \frac{i}{2}\int \frac{d^dq}{(2\pi)^d}\left[\tr\ln\left(\hat{\Delta}_\Phi - \hat{\mathbf{X}}_{\Phi\phi}\hat{\Delta}_\phi^{-1}\hat{\mathbf{X}}_{\phi\Phi}\right) +  \tr\ln\hat{\Delta}_\phi\right]\bigg\vert^{P\to P-q}_{\text{hard}}
	\end{align}
	
	Note that, in contrast to the real scalar case, $\hat{\Delta}\phi$ now includes contributions from heavy fermions. This is a consequence of the field redefinitions used to block-diagonalize the fluctuation operator and separate fields according to their statistics. As a result, $\tr\ln \hat{\Delta}\phi$ do not vanishes entirely in the hard region. However, we can isolate the non-vanishing part by unpacking the hatted operator definition:
	
	
	\begin{align}
		\tr\ln\hat{\Delta}_\phi = \tr\ln\left(\bar{\Delta}_\phi - \bar{\mathbf{X}}_{\phi\Psi}\bar{\Delta}^{-1}_\Psi\bar{\mathbf{X}}_{\Psi\phi}\right) = \tr\ln\bar{\Delta}_\phi + \tr\ln\left( 1 - \bar{\Delta}_{\phi}^{-1}\bar{\mathbf{X}}_{\phi\Psi}\bar{\Delta}^{-1}_\Psi\bar{\mathbf{X}}_{\Psi\phi}\right)
	\end{align}
	
	
	\noindent Here, the first term on the right-hand side contains only light-field contributions (both fermionic and bosonic) and thus vanishes in the hard region. The second term, however, contains heavy fermions terms and remains non-zero. Therefore:
	
	\begin{align}
		\mathcal{L}^{\text{1-loop}}_{EFT,SF}[\phi_b,\psi_b] = \frac{i}{2}\int \frac{d^dq}{(2\pi)^d}\left[\tr\ln\left(\hat{\Delta}_\Phi - \hat{\mathbf{X}}_{\Phi\phi}\hat{\Delta}_\phi^{-1}\hat{\mathbf{X}}_{\phi\Phi}\right) +  \tr\ln\left( 1 - \bar{\Delta}_{\phi}^{-1}\bar{\mathbf{X}}_{\phi\Psi}\bar{\Delta}^{-1}_\Psi\bar{\mathbf{X}}_{\Psi\phi}\right)\right]\bigg\vert^{P\to P-q}_{\text{hard}}\label{eq:L_SF}
	\end{align}
	
	In principle, one can proceed further by expressing the hatted and barred operators in terms of the original second variation quantities and applying the CDE method, as done in section (\ref{subsec:CDE_Scalar}). However, this generates several terms without offering significant insight. Instead, we will illustrate the procedure explicitly through a concrete example.
	
	We now proceed to evaluate the contribution from the fermionic path integral:
	
	\begin{align}
		\int d^dx\mathcal{L}^{\text{1-loop}}_{EFT,F}[\phi_b,\psi_b] = - i\ln{\det\bar{\Delta}_\Psi}\big\vert_{\text{hard}} = -i\Tr\ln\left(\Delta_\Psi - \tilde{\mathbf{X}}_{\Psi\psi}\Delta_\psi^{-1}\tilde{\mathbf{X}}_{\psi \Psi}\right)\big\vert_{\text{hard}}
	\end{align}
	
	Following the steps of section \ref{subsec:Func_determinants} on hoe to calculate functional determinants:
	
	\begin{align}
		\mathcal{L}^{\text{1-loop}}_{EFT,F}[\phi_b,\psi_b] = -i\int \frac{d^dq}{(2\pi)^d}\tr\ln\left(\Delta_\Psi - \tilde{\mathbf{X}}_{\Psi\psi}\Delta_\psi^{-1}\tilde{\mathbf{X}}_{\psi \Psi}\right)\bigg\vert_{\text{hard}}^{P\to P-q}
	\end{align}
	
	Using the structure of $\Delta_\Psi$ (Eq. (\ref{eq:prop}), we rewrite the integrand as:


	\begin{align}
		\mathcal{L}^{\text{1-loop}}_{EFT,F}[\phi_b,\psi_b] &= -i\int \frac{d^dq}{(2\pi)^d}\tr\ln\left(\ccj \tilde{\id} \left(\slashed{P} - M_\Psi\right) + \tilde{\mathbf{X}}_{\Psi\Psi} - \tilde{\mathbf{X}}_{\Psi\psi}\Delta_\psi^{-1}\tilde{\mathbf{X}}_{\psi \Psi}\right)\bigg\vert_{\text{hard}}^{P\to P-q}\nonumber\\
		&=-i\int \frac{d^dq}{(2\pi)^d}\left\{\tr\ln\left(\ccj\tilde{\id}(-\slashed{q}-M_\Psi)\right) + \tr\ln\left[1 - \frac{ \slashed{P}  +\tilde{\id}\ccj^{-1}\left( \tilde{\mathbf{X}}_{\Psi\Psi} - \tilde{\mathbf{X}}_{\Psi\psi}\Delta_\psi^{-1}\tilde{\mathbf{X}}_{\psi \Psi}\right)^{P\to P-q}}{(\slashed{q} + M_\Psi)}\right]\right\}_{\text{hard}}
	\end{align}
	
	The first term is a field-independent constant and can be absorbed into the normalization of the path integral. Before proceeding with the CDE, it is convenient to eliminate explicit factors of $\tilde{\id}$ and $\ccj$ by defining:
	
	
	%\begin{align}
	%	\mathcal{L}^{\text{1-loop}}_{EFT,F}[\phi_b,\psi_b] &=-i\int \frac{d^dq}{(2\pi)^d}  \tr\ln\left[1 - \frac{ \slashed{P}  +\tilde{\id}\ccj^{-1}\left( \tilde{\mathbf{X}}_{\Psi\Psi} - \tilde{\mathbf{X}}_{\Psi\psi}\Delta_\psi^{-1}\tilde{\mathbf{X}}_{\psi \Psi}\right)^{P\to P-q}}{(\slashed{q} + M_\Psi)}\right]_{\text{hard}}
	%\end{align}
	
	
	\begin{align}
		\mathbf{X}_{\Psi A} = \tilde{\id}\ccj^{-1}\tilde{\mathbf{X}}_{\Psi A}, \ \ A\in\{\Psi,\psi\}
	\end{align}
	
	With this, the expression becomes:
	
	\begin{align}
		\mathcal{L}^{\text{1-loop}}_{EFT,F}[\phi_b,\psi_b] &=-i\int \frac{d^dq}{(2\pi)^d}  \tr\ln\left[1 - \frac{ \slashed{P}  +\left( \mathbf{X}_{\Psi\Psi} - \mathbf{X}_{\Psi\psi}\Delta_\psi^{-1}\tilde{\mathbf{X}}_{\psi \Psi}\right)^{P\to P-q}}{(\slashed{q} + M_\Psi)}\right]_{\text{hard}} \label{eq:1-loopF}
	\end{align}
	
	Although $\Delta_\psi$ still contains $\ccj$ and $\tilde{\id}$ explicitly, these can also be absorbed. To see this, we apply Eq.~(\ref{eq:prop}) with $\Psi \to \psi$ and expand:
	
	\begin{align}
		\Delta_\psi^{-1}\big\vert_{\text{hard}} &= \left[\ccj\tilde{\id}(\slashed{P} - \slashed{q} - M_\psi) + \tilde{\mathbf{X}}_{\psi\psi}\right]^{-1}\bigg\vert_{\text{hard}}\nonumber\\
		& = \left[1 - \frac{\tilde{\id}\ccj^{-1} (\ccj\tilde{\id}\slashed{P} + \tilde{\mathbf{X}}_{\psi\psi})}{(\slashed{q}+M_\psi)}\right]^{-1}(-\slashed{q}-M_\psi)^{-1}\tilde{\id}\ccj^{-1}\bigg\vert_{\text{hard}}\label{eq:Taylor}
	\end{align}
	
	In the hard region, the loop momentum scales as $q \sim M_\Psi, M_\Phi$, which is much larger than the light fermion mass $M_\psi$, any external momenta, or the operators in $ \tilde{\mathbf{X}}_{\psi\psi}$. The latter is at most of zeroth order in the heavy mass expansion, since it contains two light fermion fields and does not correspond to a mass term. As a result, the term involving these light quantities appears suppressed relative to the dominant hard scale. This hierarchy justifies expanding the relevant expression as a Taylor series in powers of small ratios involving light fields and derivatives over the large momentum $q$:
	

	
	
	\begin{align}
		\Delta_\psi^{-1}\big\vert_{\text{hard}} & = \sum^{+\infty}_{n=0} \left[ \frac{\tilde{\id}\ccj^{-1} (\ccj\tilde{\id}\slashed{P} + \tilde{\mathbf{X}}_{\psi\psi})}{(\slashed{q}+M_\psi)}\right]^{n}(-\slashed{q}-M_\psi)^{-1}\tilde{\id}\ccj^{-1}\bigg\vert_{\text{hard}}\nonumber\\
		& = \sum^{+\infty}_{n=0} \left[ \frac{ (\slashed{P} + \mathbf{X}_{\psi\psi})}{(\slashed{q}+M_\psi)}\right]^{n}(-\slashed{q}-M_\psi)^{-1}\tilde{\id}\ccj^{-1}\bigg\vert_{\text{hard}}
	\end{align}
	
	\noindent where we have defined:

	\begin{align}
		\mathbf{X}_{\psi\psi} \equiv \tilde{\id}\ccj^{-1}\tilde{\mathbf{X}}_{\psi\psi}
	\end{align}
	
	Therefore, $\Delta_\psi^{-1}$ in the hard region always appears with a rightmost factor of $\tilde{\id} \ccj^{-1}$. As a result, in any product involving $\Delta_\psi^{-1}$ and some $\tilde{\mathbf{X}}_{AB}$, these factors can be absorbed via redefinitions. For example:
	
	\begin{align}
		\Delta_\psi^{-1}\tilde{\mathbf{X}}_{\psi\Psi} = \sum^{+\infty}_{n=0} \left[ \frac{ (\slashed{P} + \mathbf{X}_{\psi\psi})}{(\slashed{q}+M_\psi)}\right]^{n}(-\slashed{q}-M_\psi)^{-1}\mathbf{X}_{\psi\Psi}\bigg\vert_{\text{hard}}, \ \ \ \mathbf{X}_{\psi\Psi} \equiv \tilde{\id}\ccj^{-1}\tilde{\mathbf{X}}_{\psi\Psi}
	\end{align}
	
	The same reasoning applies to insertions of $\bar{\Delta}\psi^{-1}$. A similar property holds for the operators $\tilde{\mathbf{X}}_{\Phi B}$ and $\tilde{\mathbf{X}}_{\phi B}$, which always appear as $\mathbf{X}_{\Phi B} = \tilde{\id} \tilde{\mathbf{X}}_{\Phi B}$ and $\mathbf{X}_{\phi B} = \tilde{\id} \tilde{\mathbf{X}}_{\phi B}$. Hence, the full result can ultimately be expressed in terms of the redefined operators $\mathbf{X}_{AB}$, without any explicit factors of $\tilde{\id}$ or $\ccj$ appearing in the final operator expressions.
		
	Returning to the computation of the one-loop EFT Lagrangian from the fermionic path integral (Eq.\ (\ref{eq:1-loopF})), we can now proceed with the covariant derivative expansion (CDE). As in the case of the real scalar field and in Eq. (\ref{eq:Taylor}), the argument of the logarithm contains a small fraction due to the fact that the loop momentum lies in the hard region. Consequently, by the same reasoning presented earlier, we can expand the logarithm as a Taylor series:
	
	\begin{align}
		\mathcal{L}^{\text{1-loop}}_{EFT,F}[\phi_b,\psi_b] &= i\sum_{n=0}^{\infty}\frac{1}{n} \int \frac{d^d q}{(2\pi)^d}   \tr\left[ \frac{ \slashed{P} + \left( \mathbf{X}_{\Psi\Psi} - \mathbf{X}_{\Psi\psi} \, \Delta_\psi^{-1} \, \tilde{\mathbf{X}}_{\psi\Psi} \right) \big|_{P \to P - q} }{ \slashed{q} + M_\Psi } \right]^n_{\text{hard}} \label{eq:1-loopF1}
	\end{align}
	
	This expression, combined with Eq.\ (\ref{eq:L_SF}), gives the full one-loop EFT Lagrangian resulting from integrating out heavy fermions and scalars. We will further develop some of these expressions and elaborate on the details in the context of an explicit example.
	
	It is worth noting that this is not the most general result for integrating out heavy fields. In particular, one can also include contributions from heavy vector bosons. Although we will not consider those here, the relevant discussion can be found in Ref. \cite{BenjaminBispinor}.
	
	\section{DM toy model}
	
	\begin{align}
		\mathcal{L}_{BSM} = \bar{\psi}_T \left( i \gamma^\mu D_\mu -M_T \right) \psi_T - \frac{m_s^2}{2} s^2 + y_{DM} \left( \bar{\psi}_T t_R + \bar{t}_R \psi_T \right) s - V(s) 
	\end{align}
	
	\textcolor{red}{Considering the Higgs as a dublet, the top sector interaction lagrangian is}
	
	\begin{align}
		\mathcal{L}_{t} = \bar{\psi}_T \left( i \gamma^\mu D_\mu -M_T \right) \psi_T + \frac{1}{2}(\partial_\mu s)^2 - \frac{m_s^2}{2} s^2 + y_{DM} \left( \bar{\psi}_T t_R + \bar{t}_R \psi_T \right) s -y_{Hs} H^\dagger H s^2- V(s) 
	\end{align}
	
	\textcolor{red}{When integrating out its necessary to consider the Higgs coupling with the scalar, top, top partner(?)}
	
	
	\textcolor{red}{Considering the Higgs as a real singlet (125GeV resonance)}
	
	\begin{align}
		\mathcal{L}_{t} = \bar{\psi}_T \left( i \gamma^\mu D_\mu -M_T \right) \psi_T+ \frac{1}{2}(\partial_\mu s)^2 - \frac{m_s^2}{2} s^2 - y_{DM} \left( \bar{\psi}_T t_R + \bar{t}_R \psi_T \right) s  - y_{Ht}h\left(\bar{t}_L t_R + \bar{t}_R t_L\right)- V(s) 
	\end{align}
	
	
	\begin{align}
		V(s) = \frac{1}{4}y_{Hs} h^2 s^2 + \frac{\lambda_s}{4!}s^4 + \frac{\lambda_h}{4!}h^4 
	\end{align}
	
	Identification of the fields multiplets:
	
	\begin{align}
		&\Phi = s\\
		&\phi = h\\
		&\Psi = \psi_T\\
		&\psi = t
	\end{align}
	
	\subsection{Solving the equation of motion of the heavy fields}
	Equations of motion for the heavy fields:
	
	
	\begin{align}
		&\left[\square + M_s^2+\frac{1}{2}y_{Hs} h^2 + \frac{\lambda_s}{6}s_c^2 \right]s_c = - y_{DM}\left( \bar{\psi}_T t_R + \bar{t}_R \psi_T \right)\nonumber\\
		&\left(i\slashed{D}^{\psi_{T}} - M_T\right)\psi_{T,c} = y_{DM} t_R s_c\nonumber\\
		&\left(i\slashed{D}^{\psi_{T}} + M_T\right)\bar{\psi}_{T,c} = -y_{DM}\bar{t}_R s_c
	\end{align}
	
	\begin{align}
		&\psi_{T,c} =  y_{DM} \left(i\slashed{D}^{\psi_{T}} - M_T\right)^{-1}t_R s_c\\
		&\bar{\psi}_{T,c} = - y_{DM} \left(i\slashed{D}^{\psi_{T}} + M_T\right)^{-1}\bar{t}_R s_c
	\end{align}
	
	\begin{align}
		&\left[\square + M_s^2+2y_{Hs} h^2\right]s_c =  y_{DM}^2\left[ \left(\left(i\slashed{D}^{\psi_{T}} + M_T\right)^{-1}\bar{t}_R s_c\right) t_R - \bar{t}_R  \left(i\slashed{D}^{\psi_{T}} - M_T\right)^{-1}\bar{t}_R s_c\right]\nonumber\\
		&\left[\square + M_s^2+2y_{Hs} h^2  + y_{DM}^2\left(i\slashed{D}^{\psi_{T}} - M_T\right)^{-1}\bar{t}_R - y_{DM}^2 t_R\left(i\slashed{D}^{\psi_{T}} + M_T\right)^{-1}\bar{t}_R \right]s_c = 0\nonumber\\
		&\hspace{3cm} \implies   s_c = 0 \implies \psi_{T,c} = 0 \ \ \ \text{and} \ \ \ \bar{\psi}_{T,c} = 0
	\end{align}
	
	\textcolor{red}{The classical solution for the heavy fields are indeed zero, i checked using the tree-level matching of matchete wich suppose to be only a substitution of the equations of motion in the Uv Lagrangian.}
	
	%\begin{align}
	%	&s_c = \frac{y_{DM}}{m_S^2}\left(1+\frac{\square}{m_S^2}\right)^{-1}\left( \bar{\psi}_T t_R + \bar{t}_R \psi_T \right) = \frac{y_{DM}}{m_S^2}\left( \bar{\psi}_T t_R + \bar{t}_R \psi_T \right) + \mathcal{O}(m_S^{-4})\nonumber\\
	%	\implies &s_c\approx \frac{y_{DM}}{m_S^2}\left( \bar{\psi}_T t_R + \bar{t}_R \psi_T \right) 
	%\end{align}
	
	
	%\begin{align}
	%	\psi_{T,c} &= \frac{y_{DM}}{M_T}\frac{1}{(\frac{i\slashed{D}}{M_T} - 1)} t_R \nonumber\\
	%	&= \frac{y_{DM}^2}{M_T^2 M_s^2}\frac{1}{\left(\frac{i\slashed{D}}{M_T} - 1\right)\left(1 + \frac{\square}{M_s^2}\right)}\nonumber\\
	%\end{align}	
	
	%\begin{align}
	%	\left(\frac{i\slashed{D}}{M_T} - 1\right)\psi_{T,c} = \frac{y_{DM}}{M_T}t_R\frac{1}{\left(1 + \frac{\square}{M_s^2}\right)} \left( \bar{\psi}_T t_R + \bar{t}_R \psi_T \right) 
	%\end{align}
	
	\subsection{Tree-level matching}
	
	Substituting the equation of motion of the heavy fields in the UV Lagrangian:
	
	\begin{align}
		\mathcal{L}^{\text{tree}}_{EFT}[\phi_b,\psi_b] = \frac{1}{2} (\partial_\mu h)^2 -\frac{1}{2}m_h^2 h^2 + \bar{t}\left(i\slashed{D} -m_t\right)t + i\bar{q}\slashed{D} q- y_{Ht}h\bar{t}t
	\end{align}
	
	\subsection{One-loop matching}
	Interaction matrices:
	
	
	\begin{align}
		&\tilde{\mathbf{X}}_{\Psi\Psi} =\begin{pmatrix}
			X_{\psi_T\psi_T} & X_{\psi_T \bar{\psi}_T}\ccj^{-1} &0\\
			\ccj X_{\bar{\psi}_T \psi_T} &\ccj X_{\bar{\psi}_T \bar{\psi}_T}\ccj^{-1} &0\\
			0 &0 &0
		\end{pmatrix} =0_{3\times3}\\
		&\tilde{\mathbf{X}}_{\Psi\psi}  = \begin{pmatrix}
			X_{\psi_T\psi_T} & X_{\psi_T \bar{t}}\ccj^{-1} &0\\
			\ccj X_{\bar{\psi}_T t} &\ccj X_{\bar{\psi}_T \bar{t}}\ccj^{-1}&0\\
			0 &0 &0
		\end{pmatrix} = y_{DM} s_c 
		\begin{pmatrix} 
			0 &P_L\ccj^{-1} &0 \\
			\ccj P_R &0 &0\\
			0 &0 &0
		\end{pmatrix} = 0_{3\times3}\\
		&\tilde{\mathbf{X}}_{\psi\Psi}  = \begin{pmatrix}
			X_{t \psi_T} & X_{t \bar{\psi}_T}\ccj^{-1} &0\\
			\ccj X_{\bar{t} \psi_T} &\ccj X_{\bar{t} \bar{\psi}_T}\ccj^{-1} &0\\
			0 &0 &0
		\end{pmatrix} = y_{DM} s_c 
		\begin{pmatrix} 
			0 &P_L \ccj^{-1} &0\\
			\ccj P_R &0 &0\\
			0 &0 &0
		\end{pmatrix} = 0_{3\times3}%\frac{y_{DM}^2}{m_S^2}\left( \bar{\psi}_T t_R + \bar{t}_R \psi_T \right)  
		%\begin{pmatrix} 
		%P_R &0 \\
		%0 &P_L
		%\end{pmatrix} 
		\\
		&\tilde{\mathbf{X}}_{\Psi\phi} = \begin{pmatrix}
			0&0 &X_{\psi_T h}\\ 
			0&0&\ccj X_{\bar{\psi}^T h}\\
			0&0 &0
		\end{pmatrix} =  0_{3\times3}\\
		&\tilde{\mathbf{X}}_{\phi\Psi} =\begin{pmatrix}
			0 &0 &0\\
			0&0&0\\
			X_{ h\psi_T} & X_{h\bar{\psi}^T }\ccj^{-1} &0
		\end{pmatrix}  = 0_{3\times3}\\
		&\tilde{\mathbf{X}}_{\Psi\Phi} =  \begin{pmatrix}
			0&0&X_{\psi_T s} \\0&0&\ccj X_{\bar{\psi}_T s} \\0&0&0
		\end{pmatrix}= y_{DM}\begin{pmatrix}
			0&0&\bar{t}_R\\ 0&0&\ccj t_R\\ 0&0&0
		\end{pmatrix}\\
		&\tilde{\mathbf{X}}_{\Phi\Psi} =\begin{pmatrix}
			0&0&0 \\0&0&0\\ X_{s \psi_T } & X_{s\bar{\psi}_T }\ccj^{-1}&0 
		\end{pmatrix} =  y_{DM}\begin{pmatrix}
			0&0&0\\0&0&0\\
			\bar{t}_R & t_R\ccj^{-1} &0
		\end{pmatrix}\\
		&\tilde{\mathbf{X}}_{\psi\psi}  = \begin{pmatrix}
			X_{t t} & X_{t \bar{t}}\ccj^{-1} &0\\
			\ccj X_{\bar{t} t} &\ccj X_{\bar{t} \bar{t}}\ccj^{-1} &0\\
			0&0&0
		\end{pmatrix} = y_{Ht} \begin{pmatrix}
		0 &h \ccj^{-1} &0\\
		\ccj h & 0 &0\\
		0&0&0
		\end{pmatrix} \\
		&\tilde{\mathbf{X}}_{\psi\Phi} = \begin{pmatrix}
			0& 0 &X_{t s} \\ 0&0&\ccj X_{\bar{t} s}\\ 0&0&0
		\end{pmatrix} = y_{DM} \begin{pmatrix}
		0&0 &\bar{\psi}_{T,c} P_R \\ 0&0& \ccj P_L\psi_{T,c} \\0&0&0
		\end{pmatrix} =   0_{3\times3}\\
		& \tilde{\mathbf{X}}_{\Phi\psi} = \begin{pmatrix}
			0&0&0\\ 0&0&0\\ X_{s t} &X_{s\bar{t}}\ccj^{-1} &0
		\end{pmatrix} = y_{DM}\begin{pmatrix}
		0&0&0\\ 0&0&0\\
		\bar{\psi}_{T,c}P_R &P_L\psi_{T,c}\ccj^{-1} &0
		\end{pmatrix} =  0_{3\times3}\\
		&\tilde{\mathbf{X}}_{\psi\phi} = \begin{pmatrix}
			0&0 &X_{t h}\\ 0&0& \ccj X_{\bar{t} h} \\ 0&0&0
		\end{pmatrix} = y_{Ht}\begin{pmatrix}
			0&0& \bar{t} \\ 0&0& \ccj t \\ 0&0&0
		\end{pmatrix}\\
		&\tilde{\mathbf{X}}_{\phi\psi}  = \begin{pmatrix}
			0&0&0\\ 0&0&0\\ X_{h t} &X_{h \bar{t}}\ccj^{-1} &0
		\end{pmatrix} = y_{Ht}\begin{pmatrix}
			\bar{t} & t\ccj^{-1}
		\end{pmatrix}\\
		&\tilde{\mathbf{X}}_{\Phi\phi} = \begin{pmatrix}
			0 &0 &0\\ 0&0&0 \\ 0&0 &X_{sh}
		\end{pmatrix} = y_{Hs}h s_c = \begin{pmatrix}
		0 &0 &0\\ 0&0&0 \\ 0&0 &1
		\end{pmatrix}= 0_{3\times 3}\\
		&\tilde{\mathbf{X}}_{\Phi\Phi} = \begin{pmatrix}
			0 &0 &0\\ 0&0&0 \\ 0&0 &X_{ss}
		\end{pmatrix}  = \left(\frac{1}{2} y_{Hs} h^2 + \frac{\lambda_s}{2}s_c^2\right)\begin{pmatrix}
		0 &0 &0\\ 0&0&0 \\ 0&0 &1
		\end{pmatrix} = \frac{1}{2} y_{Hs} h^2 \begin{pmatrix}
		0 &0 &0\\ 0&0&0 \\ 0&0 &1
		\end{pmatrix}\\
		&\tilde{\mathbf{X}}_{\phi\phi} = \begin{pmatrix}
			0 &0 &0\\ 0&0&0 \\ 0&0 &X_{hh}
		\end{pmatrix} = \left(\frac{1}{2} y_{Hs} s_c^2 + \frac{\lambda_h}{2}h^2\right)\begin{pmatrix}
		0 &0 &0\\ 0&0&0 \\ 0&0 &1
		\end{pmatrix} = \frac{\lambda_h}{2}h^2\begin{pmatrix}
		0 &0 &0\\ 0&0&0 \\ 0&0 &1
		\end{pmatrix}
	\end{align}
	
	bar interaction matrices:
	
	
	\begin{align}
		&\bar{\mathbf{X}}_{\Psi\Psi} = \tilde{\mathbf{X}}_{\Psi\Psi} - \tilde{\mathbf{X}}_{\Psi\psi}\Delta^{-1}_\psi\tilde{\mathbf{X}}_{\psi \Psi}  = 0_{3\times 3}\\
		&\bar{\mathbf{X}}_{\Psi\psi}  = \tilde{\mathbf{X}}_{\Psi\psi} - \tilde{\mathbf{X}}_{\Psi\psi}\Delta^{-1}_\psi\tilde{\mathbf{X}}_{\psi \psi} = 0_{3\times 3}\\
		&\bar{\mathbf{X}}_{\psi\Psi}  = \tilde{\mathbf{X}}_{\psi\Psi} - \tilde{\mathbf{X}}_{\psi\psi}\Delta^{-1}_\psi\tilde{\mathbf{X}}_{\psi \Psi} = 0_{3\times 3}
		\\
		&\bar{\mathbf{X}}_{\Psi\phi} = \tilde{\mathbf{X}}_{\Psi\phi} - \tilde{\mathbf{X}}_{\Psi\psi}\Delta^{-1}_\psi\tilde{\mathbf{X}}_{\psi \phi} = 0_{3\times 3}\label{eq:X_Psiphi}\\
		&\bar{\mathbf{X}}_{\phi\Psi} =\tilde{\mathbf{X}}_{\phi\Psi} - \tilde{\mathbf{X}}_{\phi\psi}\Delta^{-1}_\psi\tilde{\mathbf{X}}_{\psi \Psi} = 0_{3\times 3}\label{eq:X_phiPsi}\\
		&\bar{\mathbf{X}}_{\Psi\Phi} =\tilde{\mathbf{X}}_{\Psi\Phi} - \tilde{\mathbf{X}}_{\Psi\psi}\Delta^{-1}_\psi\tilde{\mathbf{X}}_{\psi \Phi} = \tilde{\mathbf{X}}_{\Psi\Phi} \\
		&\bar{\mathbf{X}}_{\Phi\Psi} =\tilde{\mathbf{X}}_{\Phi\Psi} - \tilde{\mathbf{X}}_{\Phi\psi}\Delta^{-1}_\psi\tilde{\mathbf{X}}_{\psi \Psi} = \tilde{\mathbf{X}}_{\Phi\Psi} \\
		&\bar{\mathbf{X}}_{\psi\psi}  = \tilde{\mathbf{X}}_{\psi\psi} - \tilde{\mathbf{X}}_{\psi\psi}\Delta^{-1}_\psi\tilde{\mathbf{X}}_{\psi \psi}  \\
		&\bar{\mathbf{X}}_{\psi\Phi} = \tilde{\mathbf{X}}_{\psi\Phi} - \tilde{\mathbf{X}}_{\psi\psi}\Delta^{-1}_\psi\tilde{\mathbf{X}}_{\psi \Phi} = 0_{3\times 3}\\
		& \bar{\mathbf{X}}_{\Phi\psi} = \tilde{\mathbf{X}}_{\Phi\psi} - \tilde{\mathbf{X}}_{\Phi\psi}\Delta^{-1}_\psi\tilde{\mathbf{X}}_{\psi \psi} = 0_{3\times 3}\\
		&\bar{\mathbf{X}}_{\phi\psi}  = \tilde{\mathbf{X}}_{\phi\psi} - \tilde{\mathbf{X}}_{\phi\psi}\Delta^{-1}_\psi\tilde{\mathbf{X}}_{\psi \psi} \\
		&\bar{\mathbf{X}}_{\Phi\phi} = \tilde{\mathbf{X}}_{\Phi\phi} - \tilde{\mathbf{X}}_{\Phi\psi}\Delta^{-1}_\psi\tilde{\mathbf{X}}_{\psi \phi} = 0_{3\times 3}\\
		&\bar{\mathbf{X}}_{\Phi\Phi} = \tilde{\mathbf{X}}_{\Phi\Phi} - \tilde{\mathbf{X}}_{\Phi\psi}\Delta^{-1}_\psi\tilde{\mathbf{X}}_{\psi \Phi} = \tilde{X}_{\Phi\Phi}\\
		&\bar{\mathbf{X}}_{\phi\phi} = \tilde{\mathbf{X}}_{\phi\phi} - \tilde{\mathbf{X}}_{\phi\psi}\Delta^{-1}_\psi\tilde{\mathbf{X}}_{\psi \phi} 
		\end{align}
	
		bar inverse propagators:
		
		\begin{align}
			&\bar{\Delta}_\Psi = \Delta_{\Psi} - \tilde{\mathbf{X}}_{\Psi\psi}\Delta_\psi^{-1}\tilde{\mathbf{X}}_{\psi \Psi} = \Delta_{\Psi}\\
			&\bar{\Delta}_\psi = \Delta_{\psi} - \tilde{\mathbf{X}}_{\psi\psi}\Delta_\psi^{-1}\tilde{\mathbf{X}}_{\psi \psi} \\
			&\bar{\Delta}_\Phi = \Delta_{\Phi} - \tilde{\mathbf{X}}_{\Phi\psi}\Delta_\psi^{-1}\tilde{\mathbf{X}}_{\psi \Phi} = \Delta_{\Phi}\\
			&´\bar{\Delta}_\phi = \Delta_{\phi} - \tilde{\mathbf{X}}_{\phi\psi}\Delta_\psi^{-1}\tilde{\mathbf{X}}_{\psi \phi} 
		\end{align}
	

	

	\textcolor{red}{With only this results we can say that $\mathcal{L}^{\text{1-loop}}_{EFT,F}[\phi_b,\psi_b] $ does not give any contribution}
	
	\begin{align}
		\mathcal{L}^{\text{1-loop}}_{EFT,SF}[\phi_b,\psi_b] = \frac{i}{2}\int \frac{d^dq}{(2\pi)^d}\left[\tr\ln\left(\hat{\Delta}_\Phi - \hat{\mathbf{X}}_{\Phi\phi}\hat{\Delta}_\phi^{-1}\hat{\mathbf{X}}_{\phi\Phi}\right) +  \tr\ln\left( 1 - \bar{\Delta}_{\phi}^{-1}\bar{\mathbf{X}}_{\phi\Psi}\bar{\Delta}^{-1}_\Psi\bar{\mathbf{X}}_{\Psi\phi}\right)\right]\bigg\vert^{P\to P-q}_{\text{hard}}
	\end{align}
	
		\begin{align}
		\mathcal{L}^{\text{1-loop}}_{EFT,SF}[\phi_b,\psi_b] = \frac{i}{2}\int \frac{d^dq}{(2\pi)^d}\left[\tr\ln\left(\bar{\Delta}_\Phi 
		- \bar{\mathbf{X}}_{\Phi\Psi}\bar{\Delta}^{-1}_\Psi\bar{\mathbf{X}}_{\Psi\Phi} - \hat{\mathbf{X}}_{\Phi\phi}\hat{\Delta}_\phi^{-1}\hat{\mathbf{X}}_{\phi\Phi}\right) +  \tr\ln\left( 1 - \bar{\Delta}_{\phi}^{-1}\bar{\mathbf{X}}_{\phi\Psi}\bar{\Delta}^{-1}_\Psi\bar{\mathbf{X}}_{\Psi\phi}\right)\right]\bigg\vert^{P\to P-q}_{\text{hard}}
	\end{align}
	
	
	\begin{align}
		\mathcal{L}^{\text{1-loop}}_{EFT,SF}[\phi_b,\psi_b] = \frac{i}{2}\int \frac{d^dq}{(2\pi)^d}\left[\tr\ln\left(\Delta_\Phi 
		- \tilde{\mathbf{X}}_{\Phi\Psi}{\Delta}^{-1}_\Psi\tilde{\mathbf{X}}_{\Psi\Phi} - \hat{\mathbf{X}}_{\Phi\phi}\hat{\Delta}_\phi^{-1}\hat{\mathbf{X}}_{\phi\Phi}\right) \right]\bigg\vert^{P\to P-q}_{\text{hard}}
	\end{align}
	
	Notice that:
	
	\begin{align}
		\hat{\mathbf{X}}_{\Phi\phi}\hat{\Delta}_\phi^{-1}\hat{\mathbf{X}}_{\phi\Phi} &= \left(\bar{\mathbf{X}}_{\Phi\phi} - \bar{\mathbf{X}}_{\Phi\Psi}\bar{\Delta}^{-1}_\Psi\bar{\mathbf{X}}_{\Psi\phi}\right)\left(\bar{\Delta}_\phi - \bar{\mathbf{X}}_{\phi\Psi}\bar{\Delta}^{-1}_\Psi\bar{\mathbf{X}}_{\Psi\phi}\right)^{-1}\left(\bar{\mathbf{X}}_{\phi\Phi} - \bar{\mathbf{X}}_{\phi\Psi}\bar{\Delta}^{-1}_\Psi\bar{\mathbf{X}}_{\Psi\Phi}\right) = 0
	\end{align}
	
	
	
	\begin{align}
		\mathcal{L}^{\text{1-loop}}_{EFT,SF}[\phi_b,\psi_b] = \frac{i}{2}\int \frac{d^dq}{(2\pi)^d}\left[\tr\ln\left(\Delta_\Phi 
		- \tilde{\mathbf{X}}_{\Phi\Psi}{\Delta}^{-1}_\Psi\tilde{\mathbf{X}}_{\Psi\Phi} \right) \right]\bigg\vert^{P\to P-q}_{\text{hard}}
	\end{align}
	
	\begin{align}
		\Delta_\Phi = \tilde{\id}(-P^2_\Phi+M_\Phi^2) + \tilde{\mathbf{X}}_{\Phi\Phi} %= \tilde{\id}(-P^2+M_s^2) + 2 y_{Hs}h^2
	\end{align}
	
	
	\begin{align}
		\mathcal{L}^{\text{1-loop}}_{EFT,SF}[\phi_b,\psi_b] = \frac{i}{2}\int \frac{d^dq}{(2\pi)^d}\left\{\tr\ln\left( M_\Phi^2 -q^2\right) + \tr\ln\left[ 1-\frac{\left(2q\cdot P_\Phi - P_\Phi^2 +\tilde{\id}\left(\tilde{\mathbf{X}}_{\Phi\Phi}- \tilde{\mathbf{X}}_{\Phi\Psi}{\Delta}^{-1}_\Psi\tilde{\mathbf{X}}_{\Psi\Phi}\right)_{P\to P-q}\right)}{q^2 - M_\Phi^2} \right]\right\} \bigg\vert_{\text{hard}}
	\end{align}
	
	
	\begin{align}
		\mathcal{L}^{\text{1-loop}}_{EFT,SF}[\phi_b,\psi_b] = \frac{i}{2}\int \frac{d^dq}{(2\pi)^d}  \tr\ln\left[ 1-\frac{\left(2q\cdot P_\Phi - P_\Phi^2 +\left(\mathbf{X}_{\Phi\Phi}-\tilde{\id} \tilde{\mathbf{X}}_{\Phi\Psi}{\Delta}^{-1}_\Psi\tilde{\mathbf{X}}_{\Psi\Phi}\right)_{P\to P-q}\right)}{q^2 - M_\Phi^2} \right]_{\text{hard}}
	\end{align}
	
	\begin{align}
		\mathcal{L}^{\text{1-loop}}_{EFT,SF}[\phi_b,\psi_b] = -\frac{i}{2}\sum_{n=1}^{\infty} \frac{1}{n}\int \frac{d^dq}{(2\pi)^d}  \tr\left[\frac{2q\cdot P_\Phi - P_\Phi^2 +\left(\mathbf{X}_{\Phi\Phi}-\tilde{\id} \tilde{\mathbf{X}}_{\Phi\Psi}{\Delta}^{-1}_\Psi\tilde{\mathbf{X}}_{\Psi\Phi}\right)_{P\to P-q}}{q^2 - M_\Phi^2}\right]^n\bigg\vert_{\text{hard}}\label{eq:L_1_loop_exp}
	\end{align}
	
	We are interested in computing the effective Lagrangian up to dimension six operators. Therefore, considering the expansion above, where for $n=1$ we have the contribution:
	
	\begin{align}
		\mathcal{L}^{\text{1-loop}}_{EFT,SF}[\phi_b,\psi_b] \supset \frac{i}{2}\int \frac{d^dq}{(2\pi)^d}  \tr\frac{\left(\tilde{\id} \tilde{\mathbf{X}}_{\Phi\Psi}{\Delta}^{-1}_\Psi\tilde{\mathbf{X}}_{\Psi\Phi}\right)_{P\to P-q}}{q^2 - M_\Phi^2}\bigg\vert_{\text{hard}}
	\end{align}
	
	we need to calculate $\tilde{\id} \tilde{\mathbf{X}}_{\Phi\Psi}{\Delta}^{-1}_\Psi\tilde{\mathbf{X}}_{\Psi\Phi}\vert_{P\to P-q}$ only up to dimension six operators, since the $n\geq2$ contributions always gives equal or higher dimension operators than $n=1$.
	
	Calculating $\tilde{\id} \tilde{\mathbf{X}}_{\Phi\Psi}{\Delta}^{-1}_\Psi\tilde{\mathbf{X}}_{\Psi\Phi}\vert_{P\to P-q}$ performing an expansion in the hard region:
	\begin{align}
		\tilde{\id}\tilde{\mathbf{X}}_{\Phi\Psi}\Delta_\Psi^{-1}\tilde{\mathbf{X}}_{\Psi\Phi}\big\vert_{\text{hard}} &= \sum^{+\infty}_{l=0} \tilde{\id}\tilde{\mathbf{X}}_{\Phi\Psi}  \left[(-\slashed{q}-M_\Psi)(-\slashed{P}_{\Psi} - \mathbf{X}_{\Psi\Psi})\right]^l(-\slashed{q}-M_\Psi)^{-1}\mathbf{X}_{\Psi\Phi}\bigg\vert_{\text{hard}}\nonumber\\
		& =  \sum^{+\infty}_{l=0} \tilde{\id}\tilde{\mathbf{X}}_{\Phi\Psi} \frac{ \left[(-\slashed{q}+M_\Psi)(-\slashed{P}_{\Psi} - \mathbf{X}_{\Psi\Psi})\right]^l(-\slashed{q}+ M_\Psi)}{(q^2-M_\Psi^2)^{l+1}}\mathbf{X}_{\Psi\Phi}\bigg\vert_{\text{hard}}
	\end{align}
	
	Using that
	
	\begin{align}
		&\mathbf{X}_{\Psi\Phi} = \tilde{\id}\ccj^{-1}\tilde{\mathbf{X}}_{\Psi\Phi} = y_{DM} \tilde{\id}\ccj^{-1}\begin{pmatrix}
			0 &0 &\bar{t}_R\\
		0 & 0 &\ccj t_R\\
			0 &0 &0
		\end{pmatrix} = y_DM \begin{pmatrix}
		0 &0 & t_R\\
		0 & 0 &\ccj^{-1}\bar{t}_R\\
		0 &0 &0
		\end{pmatrix}\\
		& \tilde{\id}\tilde{\mathbf{X}}_{\Phi\Psi} = y_{DM}\tilde{\id}\begin{pmatrix}
			0&0&0\\0&0&0\\
			\bar{t}_R & t_R\ccj^{-1} &0
		\end{pmatrix} =  y_{DM}\begin{pmatrix}
		0&0&0\\0&0&0\\
		\bar{t}_R & t_R\ccj^{-1} &0
		\end{pmatrix}\\
		&\mathbf{X}_{\Psi\Psi} = 0_{3\times 3}
	\end{align}
	
	we get:
	
	\begin{align}
		\tilde{\id}\tilde{\mathbf{X}}_{\Phi\Psi}\Delta_\Psi^{-1}\tilde{\mathbf{X}}_{\Psi\Phi}\big\vert_{\text{hard}} &=y_{DM}^2 \sum^{+\infty}_{l=0} \begin{pmatrix}
			0&0&0\\0&0&0\\
			\bar{t}_R & t_R\ccj^{-1} &0
		\end{pmatrix}  \frac{ \left[(-\slashed{q}+M_T)(-\slashed{P}_{\Psi} )\right]^l(-\slashed{q}+ M_T)}{(q^2-M_T^2)^{l+1}}\begin{pmatrix}
		0 &0 & t_R\\
		0 & 0 &\ccj^{-1}\bar{t}_R\\
		0 &0 &0
		\end{pmatrix}\bigg\vert_{\text{hard}}\nonumber\\
		&% = y_{DM}^2 \sum^{+\infty}_{l=0} \begin{pmatrix}
		%\end{pmatrix} \begin{pmatrix}
		%-\slashed{P}_{\psi_T} &0\\
		%0 &\slashed{P}_{\psi_T^C}
		%\end{pmatrix}^l
		%\frac{ (-\slashed{q}+ M_T)^{l+1}}{(q^2-M_T^2)^{l+1}}
		%\begin{pmatrix}
		%	t_R\\ \ccj^{-1}\bar{t}_R
		%\end{pmatrix}\bigg\vert_{\text{hard}}
	\end{align}
	
	
	
	Since the covariant derivative matrix is proportional to the identity, the expression can be rewritten as%\footnote{Notice that analyzing the spinor index contractions in the term that has explicit charge conjugation operator appearing, we can see that the fields are actually transposed, otherwise the product makes no sense. While calculating the interaction matrices we cannot predict that, so we must analyze carefully when this cases appears.}:
	
	\begin{align}
		\tilde{\id}\tilde{\mathbf{X}}_{\Phi\Psi}\Delta_\Psi^{-1}\tilde{\mathbf{X}}_{\Psi\Phi}\big\vert_{\text{hard}} &=y_{DM}^2 \sum^{+\infty}_{l=0} \begin{pmatrix}
			\bar{t}_R & t_R\ccj^{-1} 
		\end{pmatrix}  \frac{ \left[(-\slashed{q}+M_T)(-\slashed{P}_{\Psi} )\right]^l(-\slashed{q}+ M_T)}{(q^2-M_T^2)^{l+1}}\begin{pmatrix}
			 t_R\\ \ccj^{-1}\bar{t}_R
		\end{pmatrix}\bigg\vert_{\text{hard}}
		\end{align} 
		
		
	\noindent where:
	
	\begin{align}
		\slashed{P}_{\Psi} = \begin{pmatrix}
			\slashed{P}_{\psi_T} &0\\ 0 &\slashed{P}_{\psi_T}^\ccj
		\end{pmatrix}
	\end{align}
	%\begin{align}
	%\tilde{\id}\tilde{\mathbf{X}}_{\Phi\Psi}\Delta_\Psi^{-1}\tilde{\mathbf{X}}_{\Psi\Phi}\big\vert_{\text{har%d}} &=y_{DM}^2 \sum^{+\infty}_{l=0}\Bigg[ \bar{t}_R\frac{ \left[(-\slashed{q}+M_T)(-\slashed{P}_{\psi_T} %)\right]^l(-\slashed{q}+ M_T)}{(q^2-M_T^2)^{l+1}} t_R \nonumber\\ &\hspace{5cm}+ t^T_R\ccj^{-1} \frac{ %\left[(-\slashed{q}+M_T)(-\slashed{P}_{\psi_T^C} )\right]^l(-\slashed{q}+ M_T)}{(q^2-M_T^2)^{l+1}} %\ccj^{-1}\bar{t}^T_R\Bigg]
	%\end{align}
	

	Each spinor field has dimension $3/2$, while each covariant derivative has dimension $1$. As a consequence the $l=1$ term has dimension 4, the $l=2$ has dimension $5$ and $l=3$ has dimension 6. Therefore, for our purpose, we will truncate the sum at $l=3$
	
	
		
	Substituting this into \ref{eq:L_1_loop_exp} and using that:
	
	\begin{align}
		\tilde{\mathbf{X}}_{\Phi\Phi} =  \frac{1}{2} y_{Hs} h^2 \begin{pmatrix}
			0 &0 &0\\ 0&0&0 \\ 0&0 &1
		\end{pmatrix}\\
	\end{align}
	
	
	we get:
	
	\begin{align}
		\mathcal{L}^{\text{1-loop}}_{EFT,SF}[\phi_b,\psi_b] = -\frac{i}{2}\sum_{n=1}^{\infty} \frac{1}{n}\int \frac{d^dq}{(2\pi)^d}  \tr\Bigg[\frac{2q\cdot P_\Phi - P_\Phi^2 +\frac{1}{2} y_{Hs} h^2}{q^2 - M_\Phi^2}- \frac{\tilde{\id} \tilde{\mathbf{X}}_{\Phi\Psi}{\Delta}^{-1}_\Psi\tilde{\mathbf{X}}_{\Psi\Phi}}{q^2 - M_\Phi^2}\Bigg]^n\bigg\vert_{\text{hard}}
	\end{align}
	
	
	
	
	\begin{align}
		\mathcal{L}^{\text{1-loop}}_{EFT,SF}[\phi_b,\psi_b] &= -\frac{i}{2}\sum_{n=1}^{\infty} \frac{1}{n}\int \frac{d^dq}{(2\pi)^d}  \tr\Bigg[\frac{2q\cdot P_\Phi - P_\Phi^2 +\frac{1}{2} y_{Hs} h^2}{q^2 - M_\Phi^2}\nonumber\\
		&\hspace{2.cm}- \frac{y_{DM}^2}{q^2 - M_\Phi^2} \sum^{+\infty}_{l=3} \begin{pmatrix}
			\bar{t}_R & t_R\ccj^{-1} 
		\end{pmatrix}  \frac{ \left[(-\slashed{q}+M_T)(-\slashed{P}_{\Psi} )\right]^l(-\slashed{q}+ M_T)}{(q^2-M_T^2)^{l+1}}\begin{pmatrix}
			t_R\\ \ccj^{-1}\bar{t}_R
		\end{pmatrix}\Bigg]^n\bigg\vert_{\text{hard}}
	\end{align}
	
	Before we proceed and calculate the terms in the $l$ summation, we need to clarify some aspects that will reduce the number of terms. Lets analyze the first values of $n$:
	
	
	\subsubsection*{$n=1$ contribution}
	
	The $P_\Phi$ terms does not contributes since they act on the identity on the right. Therefore:
	
	\begin{align}
		&\mathcal{L}^{\text{1-loop}}_{EFT,SF}[\phi_b,\psi_b] \supset\nonumber\\
		-&\frac{i}{2}\int \frac{d^dq}{(2\pi)^d}  \tr\Bigg[\frac{\frac{1}{2} y_{Hs} h^2}{q^2 - M_\Phi^2}
		- \frac{y_{DM}^2}{q^2 - M_\Phi^2} \sum^{+\infty}_{l=3} \begin{pmatrix}
			\bar{t}_R & t_R\ccj^{-1} 
		\end{pmatrix}  \frac{ \left[(-\slashed{q}+M_T)(-\slashed{P}_{\Psi} )\right]^l(-\slashed{q}+ M_T)}{(q^2-M_T^2)^{l+1}}\begin{pmatrix}
			t_R\\ \ccj^{-1}\bar{t}_R
		\end{pmatrix}\Bigg]\bigg\vert_{\text{hard}} \label{eq:n=1}
	\end{align}
	
	Here all terms will give us operators up to dimension six.
	
	Let's calculate the operators coming from the $l$ summation:
	
	\begin{itemize}
		\item $l=0$: the single contribution of this term is:
		
		\begin{align}
			\begin{pmatrix}
				\bar{t}_R & t_R\ccj^{-1} 
			\end{pmatrix}  \frac{ (-\slashed{q}+ M_T)}{(q^2-M_T^2)}\begin{pmatrix}
				t_R\\ \ccj^{-1}\bar{t}_R
			\end{pmatrix} = \bar{t}_R \frac{(-\slashed{q} + M_T)}{(q^2 - M_T^2)}t_R + t^T_R \ccj^{-1}\frac{(-\slashed{q} + M_T)}{(q^2 - M_T^2)}\ccj^{-1}\bar{t}^T_R
		\end{align}
		
		Since we have right-hand projectors in the fields, only the terms with an odd number of gamma matrices between the fields will not vanish. To see why, consider a generic term with $m$ gamma matrices::
		
		\begin{align}
			\bar{t}_R \underbrace{\gamma^\mu...\gamma^\nu}_{m \ \text{times}} t_R = (P_R t)^\dagger \gamma^0 \gamma^\mu...\gamma^\nu P_R t = t^\dagger P_R \gamma^0 \gamma^\mu...\gamma^\nu P_R t 
		\end{align}
		
		Using that $P_{R, L} \gamma^\mu = \gamma^\mu P_{L,R} $:
		
		\begin{align}
			\bar{t}_R \gamma^\mu...\gamma^\nu t_R = \bar{t} P_L \gamma^\mu...\gamma^\nu P_R t &= 
			\begin{cases}
				\bar{t}  \gamma^\mu...\gamma^\nu P_R^2 t, \, \text{for odd $m$}\\
				\bar{t}\gamma^\mu...\gamma^\nu P_L P_R t, \, \text{for an even $m$}
			\end{cases}\nonumber\\
			& = \begin{cases}
				\bar{t}  \gamma^\mu...\gamma^\nu P_R t, \, \text{for odd $m$}\\
				0, \, \text{for even $m$}
			\end{cases}
		\end{align}
		
		Therefore, the non vanishing contributions in $l=0$ are:
		
		\begin{align}
			\begin{pmatrix}
				\bar{t}_R & t_R\ccj^{-1} 
			\end{pmatrix}  \frac{ (-\slashed{q}+ M_T)}{(q^2-M_T^2)}\begin{pmatrix}
				t_R\\ \ccj^{-1}\bar{t}_R
			\end{pmatrix} =-\bar{t}_R \frac{\slashed{q}}{(q^2 - M_T^2)}t_R  - t^T_R\ccj^{-1}\frac{\slashed{q}}{(q^2 - M_T^2)}\ccj^{-1}\bar{t}^T_R
		\end{align}
		
		
		Using that $\ccj^{-1} = -\ccj$, $\ccj \gamma^\mu \ccj^{-1} = -(\gamma^\mu)^T$ and when contracting spinor indices we have $\psi^T \Gamma^T \bar{\psi}^T = - \bar{\psi}\Gamma\psi$:
		
		\begin{align}
			\begin{pmatrix}
				\bar{t}_R & t_R\ccj^{-1} 
			\end{pmatrix}  \frac{ (-\slashed{q}+ M_T)}{(q^2-M_T^2)}\begin{pmatrix}
				t_R\\ \ccj^{-1}\bar{t}_R
			\end{pmatrix} &=-\bar{t}_R \frac{\slashed{q}}{(q^2 - M_T^2)}t_R  - t^T_R \frac{(\gamma^\mu)^Tq_\mu}{(q^2 - M_T^2)}\bar{t}^T_R\nonumber\\
			&=-\bar{t}_R \frac{\slashed{q}}{(q^2 - M_T^2)}t_R  + \bar{t}_R \frac{\slashed{q}}{(q^2 - M_T^2)}t_R\nonumber\\
			&=0\label{eq:l=0}
		\end{align}
		
		This result show us that there is no contribution for $l=0$.
		
		\item $l=1$:
		
		
			\begin{align}
			\begin{pmatrix}
				\bar{t}_R & t_R\ccj^{-1} 
			\end{pmatrix}  \frac{ (-\slashed{q}+M_T)(-\slashed{P}_{\Psi} )(-\slashed{q}+ M_T)}{(q^2-M_T^2)^{2}}\begin{pmatrix}
				t_R\\ \ccj^{-1}\bar{t}_R
			\end{pmatrix}
		\end{align}
		
		Since terms with an even number of gamma matrices vanishes, the non-null contributions for the numerator of the expression above are:
		
		\begin{align}
			(-\slashed{q}+M_T)(-\slashed{P}_{\Psi} )(-\slashed{q}+ M_T) = -\slashed{q}\slashed{P}_{\Psi}\slashed{q} + M_T^2\slashed{P}_{\Psi}
		\end{align}
		
		Using the anticommutation relation of the gamma matrices:
		
		\begin{align}
			(-\slashed{q}+M_T)(-\slashed{P}_{\Psi} )(-\slashed{q}+ M_T) &= -2q\cdot  P_{\Psi}\slashed{q}+\slashed{P}_{\Psi}\slashed{q}\slashed{q} - M_T^2\slashed{P}_{\Psi}\nonumber\\
			&= -2q\cdot  P_{\Psi}\slashed{q}+\slashed{P}_{\Psi}q^2 - M_T^2\slashed{P}_{\Psi}
		\end{align}
		
		In the $n=1$ contribution for the effective Lagrangian, the loop momenta appears only in the $l$ summation. Therefore, we can anticipate some steps that are done when using dimensional regularization to compute the loop integral. Due to symmetry, we substitute $q^\mu q^\nu \to \frac{g^{\mu\nu}}{4} q^2$ when computing the loop integral. Then, if we already do this:
		
		\begin{align}
			(-\slashed{q}+M_T)(-\slashed{P}_{\Psi} )(-\slashed{q}+ M_T) 
			&= -\frac{1}{2}  \slashed{P}_{\Psi}q^2+\slashed{P}_{\Psi}q^2 - M_T^2\slashed{P}_{\Psi}\nonumber\\
			&=\frac{1}{2}  \slashed{P}_{\Psi}q^2- M_T^2\slashed{P}_{\Psi}\nonumber\\
			&= \left(\frac{1}{2}q^2- M_T^2\right)\slashed{P}_{\Psi}
		\end{align}
		
		Substituting this in the $l=1$ contribution for the summation and using:
		
		\begin{align}
			\slashed{P}_{\Psi} = \begin{pmatrix}
				\slashed{P}_{\psi_T} &0\\ 0 &\slashed{P}_{\psi_T}^\ccj
			\end{pmatrix}
		\end{align}
		
		we get:
		
		
		
		\begin{align}
			\begin{pmatrix}
				\bar{t}_R & t_R\ccj^{-1} 
			\end{pmatrix}  \frac{ (-\slashed{q}+M_T)(-\slashed{P}_{\psi_T} )(-\slashed{q}+ M_T)}{(q^2-M_T^2)^{2}}\begin{pmatrix}
				t_R\\ \ccj^{-1}\bar{t}_R
			\end{pmatrix} = \frac{\left(\frac{1}{2}q^2- M_T^2\right)}{(q^2 - M_T^2)^2}\left[\bar{t}_R\slashed{P}_{\Psi}t_R + t^T_R \ccj^{-1}\slashed{P}_{\psi_T}\ccj^{-1}\bar{t}^T_R \right]
		\end{align}
		
		Rewriting the second term using the same identities of the $l=0$ case:
		\begin{align}
			\begin{pmatrix}
				\bar{t}_R & t_R\ccj^{-1} 
			\end{pmatrix}  \frac{ (-\slashed{q}+M_T)(-\slashed{P}_{\Psi} )(-\slashed{q}+ M_T)}{(q^2-M_T^2)^{2}}\begin{pmatrix}
				t_R\\ \ccj^{-1}\bar{t}_R
			\end{pmatrix} &= \frac{\left(\frac{1}{2}q^2- M_T^2\right)}{(q^2 - M_T^2)^2}\left[\bar{t}_R\slashed{P}_{\psi_T}t_R + t^T_R (\gamma^\mu)^T P_{\psi_T, \mu}\bar{t}^T_R \right]\nonumber\\
			&= \frac{\left(\frac{1}{2}q^2- M_T^2\right)}{(q^2 - M_T^2)^2}\left[\bar{t}_R\slashed{P}_{\psi_T}t_R - \bar{t}_R \gamma^\mu P^T_{\psi_T, \mu}t_R \right]
		\end{align}
		
		Here the transpose in the derivative means that it acts on the left. Therefore, to keep this clear we can introduce an arrow to indicate the action of the covariant derivative:
		
		\begin{align}
			\begin{pmatrix}
				\bar{t}_R & t_R\ccj^{-1} 
			\end{pmatrix}  \frac{ (-\slashed{q}+M_T)(-\slashed{P}_{\Psi} )(-\slashed{q}+ M_T)}{(q^2-M_T^2)^{2}}\begin{pmatrix}
				t_R\\ \ccj^{-1}\bar{t}_R
			\end{pmatrix} &=\frac{\left(\frac{1}{2}q^2- M_T^2\right)}{(q^2 - M_T^2)^2}\left[\bar{t}_R\slashed{P}_{\psi_T}t_R - \bar{t}_R \overleftarrow{\slashed{P}}_{\psi_T}t_R \right]
		\end{align}
		
		Using that $P_\mu = i D_\mu$ and keeping the chirality projectors explicitly:
		
		\begin{align}
			\begin{pmatrix}
				\bar{t}_R & t_R\ccj^{-1} 
			\end{pmatrix}  \frac{ (-\slashed{q}+M_T)(-\slashed{P}_{\Psi} )(-\slashed{q}+ M_T)}{(q^2-M_T^2)^{2}}\begin{pmatrix}
				t_R\\ \ccj^{-1}\bar{t}_R
			\end{pmatrix} &=i\frac{\left(\frac{1}{2}q^2- M_T^2\right)}{(q^2 - M_T^2)^2}\left[\bar{t}\, \gamma^\mu\, P_R\, D_\mu t - D_\mu \bar{t} \,\gamma^\mu\, P_R \, t \right]\label{eq:l=1}
 		\end{align}
 		
 		\item $l=2$: 
 		
 		The $l=2$ contribution ig given by:
 		
 		\begin{align}
 			\begin{pmatrix}
 				\bar{t}_R & t_R\ccj^{-1} 
 			\end{pmatrix}  \frac{ (-\slashed{q}+M_T)(-\slashed{P}_{\Psi} )(-\slashed{q}+M_T)(-\slashed{P}_{\Psi} )(-\slashed{q}+ M_T)}{(q^2-M_T^2)^{3}}\begin{pmatrix}
 				t_R\\ \ccj^{-1}\bar{t}_R
 			\end{pmatrix} 
 		\end{align}
 		
 		Let's focus first in the numerator of the fraction above: 
 		\begin{align}
 			&(-\slashed{q}+M_\Psi)(-\slashed{P}_{\Psi} )(-\slashed{q}+M_\Psi)(-\slashed{P}_{\Psi} )(-\slashed{q}+ M_\Psi) = (\slashed{q}\slashed{P}_{\Psi}-M_\Psi\slashed{P}_{\Psi})(\slashed{q}\slashed{P}_{\Psi}-M_\Psi\slashed{P})(-\slashed{q}+ M_\Psi)\nonumber\\
 			& = (\slashed{q}\slashed{P}_{\Psi}\slashed{q}\slashed{P}_{\Psi} - M_\Psi\slashed{q}\slashed{P}_{\Psi}\slashed{P}_{\Psi} - M_\Psi\slashed{P}_{\Psi}\slashed{q}\slashed{P}_{\Psi} + M_\Psi^2\slashed{P}_{\Psi}\slashed{P}_{\Psi})(-\slashed{q}+ M_\Psi)\nonumber\\
 			& = -\slashed{q}\slashed{P}_{\Psi}\slashed{q}\slashed{P}_{\Psi}\slashed{q} + M_\Psi\slashed{q}\slashed{P}_{\Psi}\slashed{P}_{\Psi}\slashed{q} + M_\Psi\slashed{P}_{\Psi}\slashed{q}\slashed{P}_{\Psi}\slashed{q} - M_\Psi^2\slashed{P}_{\Psi}\slashed{P}_{\Psi}\slashed{q} + M_\Psi\slashed{q}\slashed{P}_{\Psi}\slashed{q}\slashed{P}_{\Psi} - M_\Psi^2\slashed{q}\slashed{P}_{\Psi}\slashed{P}_{\Psi} \nonumber\\& \hspace{9.7cm}- M_\Psi^2\slashed{P}_{\Psi}\slashed{q}\slashed{P}_{\Psi} + M_\Psi^3\slashed{P}_{\Psi}\slashed{P}_{\Psi}
 		\end{align}
 		
 		Discarding terms with even number of gamma matrices:
 		
 		\begin{align}
 			&=-\slashed{q}\slashed{P}_{\Psi}\slashed{q}\slashed{P}_{\Psi}\slashed{q}   - M_\Psi^2\slashed{P}_{\Psi}\slashed{P}_{\Psi}\slashed{q}- M_\Psi^2\slashed{q}\slashed{P}_{\Psi}\slashed{P}_{\Psi} - M_\Psi^2\slashed{P}_{\Psi}\slashed{q}\slashed{P}_{\Psi} \label{eq:l=2}
 		\end{align}
 		
 		Since all loop momenta contribution in the numerator comes from this terms and we know that loop integrals with odd powers of $q$ are zero due the symmetry, all terms in the $l=2$ vanishes.
 		
 		
 		
 		\item $l=3$
 		
 		The $l=3$ contribution ig given by:
 		
 		\begin{align}
 			\begin{pmatrix}
 				\bar{t}_R & t_R\ccj^{-1} 
 			\end{pmatrix}  \frac{ (-\slashed{q}+M_T)(-\slashed{P}_{\Psi}(-\slashed{q}+M_T)(-\slashed{P}_{\Psi} )(-\slashed{q}+M_T)(-\slashed{P}_{\Psi} )(-\slashed{q}+ M_T)}{(q^2-M_T^2)^{4}}\begin{pmatrix}
 				t_R\\ \ccj^{-1}\bar{t}_R
 			\end{pmatrix} 
 		\end{align}
 		
 		Let's focus first in the numerator of the fraction above: 
 		
 		\begin{align}
 			&(-\slashed{q}+M_\Psi)(-\slashed{P}_{\Psi} )(-\slashed{q}+M_\Psi)(-\slashed{P}_{\Psi} )(-\slashed{q}+M_\Psi)(-\slashed{P}_{\Psi} )(-\slashed{q}+ M_\Psi) = \nonumber\\
 			&(\slashed{q}\slashed{P}_{\Psi} - M_\Psi\slashed{P}_{\Psi})\nonumber\\&(-\slashed{q}\slashed{P}_{\Psi}\slashed{q}\slashed{P}_{\Psi}\slashed{q} + M_\Psi\slashed{q}\slashed{P}_{\Psi}\slashed{P}_{\Psi}\slashed{q} + M_\Psi\slashed{P}_{\Psi}\slashed{q}\slashed{P}_{\Psi}\slashed{q} - M_\Psi^2\slashed{P}_{\Psi}\slashed{P}_{\Psi}\slashed{q} + M_\Psi\slashed{q}\slashed{P}_{\Psi}\slashed{q}\slashed{P}_{\Psi} - M_\Psi^2\slashed{q}\slashed{P}_{\Psi}\slashed{P}_{\Psi} - M_\Psi^2\slashed{P}_{\Psi}\slashed{q}\slashed{P}_{\Psi} 
 			\nonumber\\ &\hspace{13.6cm}+ M_\Psi^3\slashed{P}_{\Psi}\slashed{P}_{\Psi})
 		\end{align}
 		
 		Again, since we know that only terms with an even powers of $q$ do not vanish:
 		
 		\begin{align}
 			&(-\slashed{q}+M_\Psi)(-\slashed{P}_{\Psi} )(-\slashed{q}+M_\Psi)(-\slashed{P}_{\Psi} )(-\slashed{q}+M_\Psi)(-\slashed{P}_{\Psi} )(-\slashed{q}+ M_\Psi)\nonumber\\
 			&=-\slashed{q}\slashed{P}_{\Psi}\slashed{q}\slashed{P}_{\Psi}\slashed{q}\slashed{P}_{\Psi}\slashed{q} - M_\Psi^2\slashed{q}\slashed{P}_{\Psi}\slashed{P}_{\Psi}\slashed{P}_{\Psi}\slashed{q}  - M_\Psi^2\slashed{q}\slashed{P}_{\Psi}\slashed{q}\slashed{P}_{\Psi}\slashed{P}_{\Psi} - M_\Psi^2\slashed{q}\slashed{P}_{\Psi}\slashed{P}_{\Psi}\slashed{q}\slashed{P}_{\Psi} 
 			- M_\Psi^2\slashed{P}_{\Psi}\slashed{q}\slashed{P}_{\Psi}\slashed{P}_{\Psi}\slashed{q} \nonumber\\ &\hspace{6.3cm}- M_\Psi^2\slashed{P}_{\Psi}\slashed{P}_{\Psi}\slashed{q}\slashed{P}_{\Psi}\slashed{q} - M_\Psi^2\slashed{P}_{\Psi}\slashed{q}\slashed{P}_{\Psi}\slashed{q}\slashed{P}_{\Psi} 
 			- M_\Psi^4\slashed{P}_{\Psi}\slashed{P}_{\Psi}\slashed{P}_{\Psi}
 		\end{align}
 		
 		Keeping only the terms with an even number of $q$:
 		
 		
 		\begin{align}
 			&(-\slashed{q}+M_\Psi)(-\slashed{P}_{\Psi} )(-\slashed{q}+M_\Psi)(-\slashed{P}_{\Psi} )(-\slashed{q}+M_\Psi)(-\slashed{P}_{\Psi} )(-\slashed{q}+ M_\Psi)\nonumber\\
 			&= -\slashed{q}\slashed{P}_{\Psi}\slashed{q}\slashed{P}_{\Psi}\slashed{q}\slashed{P}_{\Psi}\slashed{q}- M_\Psi^2\slashed{q}\slashed{P}_{\Psi}\slashed{P}_{\Psi}\slashed{P}_{\Psi}\slashed{q}  - M_\Psi^2\slashed{q}\slashed{P}_{\Psi}\slashed{q}\slashed{P}_{\Psi}\slashed{P}_{\Psi} - M_\Psi^2\slashed{q}\slashed{P}_{\Psi}\slashed{P}_{\Psi}\slashed{q}\slashed{P}_{\Psi} 
 			- M_\Psi^2\slashed{P}_{\Psi}\slashed{q}\slashed{P}_{\Psi}\slashed{P}_{\Psi}\slashed{q} \nonumber\\ &\hspace{6.3cm}- M_\Psi^2\slashed{P}_{\Psi}\slashed{P}_{\Psi}\slashed{q}\slashed{P}_{\Psi}\slashed{q} - M_\Psi^2\slashed{P}_{\Psi}\slashed{q}\slashed{P}_{\Psi}\slashed{q}\slashed{P}_{\Psi} 
 			- M_\Psi^4\slashed{P}_{\Psi}\slashed{P}_{\Psi}\slashed{P}_{\Psi}
 		\end{align}
 		%\begin{align}
 		%	&(-\slashed{q}+M_\Psi)(-\slashed{P}_{\Psi} )(-\slashed{q}+M_\Psi)(-\slashed{P}_{\Psi} )(-\slashed{q}+M_\Psi)(-\slashed{P}_{\Psi} )(-\slashed{q}+ M_\Psi)\nonumber\\
 		%	&= - M_\Psi^2\slashed{q}\slashed{P}_{\Psi}\slashed{q}P_{\Psi}^2  - M_\Psi^2\slashed{q}\slashed{P}_{\Psi}\slashed{q}P_{\Psi}^2 - M_\Psi^2P_{\Psi}^2\slashed{P}_{\Psi} q^2
 		%	- M_\Psi^2\slashed{P}_{\Psi}P_{\Psi}^2q^2 - M_\Psi^2P_{\Psi}^2\slashed{q}\slashed{P}_{\Psi}\slashed{q}\nonumber\\&\hspace{9cm} - M_\Psi^2\slashed{P}_{\Psi}\slashed{q}\slashed{P}_{\Psi}\slashed{q}\slashed{P}_{\Psi} 
 		%	- M_\Psi^4\slashed{P}_{\Psi}\slashed{P}_{\Psi}\slashed{P}_{\Psi}
 		%\end{align}
 		
 		Anticipating the substitution $q^\mu q^\nu \to \frac{g^{\mu\nu}}{4} q^2$ and $q^\mu q^\nu q^\rho q^\sigma \to  \frac{1}{24} (g^{\mu\nu}g^{\sigma\rho} + g^{\mu\rho} g^{\nu\sigma} + g^{\mu\sigma} g^{\nu\rho})q^4$:
 		
 		\begin{align}
 			&(-\slashed{q}+M_\Psi)(-\slashed{P}_{\Psi} )(-\slashed{q}+M_\Psi)(-\slashed{P}_{\Psi} )(-\slashed{q}+M_\Psi)(-\slashed{P}_{\Psi} )(-\slashed{q}+ M_\Psi)\nonumber\\
 			&= -\frac{1}{24}\gamma^\mu\gamma^\alpha\gamma^\nu\gamma^\beta\gamma^\rho\gamma^\lambda\gamma^\sigma(g^{\mu\nu}g^{\sigma\rho} + g^{\mu\rho} g^{\nu\sigma} + g^{\mu\sigma} g^{\nu\rho})P_{\Psi, \alpha} P_{\Psi, \beta} P_{\Psi, \lambda} q^4\nonumber\\&- \frac{1}{4}M_\Psi^2\gamma^\mu\gamma^\nu\gamma^\rho\gamma^\sigma\gamma_\mu P_{\Psi, \nu}P_{\Psi, \rho}P_{\Psi,\sigma}q^2  -\frac{1}{4} M_\Psi^2\gamma^\mu\gamma^\nu\gamma_\mu P_{\Psi,\nu}\slashed{P}_{\Psi}\slashed{P}_{\Psi}q^2 -\frac{1}{4} M_\Psi^2\gamma^\mu\gamma^\nu\gamma^\rho\gamma_\mu P_{\Psi,\nu}P_{\Psi,\rho}\slashed{P}_{\Psi} q^2
 			\nonumber\\ &- \frac{1}{4}M_\Psi^2\slashed{P}_{\Psi}\gamma^\mu\gamma^\nu\gamma^\rho\gamma_\mu P_{\Psi,\nu} P_{\Psi,\rho}q^2 -\frac{1}{4} M_\Psi^2\slashed{P}_{\Psi}\slashed{P}_{\Psi}\gamma^\mu\gamma^\nu\gamma_\mu P_{\Psi,\nu}q^2 -\frac{1}{4} M_\Psi^2\slashed{P}_{\Psi}\gamma^\mu\gamma^\nu\gamma_\mu P_{\Psi,\nu}\slashed{P}_{\Psi} q^2
 			- M_\Psi^4\slashed{P}_{\Psi}\slashed{P}_{\Psi}\slashed{P}_{\Psi}
 		\end{align}
 		
 		\begin{align}
 			&(-\slashed{q}+M_\Psi)(-\slashed{P}_{\Psi} )(-\slashed{q}+M_\Psi)(-\slashed{P}_{\Psi} )(-\slashed{q}+M_\Psi)(-\slashed{P}_{\Psi} )(-\slashed{q}+ M_\Psi)\nonumber\\
 			&= -\frac{1}{24}(\gamma^\mu\gamma^\alpha\gamma_\mu\gamma^\beta\gamma^\rho\gamma^\lambda\gamma_\rho + \gamma^\mu\gamma^\alpha\gamma^\nu\gamma^\beta\gamma_\mu\gamma^\lambda\gamma_\nu + \gamma^\mu\gamma^\alpha\gamma^\nu\gamma^\beta\gamma_\nu\gamma^\lambda\gamma_\mu)P_{\Psi, \alpha} P_{\Psi, \beta} P_{\Psi, \lambda} q^4\nonumber\\&- \frac{1}{4}M_\Psi^2\gamma^\mu\gamma^\nu\gamma^\rho\gamma^\sigma\gamma_\mu P_{\Psi, \nu}P_{\Psi, \rho}P_{\Psi,\sigma}q^2  -\frac{1}{4} M_\Psi^2\gamma^\mu\gamma^\nu\gamma_\mu P_{\Psi,\nu}\slashed{P}_{\Psi}\slashed{P}_{\Psi}q^2 -\frac{1}{4} M_\Psi^2\gamma^\mu\gamma^\nu\gamma^\rho\gamma_\mu P_{\Psi,\nu}P_{\Psi,\rho}\slashed{P}_{\Psi} q^2
 			\nonumber\\ &- \frac{1}{4}M_\Psi^2\slashed{P}_{\Psi}\gamma^\mu\gamma^\nu\gamma^\rho\gamma_\mu P_{\Psi,\nu} P_{\Psi,\rho}q^2 -\frac{1}{4} M_\Psi^2\slashed{P}_{\Psi}\slashed{P}_{\Psi}\gamma^\mu\gamma^\nu\gamma_\mu P_{\Psi,\nu}q^2 -\frac{1}{4} M_\Psi^2\slashed{P}_{\Psi}\gamma^\mu\gamma^\nu\gamma_\mu P_{\Psi,\nu}\slashed{P}_{\Psi} q^2
 			- M_\Psi^4\slashed{P}_{\Psi}\slashed{P}_{\Psi}\slashed{P}_{\Psi}
 		\end{align}
 		
 		%\begin{align}
 		%	&(-\slashed{q}+M_\Psi)(-\slashed{P}_{\Psi} )(-\slashed{q}+M_\Psi)(-\slashed{P}_{\Psi} )(-\slashed{q}+M_\Psi)(-\slashed{P}_{\Psi} )(-\slashed{q}+ M_\Psi)\nonumber\\
 		%	&= - 2M_\Psi^2\slashed{q}\slashed{P}_{\Psi}\slashed{q}P_{\Psi}^2  - M_\Psi^2P_{\Psi}^2\slashed{P}_{\Psi} q^2
 		%	- M_\Psi^2\slashed{P}_{\Psi}P_{\Psi}^2q^2 - M_\Psi^2P_{\Psi}^2\slashed{q}\slashed{P}_{\Psi}\slashed{q} - M_\Psi^2\slashed{P}_{\Psi}\slashed{q}\slashed{P}_{\Psi}\slashed{q}\slashed{P}_{\Psi} 
 		%	- M_\Psi^4\slashed{P}_{\Psi}\slashed{P}_{\Psi}\slashed{P}_{\Psi}
 		%\end{align}
 		
 		
 		%Anticipating the substitution $q^\mu q^\nu \to \frac{g^{\mu\nu}}{4} q^2$:
 		
 		%\begin{align}
 		%	&(-\slashed{q}+M_\Psi)(-\slashed{P}_{\Psi} )(-\slashed{q}+M_\Psi)(-\slashed{P}_{\Psi} )(-\slashed{q}+M_\Psi)(-\slashed{P}_{\Psi} )(-\slashed{q}+ M_\Psi)\nonumber\\
 		%	&= - 4M_\Psi^2q\cdot P_{\Psi}\slashed{q}P_{\Psi}^2 + 2M_\Psi^2\slashed{P}_{\Psi} P_{\Psi}^2q^2 - M_\Psi^2P_{\Psi}^2\slashed{P}_{\Psi} q^2
 		%	- M_\Psi^2\slashed{P}_{\Psi}P_{\Psi}^2q^2 
 		%	- 2M_\Psi^2P_{\Psi}^2 q\cdot P_{\Psi}\slashed{q} + M_\Psi^2P_{\Psi}^2\slashed{P}_{\Psi}q^2\nonumber\\
 		%	& - 2 M_\Psi^2P_{\Psi}\cdot q\slashed{P}_{\Psi}\slashed{q}\slashed{P}_{\Psi} 
 		%	- M_\Psi^2\slashed{P}_{\Psi}\slashed{q}\slashed{P}_{\Psi}\slashed{q}\slashed{P}_{\Psi} 
 		%	- M_\Psi^4\slashed{P}_{\Psi}\slashed{P}_{\Psi}\slashed{P}_{\Psi}
 		%\end{align}
 		
 		Using that:
 		
 		\begin{align}
 			&\gamma^\mu\gamma^\nu\gamma_\mu  = -2\gamma^\nu\\
 			&\gamma^\mu\gamma^\nu\gamma^\rho\gamma_\mu = 4 g^{\nu\rho} - \epsilon \gamma^{\nu\rho}\\
 			&\gamma^\mu\gamma^\nu\gamma^\rho\gamma^\sigma\gamma_\mu=-2\gamma^\sigma\gamma^\rho\gamma^\nu + \epsilon \gamma^\nu\gamma^\rho\gamma^\sigma
 		\end{align}
 		
 		we get:
 		
 		\begin{align}
 			&(-\slashed{q}+M_\Psi)(-\slashed{P}_{\Psi} )(-\slashed{q}+M_\Psi)(-\slashed{P}_{\Psi} )(-\slashed{q}+M_\Psi)(-\slashed{P}_{\Psi} )(-\slashed{q}+ M_\Psi)\nonumber\\
 			&= -\frac{1}{24}(4\gamma^\alpha\gamma^\beta\gamma^\lambda -8 g^{\alpha\lambda}\gamma^\beta +4 \gamma^\lambda\gamma^\beta\gamma^\alpha)P_{\Psi, \alpha} P_{\Psi, \beta} P_{\Psi, \lambda} q^4\nonumber\\& + \frac{1}{2}M_\Psi^2 \gamma^\sigma\gamma^\rho\gamma^\nu P_{\Psi, \nu} P_{\Psi,\rho} P_{\Psi,\sigma}q^2  +\frac{1}{2} M_\Psi^2 \slashed{P}_{\Psi}\slashed{P}_{\Psi}\slashed{P}_{\Psi}q^2 - M_\Psi^2P_{\Psi}^2\slashed{P}_{\Psi} q^2
 			\nonumber\\ &- M_\Psi^2\slashed{P}_{\Psi} P_{\Psi}^2 q^2 +\frac{1}{2} M_\Psi^2\slashed{P}_{\Psi}\slashed{P}_{\Psi} \slashed{P}_{\Psi}q^2 +\frac{1}{2} M_\Psi^2\slashed{P}_{\Psi} \slashed{P}_{\Psi}\slashed{P}_{\Psi} q^2
 			- M_\Psi^4\slashed{P}_{\Psi}\slashed{P}_{\Psi}\slashed{P}_{\Psi}
 		\end{align}
 		
 		
 		\begin{align}
 			&(-\slashed{q}+M_\Psi)(-\slashed{P}_{\Psi} )(-\slashed{q}+M_\Psi)(-\slashed{P}_{\Psi} )(-\slashed{q}+M_\Psi)(-\slashed{P}_{\Psi} )(-\slashed{q}+ M_\Psi)\nonumber\\
 			&= -\frac{1}{24}(4\gamma^\alpha\gamma^\beta\gamma^\lambda -8 g^{\alpha\lambda}\gamma^\beta +8 g^{\lambda\beta}\gamma^\alpha - 8g^{\lambda\alpha}\gamma^\beta + 8g^{\alpha\beta}\gamma^\lambda - 4\gamma^\alpha\gamma^\beta \gamma^\lambda)P_{\Psi, \alpha} P_{\Psi, \beta} P_{\Psi, \lambda} q^4\nonumber\\&  +\frac{1}{2}M_\Psi^2 (2 g^{\sigma\rho}\gamma^\nu - 2g^{\sigma\nu}\gamma^\rho + 2g^{\nu\rho} \gamma^{\sigma}- \gamma^\nu\gamma^\rho \gamma^\sigma)  P_{\Psi, \nu} P_{\Psi,\rho} P_{\Psi,\sigma}q^2  +\frac{1}{2} M_\Psi^2 \slashed{P}_{\Psi}\slashed{P}_{\Psi}\slashed{P}_{\Psi}q^2 - M_\Psi^2P_{\Psi}^2\slashed{P}_{\Psi} q^2
 			\nonumber\\ &- M_\Psi^2\slashed{P}_{\Psi} P_{\Psi}^2 q^2 +\frac{1}{2} M_\Psi^2\slashed{P}_{\Psi}\slashed{P}_{\Psi} \slashed{P}_{\Psi}q^2 +\frac{1}{2} M_\Psi^2\slashed{P}_{\Psi} \slashed{P}_{\Psi}\slashed{P}_{\Psi} q^2
 			- M_\Psi^4\slashed{P}_{\Psi}\slashed{P}_{\Psi}\slashed{P}_{\Psi}
 		\end{align}
 		
 		\begin{align}
 			&(-\slashed{q}+M_\Psi)(-\slashed{P}_{\Psi} )(-\slashed{q}+M_\Psi)(-\slashed{P}_{\Psi} )(-\slashed{q}+M_\Psi)(-\slashed{P}_{\Psi} )(-\slashed{q}+ M_\Psi)\nonumber\\
 			&= -\frac{1}{24}( -16 g^{\alpha\lambda}\gamma^\beta +8 g^{\lambda\beta}\gamma^\alpha +8g^{\alpha\beta}\gamma^{\lambda})P_{\Psi, \alpha} P_{\Psi, \beta} P_{\Psi, \lambda} q^4\nonumber\\&  +\frac{1}{2}M_\Psi^2 (2 g^{\sigma\rho}\gamma^\nu - 2g^{\sigma\nu}\gamma^\rho + 2g^{\nu\rho} \gamma^{\sigma}- \gamma^\nu\gamma^\rho \gamma^\sigma)  P_{\Psi, \nu} P_{\Psi,\rho} P_{\Psi,\sigma}q^2  +\frac{1}{2} M_\Psi^2 \slashed{P}_{\Psi}\slashed{P}_{\Psi}\slashed{P}_{\Psi}q^2 - M_\Psi^2P_{\Psi}^2\slashed{P}_{\Psi} q^2
 			\nonumber\\ &- M_\Psi^2\slashed{P}_{\Psi} P_{\Psi}^2 q^2 +\frac{1}{2} M_\Psi^2\slashed{P}_{\Psi}\slashed{P}_{\Psi} \slashed{P}_{\Psi}q^2 +\frac{1}{2} M_\Psi^2\slashed{P}_{\Psi} \slashed{P}_{\Psi}\slashed{P}_{\Psi} q^2
 			- M_\Psi^4\slashed{P}_{\Psi}\slashed{P}_{\Psi}\slashed{P}_{\Psi}
 		\end{align}
 		
 		\begin{align}
 			&(-\slashed{q}+M_\Psi)(-\slashed{P}_{\Psi} )(-\slashed{q}+M_\Psi)(-\slashed{P}_{\Psi} )(-\slashed{q}+M_\Psi)(-\slashed{P}_{\Psi} )(-\slashed{q}+ M_\Psi)\nonumber\\
 			&= \frac{2}{3}P_{\Psi}^\mu \slashed{P}_{\Psi} P_{\Psi, \mu} q^4 - \frac{1}{3}\slashed{P}_{\Psi}P^2_\Psi q^4 - \frac{1}{3} P_{\Psi}^2\slashed{P}_{\Psi} q^4\nonumber\\&  +M_\Psi^2  \slashed{P}_{\Psi} P^2_{\Psi}q^2  - M_\Psi^2 P_{\Psi}^\mu \slashed{P}_{\Psi}P_{\Psi, \mu}q^2 + M_\Psi^2 P^2_{\Psi}\slashed{P}_{\Psi} q^2 - \frac{1}{2} M_\Psi^2 \slashed{P}_{\Psi} \slashed{P}_{\Psi} \slashed{P}_{\Psi}q^2  +\frac{1}{2} M_\Psi^2 \slashed{P}_{\Psi}\slashed{P}_{\Psi}\slashed{P}_{\Psi}q^2 - M_\Psi^2P_{\Psi}^2\slashed{P}_{\Psi} q^2
 			\nonumber\\ &- M_\Psi^2\slashed{P}_{\Psi} P_{\Psi}^2 q^2 +\frac{1}{2} M_\Psi^2\slashed{P}_{\Psi}\slashed{P}_{\Psi} \slashed{P}_{\Psi}q^2 +\frac{1}{2} M_\Psi^2\slashed{P}_{\Psi} \slashed{P}_{\Psi}\slashed{P}_{\Psi} q^2
 			- M_\Psi^4\slashed{P}_{\Psi}\slashed{P}_{\Psi}\slashed{P}_{\Psi}
 		\end{align}
 		
 		\begin{align}
 			&(-\slashed{q}+M_\Psi)(-\slashed{P}_{\Psi} )(-\slashed{q}+M_\Psi)(-\slashed{P}_{\Psi} )(-\slashed{q}+M_\Psi)(-\slashed{P}_{\Psi} )(-\slashed{q}+ M_\Psi)\nonumber\\
 			&= \frac{2}{3}P_{\Psi}^\mu \slashed{P}_{\Psi} P_{\Psi, \mu} q^4 - \frac{1}{3}\slashed{P}_{\Psi}P^2_\Psi q^4 - \frac{1}{3} P_{\Psi}^2\slashed{P}_{\Psi} q^4  - M_\Psi^2 P_{\Psi}^\mu \slashed{P}_{\Psi}P_{\Psi, \mu}q^2 + M_\Psi^2\slashed{P}_{\Psi}\slashed{P}_{\Psi} \slashed{P}_{\Psi}q^2 
 			- M_\Psi^4\slashed{P}_{\Psi}\slashed{P}_{\Psi}\slashed{P}_{\Psi}
 		\end{align}
 		
 		Using that:
 		
 		\begin{align}
 			\gamma^\mu\gamma^\nu\gamma^\rho = \Gamma^{\mu\rho\nu} + g^{\nu\rho}\gamma^\mu + g^{\mu\nu}\gamma^\rho - g^{\mu\rho}\gamma^\nu 
 		\end{align}
 		
 		where $\Gamma^{\mu\nu\rho}$ is the antisymmetrized and normalized product of three Dirac matrices.
 		
 		we get:
 		
 		
 		\begin{align}
 			&(-\slashed{q}+M_\Psi)(-\slashed{P}_{\Psi} )(-\slashed{q}+M_\Psi)(-\slashed{P}_{\Psi} )(-\slashed{q}+M_\Psi)(-\slashed{P}_{\Psi} )(-\slashed{q}+ M_\Psi)\nonumber\\
 			&= \frac{2}{3}P_{\Psi}^\mu \slashed{P}_{\Psi} P_{\Psi, \mu} q^4 - \frac{1}{3}\slashed{P}_{\Psi}P^2_\Psi q^4 - \frac{1}{3} P_{\Psi}^2\slashed{P}_{\Psi} q^4  - M_\Psi^2 P_{\Psi}^\mu \slashed{P}_{\Psi}P_{\Psi, \mu}q^2 \nonumber\\
 			& +M_\Psi^2\slashed{P}_{\Psi}P^2{\Psi}q^2 + M_\Psi^2 P^2{\Psi}\slashed{P}_{\Psi} - M_\Psi^2 P_{\Psi}^\mu\slashed{P}_{\Psi}P_{\Psi, \mu}q^2 -M_\Psi^4\slashed{P}_{\Psi}P^2_{\Psi} - M_\Psi^4 P^2_{\Psi}\slashed{P}_{\Psi} + M_\Psi^4 P_{\Psi}^\mu\slashed{P}_{\Psi}P_{\Psi, \mu}
 			\nonumber\\&+ M_\Psi^2\Gamma^{\mu\nu\rho }P_{\Psi,\mu}P_{\Psi,\nu} P_{\Psi,\rho}q^2 
 			- M_\Psi^4\Gamma^{\mu\nu\rho }P_{\Psi,\mu}P_{\Psi,\nu} P_{\Psi,\rho}
 		\end{align}
 		
 		\begin{align}
 			&(-\slashed{q}+M_\Psi)(-\slashed{P}_{\Psi} )(-\slashed{q}+M_\Psi)(-\slashed{P}_{\Psi} )(-\slashed{q}+M_\Psi)(-\slashed{P}_{\Psi} )(-\slashed{q}+ M_\Psi)\nonumber\\
 			&= \frac{2}{3}P_{\Psi}^\mu \slashed{P}_{\Psi} P_{\Psi, \mu} q^4 - \frac{1}{3}\slashed{P}_{\Psi}P^2_\Psi q^4 - \frac{1}{3} P_{\Psi}^2\slashed{P}_{\Psi} q^4  -2 M_\Psi^2 P_{\Psi}^\mu \slashed{P}_{\Psi}P_{\Psi, \mu}q^2 \nonumber\\
 			& +M_\Psi^2\slashed{P}_{\Psi}P^2_{\Psi}q^2 + M_\Psi^2 P^2_{\Psi}\slashed{P}_{\Psi}  -M_\Psi^4\slashed{P}_{\Psi}P^2_{\Psi} - M_\Psi^4 P^2_{\Psi}\slashed{P}_{\Psi} + M_\Psi^4 P_{\Psi}^\mu\slashed{P}_{\Psi}P_{\Psi, \mu}
 			\nonumber\\&+ M_\Psi^2\Gamma^{\mu\nu\rho }P_{\Psi,\mu}P_{\Psi,\nu} P_{\Psi,\rho}q^2 
 			- M_\Psi^4\Gamma^{\mu\nu\rho }P_{\Psi,\mu}P_{\Psi,\nu} P_{\Psi,\rho}
 		\end{align}
 		
 		
 		\begin{align}
 			&(-\slashed{q}+M_\Psi)(-\slashed{P}_{\Psi} )(-\slashed{q}+M_\Psi)(-\slashed{P}_{\Psi} )(-\slashed{q}+M_\Psi)(-\slashed{P}_{\Psi} )(-\slashed{q}+ M_\Psi)\nonumber\\
 			&=\left(\frac{2}{3}q^4 - 2 M_\Psi^2 q^2 + M_\Psi^4\right)P_{\Psi}^\mu \slashed{P}_{\Psi} P_{\Psi, \mu} + \left(-\frac{1}{3}q^4 + M_\Psi^2 - M_\Psi^4\right)\slashed{P}_{\Psi}P^2_{\Psi}\nonumber\\
 			&\quad\left(-\frac{1}{3}q^4 + M_\Psi^2 - M_\Psi^4\right)P^2_{\Psi}\slashed{P}_{\Psi}  + \left(M^2q^2 - M^4\right)\Gamma^{\mu\nu\rho }P_{\Psi,\mu}P_{\Psi,\nu} P_{\Psi,\rho}
 		\end{align}
 		
 		Substituting this in the $l=3$ contribution:
 		
 		\begin{align}
 			&\begin{pmatrix}
 				\bar{t}_R & t_R\ccj^{-1} 
 			\end{pmatrix}  \frac{  \left[(-\slashed{q}+M_T)(-\slashed{P}_{\Psi} )\right]^3(-\slashed{q}+ M_T)}{(q^2-M_T^2)^{4}}\begin{pmatrix}
 				t_R\\ \ccj^{-1}\bar{t}_R
 			\end{pmatrix} = \nonumber\\
 			 & \frac{1}{(q^2-M_T^2)^4}\begin{pmatrix}
 			 	\bar{t}_R & t_R\ccj^{-1} 
 			 \end{pmatrix} \Bigg[\left(\frac{2}{3}q^4 - 2 M_T^2 q^2 + M_T^4\right)P_{\Psi}^\mu \slashed{P}_{\Psi} P_{\Psi, \mu} + \left(-\frac{1}{3}q^4 + M_T^2 - M_T^4\right)\slashed{P}_{\Psi}P^2_{\Psi}
 			 \nonumber\\
 			 &\hspace{3.5cm}+\left(-\frac{1}{3}q^4 + M_T^2 - M_T^4\right)P^2_{\Psi}\slashed{P}_{\Psi}  + \left(M_T^2q^2 - M_T^4\right)\Gamma^{\mu\nu\rho }P_{\Psi,\mu}P_{\Psi,\nu} P_{\Psi,\rho}\Bigg]\begin{pmatrix}
 			 	t_R\\ \ccj^{-1}\bar{t}_R
 			 \end{pmatrix}
 		\end{align}
 		
 		Doing the exact same steps as in the $l=0$ and $l=1$ case, we prove that the terms with the presence of the charge conjugation operators are equal to the terms without it but with a general minus sign difference and with the derivatives applied in the barred fields. Therefore:
 		
		\begin{align}
			&\begin{pmatrix}
				\bar{t}_R & t_R\ccj^{-1} 
			\end{pmatrix}  \frac{  \left[(-\slashed{q}+M_T)(-\slashed{P}_{\Psi} )\right]^3(-\slashed{q}+ M_T)}{(q^2-M_T^2)^{4}}\begin{pmatrix}
				t_R\\ \ccj^{-1}\bar{t}_R
			\end{pmatrix} = \nonumber\\
			&\frac{-i}{(q^2-M_T^2)^4}\Bigg[\left(\frac{2}{3}q^4 - 2 M_\Psi^2 q^2 + M_\Psi^4\right) \left(\bar{t}\, \gamma^\mu \, P_R \, D_\nu D_\mu D_\nu t\right) +\left(-\frac{1}{3}q^4 + M_\Psi^2 - M_\Psi^4\right)\left(\bar{t}\, \gamma^\mu \, P_R \, D_\mu D^2 t\right)\nonumber\\
			&+\left(-\frac{1}{3}q^4 + M_\Psi^2 - M_\Psi^4\right)\left(\bar{t}\, \gamma^\mu \, P_R \, D^2D_\mu t\right) +\left(M^2q^2 - M^4\right) \left(\bar{t} \, \Gamma^{\mu\nu\rho } \, P_R \, D_\mu D_\nu D_\rho t \right) - 
			\begin{pmatrix}
					D_\mu ... D_\nu \to \overleftarrow{D}_\mu ... \overleftarrow{D}_{\nu}
			\end{pmatrix}\Bigg]\label{eq:l=3}
		\end{align}
		
		Here $-\begin{pmatrix}
			D_\mu ... D_\nu \to \overleftarrow{D}_\mu ... \overleftarrow{D}_{\nu}
		\end{pmatrix}$ means that the rest of the terms are equal to the previous ones, with the derivatives acting on $\bar{t}$ instead of $t$ and an overall minus sign.

	\end{itemize}
	
	Now we substitute (\ref{eq:l=0}), (\ref{eq:l=1}), (\ref{eq:l=2}), (\ref{eq:l=3}) into Eq. (\ref{eq:n=1}):
	
	
	\begin{align}
		&\mathcal{L}^{\text{1-loop}}_{EFT,SF}[\phi_b,\psi_b] \supset\nonumber\\
		-&\frac{i}{2}\int \frac{d^dq}{(2\pi)^d}  \tr\Bigg\{\frac{\frac{1}{2} y_{Hs} h^2}{q^2 - M_s^2}
		- iy_{DM}^2\frac{\left(\frac{1}{2}q^2- M_T^2\right)}{(q^2 - M_s^2)(q^2 - M_T^2)^2}\left(\bar{t}\, \gamma^\mu\, P_R\, D_\mu t  \right)\nonumber\\
		&-\frac{i y_{DM}^2}{(q^2 - M_s^2)(q^2-M_T^2)^4}\Bigg[\left(\frac{2}{3}q^4 - 2 M_\Psi^2 q^2 + M_\Psi^4\right) \left(\bar{t}\, \gamma^\mu \, P_R \, D_\nu D_\mu D_\nu t\right) +\left(-\frac{1}{3}q^4 + M_T^2 - M_T^4\right)\left(\bar{t}\, \gamma^\mu \, P_R \, D_\mu D^2 t\right)\nonumber\\
		&+\left(-\frac{1}{3}q^4 + M_T^2 - M_T^4\right)\left(\bar{t}\, \gamma^\mu \, P_R \, D^2D_\mu t\right) +\left(M_T^2q^2 - M_T^4\right) \left(\bar{t} \, \Gamma^{\mu\nu\rho } \, P_R \, D_\mu D_\nu D_\rho t \right) - 
		\begin{pmatrix}
			D_\mu ... D_\nu \to \overleftarrow{D}_\mu ... \overleftarrow{D}_{\nu}
		\end{pmatrix}\Bigg] \Bigg\}\bigg\vert_{\text{hard}} 
	\end{align}
	
	We can compact the loop integrals that appears using the following notation:
	
	\begin{align}
		\int \frac{d^d q}{(2\pi)^d} \frac{q^{\mu_1} q^{\mu_2} ... q^{\mu_{2n_c}}}{(q^2 - M_i^2)^{n_i}(q^2 - M_j^2)^{n_j} (q^2)^{n_L}}\equiv \frac{i}{16\pi^2}g^{\mu_1 \mu_2...\mu_{2n_c}}\tilde{\lf}[q^{2n_c}]^{n_i, n_j, ..., n_L}_{ij...0}
	\end{align}
	
	\noindent where $g^{\mu_1 \mu_2...\mu_{2n_c}}$ is the completely symmetric combination of metric tensor with $2n_C$ indices. For $n_c = 0$ we define the shorthand notation $\tilde{\lf}[q^0]^{n_i, n_j, ..., n_L}_{ij...0} \equiv \tilde{\lf}^{n_i, n_j, ..., n_L}_{ij...0}$.
	
	Using this definition, the $n=1$ contribution can be written as:
	
	\begin{align}
		&\mathcal{L}^{\text{1-loop}}_{EFT,SF}[\phi_b,\psi_b] \supset \frac{1}{64\pi^2} y_{Hs} \tilde{\lf}^{1,0}_{s} h^2 -\frac{y_{DM}^2}{16\pi^2} \left(\frac{1}{4}\tilde{\lf}^{2,1,-1}_{T,s} - \frac{1}{2}M_T^2 \tilde{\lf}^{2,1,0}_{T,s}\right)\left(\bar{t}\, \gamma^\mu\, P_R\, D_\mu t - D_\mu\bar{t}\, \gamma^\mu\, P_R\,  t\right) \nonumber\\
		&-\frac{y_{DM}^2}{32\pi^2} \left(\frac{2}{3}\tilde{\lf}^{4,1,-2}_{T,s} - 2M_T^2 \tilde{\lf}^{4,1,-1}_{T,s} + M_T^4 \tilde{\lf}^{4,1,0}_{T,s}\right)\left(\bar{t}\, \gamma^\mu \, P_R \, D_\nu D_\mu D_\nu t - D_\nu D_\mu D_\nu\bar{t}\, \gamma^\mu \, P_R \, t\right)
		\nonumber\\&- \frac{y_{DM}^2}{32\pi^2}\left(-\frac{1}{3}\tilde{\lf}^{4,1,-2}_{T,s}+M_T^2 \tilde{\lf}^{4,1,-1}_{T,s} - M_T^4 \tilde{\lf}^{4,1,0}_{T,s}\right)\left(\bar{t}\, \gamma^\mu \, P_R \, D_\mu D^2 t - D_\mu D^2\bar{t}\, \gamma^\mu \, P_R \,  t\right)
		\nonumber\\&- \frac{y_{DM}^2}{32\pi^2}\left(-\frac{1}{3}\tilde{\lf}^{4,1,-2}_{T,s}+M_T^2 \tilde{\lf}^{4,1,-1}_{T,s} - M_T^4 \tilde{\lf}^{4,1,0}_{T,s}\right)\left(\bar{t}\, \gamma^\mu \, P_R \, D^2D_\mu t - D^2D_\mu \bar{t}\, \gamma^\mu \, P_R \,  t \right)
		\nonumber\\ &-\frac{y_{DM}^2}{32\pi^2}\left(M_T^2\tilde{\lf}^{4,1,-1}_{T,s} - M_T^4\tilde{\lf}^{4,1,0}_{T,s}\right)\left(\bar{t} \, \Gamma^{\mu\nu\rho } \, P_R \, D_\mu D_\nu D_\rho t -  D_\mu D_\nu D_\rho\bar{t} \, \Gamma^{\mu\nu\rho } \, P_R \, t \right) 
	\end{align}
	
	In order to simplify this expression, we can use this reduction formula to loop integrals:
	
	\begin{align}
		\tilde{\lf}[q^{2n_c}]^{n_i, n_j, ..., n_L}_{ij...0} = \frac{1}{M_i^2}\left(\tilde{\lf}[q^{2n_c}]^{n_i, n_j, ..., n_L-1}_{ij...0} - \tilde{\lf}[q^{2n_c}]^{n_i-1, n_j, ..., n_L}_{ij...0} \right)
	\end{align}
	
	Using this, we get:
	
	\begin{align}
		&\mathcal{L}^{\text{1-loop}}_{EFT,SF}[\phi_b,\psi_b] \supset \frac{1}{64\pi^2} y_{Hs} \tilde{\lf}^{1,0}_{s} h^2 +i\frac{y_{DM}^2}{16\pi^2} \left(  \frac{1}{2} \tilde{\lf}^{1,1,0}_{T,s}-\frac{1}{4}\tilde{\lf}^{2,1,-1}_{T,s}\right)\left(\bar{t}\, \gamma^\mu\, P_R\, D_\mu t - D_\mu\bar{t}\, \gamma^\mu\, P_R\,  t\right) \nonumber\\
		&+i\frac{y_{DM}^2}{16\pi^2} \left(\frac{1}{6}\tilde{\lf}^{4,1,-2}_{T,s} - \frac{1}{2} \tilde{\lf}^{2,1,0}_{T,s} \right)\left(\bar{t}\, \gamma^\mu \, P_R \, D_\nu D_\mu D_\nu t - D_\nu D_\mu D_\nu\bar{t}\, \gamma^\mu \, P_R \, t\right)
		\nonumber\\&+ i\frac{y_{DM}^2}{16\pi^2}\left(\frac{1}{6}\tilde{\lf}^{4,1,-2}_{T,s}- \frac{1}{2}\tilde{\lf}^{3,1,-1}_{T,s} +  \frac{1}{2}\tilde{\lf}^{2,1,0}_{T,,s}\right)\left(\bar{t}\, \gamma^\mu \, P_R \, D_\mu D^2 t - D_\mu D^2\bar{t}\, \gamma^\mu \, P_R \,  t\right)
		\nonumber\\&+ i\frac{y_{DM}^2}{16\pi^2}\left(\frac{1}{6}\tilde{\lf}^{4,1,-2}_{T,s}- \frac{1}{2}\tilde{\lf}^{3,1,-1}_{T,s} +  \frac{1}{2}\tilde{\lf}^{2,1,0}_{T,s}\right)\left(\bar{t}\, \gamma^\mu \, P_R \, D^2D_\mu t - D^2D_\mu \bar{t}\, \gamma^\mu \, P_R \,  t \right)
		\nonumber\\ &+i\frac{y_{DM}^2}{16\pi^2}\left(-\frac{1}{2}\tilde{\lf}^{3,1,-1}_{T,s} + \frac{1}{2}\tilde{\lf}^{2,1,0}_{T,s}\right)\left(\bar{t} \, \Gamma^{\mu\nu\rho } \, P_R \, D_\mu D_\nu D_\rho t -  D_\mu D_\nu D_\rho\bar{t} \, \Gamma^{\mu\nu\rho } \, P_R \, t \right) 
	\end{align}
	
	
	
	Now we can proceed to the $n=2$ calculations
	
\subsubsection*{ $n=2$ contribution}
	
	Discarding the total derivative terms and the covariant derivatives acting on the identity on the right:
	
	
	\begin{align}
		&\mathcal{L}^{\text{1-loop}}_{EFT,SF}[\phi_b,\psi_b] \supset\nonumber\\
		& -\frac{i}{4}\int \frac{d^dq}{(2\pi)^d}   \tr\Bigg[\frac{ - P_s^2 +\frac{1}{2} y_{Hs} h^2}{q^2 - M_s^2}
		- \frac{y_{DM}^2}{q^2 - M_s^2} \sum^{+\infty}_{l=3} \begin{pmatrix}
			\bar{t}_R & t_R\ccj^{-1} 
		\end{pmatrix}  \frac{ \left[(-\slashed{q}+M_T)(-\slashed{P}_{\Psi} )\right]^l(-\slashed{q}+ M_T)}{(q^2-M_T^2)^{l+1}}\begin{pmatrix}
			t_R\\ \ccj^{-1}\bar{t}_R
		\end{pmatrix}\Bigg]
		\nonumber\\&
		\hspace{2.5cm} \Bigg[\frac{\frac{1}{2} y_{Hs} h^2}{q^2 - M_s^2}
		- \frac{y_{DM}^2}{q^2 - M_s^2} \sum^{+\infty}_{l=3} \begin{pmatrix}
			\bar{t}_R & t_R\ccj^{-1} 
		\end{pmatrix}  \frac{ \left[(-\slashed{q}+M_T)(-\slashed{P}_{\Psi} )\right]^l(-\slashed{q}+ M_T)}{(q^2-M_T^2)^{l+1}}\begin{pmatrix}
			t_R\\ \ccj^{-1}\bar{t}_R
		\end{pmatrix}\Bigg]\bigg\vert_{\text{hard}}
	\end{align}
	
	Notice that we could have a term proportional to $P_s^2 h^2$, however it vanishes using integration by parts.
	
	Since the operators in the $l$ summation starts with dimension $4$, the contributions up to dimension six will be:
	
	\begin{align}
		&\mathcal{L}^{\text{1-loop}}_{EFT,SF}[\phi_b,\psi_b] \supset\nonumber\\
		& -\frac{i}{4}\int \frac{d^dq}{(2\pi)^d}   \tr\Bigg[\frac{\frac{1}{4} y_{Hs}^2 h^4}{(q^2 - M_\Phi^2)^2}
		- \frac{y_{DM}^2y_{Hs} h^2}{(q^2 - M_\Phi^2)^2}  \begin{pmatrix}
			\bar{t}_R & t_R\ccj^{-1} 
		\end{pmatrix}  \frac{ (-\slashed{q}+M_T)(-\slashed{P}_{\Psi} )(-\slashed{q}+ M_T)}{(q^2-M_T^2)^{2}}\begin{pmatrix}
			t_R\\ \ccj^{-1}\bar{t}_R
		\end{pmatrix}\Bigg]_{\text{hard}}
	\end{align}
	
	Using the result (\ref{eq:l=1}):
	
	\begin{align}
		\mathcal{L}^{\text{1-loop}}_{EFT,SF}[\phi_b,\psi_b] &\supset -\frac{i}{4}\int \frac{d^dq}{(2\pi)^d}   \tr\Bigg[\frac{\frac{1}{4} y_{Hs}^2 h^4}{(q^2 - M_\Phi^2)^2}
		- i\frac{y_{DM}^2y_{Hs} h^2}{(q^2 - M_\Phi^2)^2}  \frac{\left(\frac{1}{2}q^2- M_T^2\right)}{(q^2 - M_T^2)^2}\left[\bar{t}\, \gamma^\mu\, P_R\, D_\mu t - D_\mu \bar{t} \,\gamma^\mu\, P_R \, t \right]\Bigg]_{\text{hard}}\nonumber\\
		&\supset \frac{y_{Hs}^2}{256\pi^2} \tilde{\lf}^{2,0}_{s} h^4-i\frac{y_{DM}^2 y_{Hs}}{16\pi^2}\left(\frac{1}{8}\tilde{\lf}^{2,2,-1}_{T,s} -\frac{1}{4} M_T^2\tilde{\lf}^{2,2,0}_{T,s} \right)h^2\left(\bar{t}\, \gamma^\mu\, P_R\, D_\mu t - D_\mu \bar{t} \,\gamma^\mu\, P_R \, t \right) 
	\end{align}
	
	Using the reduction formula:
	
	
	\begin{align}
		&\mathcal{L}^{\text{1-loop}}_{EFT,SF}[\phi_b,\psi_b] \supset \frac{y_{Hs}^2}{256\pi^2} \tilde{\lf}^{2,0}_{s} h^4-i\frac{y_{DM}^2 y_{Hs}}{16\pi^2}\left(\frac{1}{8}\tilde{\lf}^{2,2,-1}_{T,s} -\frac{1}{4} \tilde{\lf}^{2,1,0}_{T,s} \right) h^2\left(\bar{t}\, \gamma^\mu\, P_R\, D_\mu t - D_\mu \bar{t} \,\gamma^\mu\, P_R \, t \right) 
	\end{align}
	
	\subsubsection*{$n=3$ contribution}
	
	Discarding the derivatives acting on the identity on the right:
	
	\begin{align}
		&\mathcal{L}^{\text{1-loop}}_{EFT,SF}[\phi_b,\psi_b] \supset\nonumber\\
		& -\frac{i}{6}\int \frac{d^dq}{(2\pi)^d}   \tr
		\Bigg[\frac{2q\cdot P_s - P_s^2 +\frac{1}{2} y_{Hs} h^2}{q^2 - M_s^2}
		- \frac{y_{DM}^2}{q^2 - M_s^2} \sum^{+\infty}_{l=3} \begin{pmatrix}
			\bar{t}_R & t_R\ccj^{-1} 
		\end{pmatrix}  \frac{ \left[(-\slashed{q}+M_T)(-\slashed{P}_{\Psi} )\right]^l(-\slashed{q}+ M_T)}{(q^2-M_T^2)^{l+1}}\begin{pmatrix}
			t_R\\ \ccj^{-1}\bar{t}_R
		\end{pmatrix}\Bigg]
		\nonumber\\
		&\quad\Bigg[\frac{2q\cdot P_s - P_s^2 +\frac{1}{2} y_{Hs} h^2}{q^2 - M_s^2}
		- \frac{y_{DM}^2}{q^2 - M_s^2} \sum^{+\infty}_{l=3} \begin{pmatrix}
			\bar{t}_R & t_R\ccj^{-1} 
		\end{pmatrix}  \frac{ \left[(-\slashed{q}+M_T)(-\slashed{P}_{\Psi} )\right]^l(-\slashed{q}+ M_T)}{(q^2-M_T^2)^{l+1}}\begin{pmatrix}
			t_R\\ \ccj^{-1}\bar{t}_R
		\end{pmatrix}\Bigg]
		\nonumber\\
		&\hspace{2.5cm} \Bigg[\frac{\frac{1}{2} y_{Hs} h^2}{q^2 - M_s^2}
		- \frac{y_{DM}^2}{q^2 - M_s^2} \sum^{+\infty}_{l=3} \begin{pmatrix}
			\bar{t}_R & t_R\ccj^{-1} 
		\end{pmatrix}  \frac{ \left[(-\slashed{q}+M_T)(-\slashed{P}_{\Psi} )\right]^l(-\slashed{q}+ M_T)}{(q^2-M_T^2)^{l+1}}\begin{pmatrix}
			t_R\\ \ccj^{-1}\bar{t}_R
		\end{pmatrix}\Bigg]\bigg\vert_{\text{hard}}
	\end{align}
	
	Here the $l$ summation does not contributes anymore, every operator including any term of it will have dimension greater than 6. Therefore:
	
	
	\begin{align}
		\mathcal{L}^{\text{1-loop}}_{EFT,SF}[\phi_b,\psi_b] &\supset -\frac{i}{6}\int \frac{d^dq}{(2\pi)^d}   \tr
		\Bigg[\frac{2q\cdot P_s - P_s^2 +\frac{1}{2} y_{Hs} h^2}{q^2 - M_s^2}
		\Bigg]
		\Bigg[\frac{2q\cdot P_s - P_s^2 +\frac{1}{2} y_{Hs} h^2}{q^2 - M_s^2}
		\Bigg]
		\Bigg[\frac{\frac{1}{2} y_{Hs} h^2}{q^2 - M_s^2} \Bigg]\bigg\vert_{\text{hard}}\nonumber\\
		&\supset -\frac{1}{384\pi^2} y_{Hs}^2 \tilde{\lf}^{3,0}_{s}\, \left( P^2_sh^4+h^2 P^2_s h^2\right)+\frac{1}{768\pi^2} y_{Hs}^3 \tilde{\lf}^{3,0}_{s}\, h^6\nonumber\\
		&\supset \frac{1}{384\pi^2} y_{Hs}^2 \tilde{\lf}^{3,0}_{s}\, (12 h^2 \partial_\mu h\partial^\mu h +4 h^3\square h +  2h^3  \square h + 2h^2 \partial_\mu h \partial^\mu h)- \frac{1}{768\pi^2} y_{Hs}^3 \tilde{\lf}^{3,0}_{s}\, h^6\nonumber\\
		&\supset \frac{7}{192\pi^2} y_{Hs}^2 \tilde{\lf}^{3,0}_{s}\, h^2 \partial_\mu h\partial^\mu h + \frac{1}{64\pi^2} y_{Hs}^2 \tilde{\lf}^{3,0}_{s}\, h^3\square h- \frac{1}{768\pi^2} y_{Hs}^3 \tilde{\lf}^{3,0}_{s}\, h^6
	\end{align}
	
	
	\noindent where in the second to third line we used that $P_s = i\partial$.
	
	
	\subsubsection*{$n=4$ contribution}
	
	Similarly to the $n=3$ contribution, the $l$ summation does not contributes anymore. Therefore, the only operator of dimension lower than 7 will be:
	
	\begin{align}
		&\mathcal{L}^{\text{1-loop}}_{EFT,SF}[\phi_b,\psi_b] \nonumber\\
		&\supset -\frac{i}{8}\int \frac{d^dq}{(2\pi)^d}   \tr
		\Bigg[\frac{2q\cdot P_s - P_s^2 +\frac{1}{2} y_{Hs} h^2}{q^2 - M_s^2}
		\Bigg]
			\Bigg[\frac{2q\cdot P_s - P_s^2 +\frac{1}{2} y_{Hs} h^2}{q^2 - M_s^2}
		\Bigg]
		\Bigg[\frac{2q\cdot P_s - P_s^2 +\frac{1}{2} y_{Hs} h^2}{q^2 - M_s^2}
		\Bigg]
		\Bigg[\frac{\frac{1}{2} y_{Hs} h^2}{q^2 - M_s^2} \Bigg]\bigg\vert_{\text{hard}}\nonumber\\
	\end{align}
	
	\begin{align}
		&\mathcal{L}^{\text{1-loop}}_{EFT,SF}[\phi_b,\psi_b] \supset -\frac{i}{8}\int \frac{d^dq}{(2\pi)^d}   \tr
		\Bigg[\frac{ y_{Hs}^2 h^2 (q\cdot P_s)(q\cdot P_s)h^2 + y_{Hs}^2 (q\cdot P_s)h^2 (q\cdot P_s)h^2 + y_{Hs}^2(q\cdot P_s)(q\cdot P_s) h^2 h^2}{(q^2 - M_s^2)^4}\Bigg]
	\end{align}
	
	Recall that in the loop integration we have the substitution $q^\mu q^\nu \to \frac{g^{\mu\nu}}{4}q^2$. As a consequence:
	
	\begin{align}
		\mathcal{L}^{\text{1-loop}}_{EFT,SF}[\phi_b,\psi_b] &\supset -\frac{i}{32} y_{Hs}^2\left( h^2  P_s^2 h^2 + P_s h^2 P_s h^2 + P_s h^2\right)\int \frac{d^dq}{(2\pi)^d}   
		\frac{  q^2}{(q^2 - M_s^2)^4}\nonumber\\
		&\supset \frac{1}{512\pi^2} y_{Hs}^2\tilde{\lf}^{4,-1}_{s}\left( 2h^3 P_s^2 h + 2h^2 P_{s,\mu}hP_s^\mu h + 6 h^2 P_{s,\mu}hP_s^\mu h + 2 h^3 P_s^2 h + 12 h^2 P_{s,\mu}hP_s^\mu h + 4h^3 P_s^2h\right)\nonumber\\
		&\supset \frac{1}{512\pi^2} y_{Hs}^2\tilde{\lf}^{4,-1}_{s}\left( 8h^3 P_s^2 h + 20h^2 P_{s,\mu}hP_s^\mu h \right)\nonumber\\
		&\supset- \frac{1}{64\pi^2} y_{Hs}^2 \tilde{\lf}^{4,-1}_{s}h^3  \square h - \frac{5}{128\pi^2} y_{Hs}^2 \tilde{\lf}^{4,-1}_{s} h^2 \partial_\mu h \partial^\mu h
	\end{align}
	
	
	\subsubsection*{All one-loop contribution up to dimension six for scalar-fermion path integral}
	

	
	\begin{align}
		&\mathcal{L}^{\text{1-loop}}_{EFT}[\phi_b,\psi_b] \supset \frac{1}{64\pi^2} y_{Hs} \tilde{\lf}^{1,0}_{s} h^2 +i\frac{y_{DM}^2}{16\pi^2} \left(  \frac{1}{2} \tilde{\lf}^{1,1,0}_{T,s}-\frac{1}{4}\tilde{\lf}^{2,1,-1}_{T,s}\right)\left(\bar{t}\, \gamma^\mu\, P_R\, D_\mu t - D_\mu\bar{t}\, \gamma^\mu\, P_R\,  t\right) \nonumber\\
		&+i\frac{y_{DM}^2}{16\pi^2} \left(\frac{1}{6}\tilde{\lf}^{4,1,-2}_{T,s} - \frac{1}{2} \tilde{\lf}^{2,1,0}_{T,s} \right)\left(\bar{t}\, \gamma^\mu \, P_R \, D_\nu D_\mu D_\nu t - D_\nu D_\mu D_\nu\bar{t}\, \gamma^\mu \, P_R \, t\right)
		\nonumber\\&+ i\frac{y_{DM}^2}{16\pi^2}\left(\frac{1}{6}\tilde{\lf}^{4,1,-2}_{T,s}- \frac{1}{2}\tilde{\lf}^{3,1,-1}_{T,s} +  \frac{1}{2}\tilde{\lf}^{2,1,0}_{T,,s}\right)\left(\bar{t}\, \gamma^\mu \, P_R \, D_\mu D^2 t - D_\mu D^2\bar{t}\, \gamma^\mu \, P_R \,  t\right)
		\nonumber\\&+ i\frac{y_{DM}^2}{16\pi^2}\left(\frac{1}{6}\tilde{\lf}^{4,1,-2}_{T,s}- \frac{1}{2}\tilde{\lf}^{3,1,-1}_{T,s} +  \frac{1}{2}\tilde{\lf}^{2,1,0}_{T,s}\right)\left(\bar{t}\, \gamma^\mu \, P_R \, D^2D_\mu t - D^2D_\mu \bar{t}\, \gamma^\mu \, P_R \,  t \right)
		\nonumber\\ &+i\frac{y_{DM}^2}{16\pi^2}\left(-\frac{1}{2}\tilde{\lf}^{3,1,-1}_{T,s} + \frac{1}{2}\tilde{\lf}^{2,1,0}_{T,s}\right)\left(\bar{t} \, \Gamma^{\mu\nu\rho } \, P_R \, D_\mu D_\nu D_\rho t -  D_\mu D_\nu D_\rho\bar{t} \, \Gamma^{\mu\nu\rho } \, P_R \, t \right) \nonumber\\
		&+\frac{y_{Hs}^2}{256\pi^2} \tilde{\lf}^{2,0}_{s} h^4+i\frac{y_{DM}^2 y_{Hs}}{16\pi^2}\left(\frac{1}{4} \tilde{\lf}^{2,1,0}_{T,s} - \frac{1}{8}\tilde{\lf}^{2,2,-1}_{T,s} \right) h^2\left(\bar{t}\, \gamma^\mu\, P_R\, D_\mu t - D_\mu \bar{t} \,\gamma^\mu\, P_R \, t \right)\nonumber\\
		&+\frac{1}{192\pi^2} y_{Hs}^2 \left(7\tilde{\lf}^{3,0}_{s} - 3\tilde{\lf}^{4,-1}_{s}\right)\, h^3\square h + \frac{1}{128\pi^2} y_{Hs}^2 \left(2\tilde{\lf}^{3,0}_{s} - 5\tilde{\lf}^{4,-1}_{s}\right)\, h^2 \partial_\mu h\partial^\mu h- \frac{1}{768\pi^2} y_{Hs}^3 \tilde{\lf}^{3,0}_{s}\, h^6
	\end{align}
	
	\subsubsection*{All one-loop contribution up to dimension six for fermionic path integral}
	
	The fermionic path integral contribution is given by (Eq. (\ref{eq:1-loopF1})):
	
	\begin{align}
		\mathcal{L}^{\text{1-loop}}_{EFT,F}[\phi_b,\psi_b] &= \frac{i}{2}\sum_{n=0}^{\infty}\frac{1}{n} \int \frac{d^d q}{(2\pi)^d}   \tr\left[ \frac{ \slashed{P} + \left( \mathbf{X}_{\Psi\Psi} - \mathbf{X}_{\Psi\psi} \, \Delta_\psi^{-1} \, \tilde{\mathbf{X}}_{\psi\Psi} \right) \big|_{P \to P - q} }{ \slashed{q} + M_\Psi } \right]^n_{\text{hard}} 
	\end{align}
	
	
	We know that:
	
	\begin{align}
		\mathbf{X}_{\Psi\Psi} = \mathbf{X}_{\Psi\psi} = 0_{3\times 3}
	\end{align}
	
	Therefore, this expression simplifies to:
	
	\begin{align}
		\mathcal{L}^{\text{1-loop}}_{EFT,F}[\phi_b,\psi_b] &= \frac{i}{2}\sum_{n=0}^{\infty}\frac{1}{n} \int \frac{d^d q}{(2\pi)^d}   \tr\left[ \frac{ \slashed{P}  }{ \slashed{q} + M_\Psi } \right]^n_{\text{hard}} = \frac{i}{2}\sum_{n=0}^{\infty}\frac{1}{n} \int \frac{d^d q}{(2\pi)^d}   \tr\left[ \frac{ \slashed{P} (-\slashed{q} +M_T)  }{ q^2 + M_T^2 } \right]^n
	\end{align}
	
	Notice that the Gauge group behind this covariant derivatives is not abelian and, as a consequence, the commutator $[P^\mu,P^\nu]$ are related to a Gauge strength tensor. Therefore, some terms in this expansion involving this commutators will generate pure Gauge contributions for the one-loop EFT Lagrangian.
	
	The terms $n=0$ and $n=1$ do not contributes, since the first generates a infinite constant and the second is an open covariant derivative acting on the identity on the left.
	
	
	
	\begin{itemize}
		\item $n=2$:
		
		\begin{align}
			\mathcal{L}^{\text{1-loop}}_{EFT,F}[\phi_b,\psi_b] &=\frac{i}{4} \int \frac{d^d q}{(2\pi)^d}   \tr\left[ \frac{ \slashed{P} (-\slashed{q} +M_T)  }{ q^2 + M_T^2 } \right]\left[ \frac{ \slashed{P} (-\slashed{q} +M_T)  }{ q^2 + M_T^2 } \right]
		\end{align}
		
		Since loop integrals with odd powers of $q$ vanishes:
		
		\begin{align}
		\mathcal{L}^{\text{1-loop}}_{EFT,F}[\phi_b,\psi_b] &=\frac{i}{4} \int \frac{d^d q}{(2\pi)^d}   \tr\left[ \frac{ \slashed{P}\slashed{q}\slashed{P}\slashed{q} +M_T^2\slashed{P}\slashed{P})  }{ (q^2 + M_T^2)^2 } \right]
		\end{align}
		
		Using the substitution $q^\mu q^\nu \to \frac{g^{\mu\nu}}{4}q^2$ in the loop integral:
		
		\begin{align}
			\mathcal{L}^{\text{1-loop}}_{EFT,F}[\phi_b,\psi_b] &=\frac{i}{4} \int \frac{d^d q}{(2\pi)^d}   \tr\left[\frac{ \frac{1}{4} \gamma^\mu\gamma^\rho \gamma^\nu\gamma_\rho P_\mu P_{\nu}q^2 +M_T^2\slashed{P}\slashed{P})  }{ (q^2 + M_T^2)^2 } \right]\nonumber\\
			&=\frac{i}{4} \int \frac{d^d q}{(2\pi)^d}   \tr\left[ \frac{ -\frac{1}{2}\gamma^\mu \gamma^\nu P_\mu P_{\nu}q^2 +M_T^2\slashed{P}\slashed{P})  }{ (q^2 + M_T^2)^2 } \right]\nonumber\\
		\end{align}
		
		\item $n=4$
		
		\begin{align}
			\mathcal{L}^{\text{1-loop}}_{EFT,F}[\phi_b,\psi_b] &=\frac{i}{4} \int \frac{d^d q}{(2\pi)^d}   \tr\left[ \frac{ \slashed{P} (-\slashed{q} +M_T)  }{ q^2 + M_T^2 } \right]\left[ \frac{ \slashed{P} (-\slashed{q} +M_T)  }{ q^2 + M_T^2 } \right]\left[ \frac{ \slashed{P} (-\slashed{q} +M_T)  }{ q^2 + M_T^2 } \right]\left[ \frac{ \slashed{P} (-\slashed{q} +M_T)  }{ q^2 + M_T^2 } \right]
		\end{align}
		
		Lets focus in the numerator first:
		
		\begin{align}
			\left[\slashed{P} (-\slashed{q} +M_T)\right]^4 &= \slashed{P} (-\slashed{q} +M_T)\slashed{P} (-\slashed{q} +M_T)\slashed{P} (-\slashed{q} +M_T)\slashed{P} (-\slashed{q} +M_T)\nonumber\\
			&=\slashed{P} (-\slashed{q} +M_T)\slashed{P} (-\slashed{q} +M_T)(\slashed{P}\slashed{q}\slashed{P}\slashed{q} - M_T \slashed{P}\slashed{q}\slashed{P} - M_T \slashed{P}\slashed{P}\slashed{q} + M_T^2\slashed{P}\slashed{P})\nonumber\\
			& = \slashed{P} (-\slashed{q} +M_T) (-\slashed{P}\slashed{q}\slashed{P}\slashed{q}\slashed{P}\slashed{q} + M_T \slashed{P}\slashed{q}\slashed{P}\slashed{q}\slashed{P} + M_T \slashed{P}\slashed{q}\slashed{P}\slashed{P}\slashed{q} - M_T^2\slashed{P}\slashed{q}\slashed{P}\slashed{P} \nonumber\\
			&+M_T\slashed{P}\slashed{P}\slashed{q}\slashed{P}\slashed{q} - M_T^2 \slashed{P}\slashed{P}\slashed{q}\slashed{P} - M_T^2 \slashed{P}\slashed{P}\slashed{P}\slashed{q} + M_T^3\slashed{P}\slashed{P}\slashed{P} )\nonumber\\
			&= \slashed{P}\slashed{q}\slashed{P}\slashed{q}\slashed{P}\slashed{q}\slashed{P}\slashed{q} - M_T \slashed{P}\slashed{q}\slashed{P}\slashed{q}\slashed{P}\slashed{q}\slashed{P} - M_T \slashed{P}\slashed{q}\slashed{P}\slashed{q}\slashed{P}\slashed{P}\slashed{q} + M_T^2\slashed{P}\slashed{q}\slashed{P}\slashed{q}\slashed{P}\slashed{P} \nonumber\\
			&-M_T\slashed{P}\slashed{q}\slashed{P}\slashed{P}\slashed{q}\slashed{P}\slashed{q} + M_T^2 \slashed{P}\slashed{q}\slashed{P}\slashed{P}\slashed{q}\slashed{P} + M_T^2\slashed{P}\slashed{q} \slashed{P}\slashed{P}\slashed{P}\slashed{q} - M_T^3\slashed{P}\slashed{q}\slashed{P}\slashed{P}\slashed{P}\nonumber\\
			& -M_T\slashed{P}\slashed{P}\slashed{q}\slashed{P}\slashed{q}\slashed{P}\slashed{q} + M_T^2 \slashed{P}\slashed{P}\slashed{q}\slashed{P}\slashed{q}\slashed{P} + M_T^2\slashed{P} \slashed{P}\slashed{q}\slashed{P}\slashed{P}\slashed{q} - M_T^3\slashed{P}\slashed{P}\slashed{q}\slashed{P}\slashed{P} \nonumber\\
			&+M_T^2\slashed{P}\slashed{P}\slashed{P}\slashed{q}\slashed{P}\slashed{q} - M_T^3 \slashed{P}\slashed{P}\slashed{P}\slashed{q}\slashed{P} - M_T^3 \slashed{P}\slashed{P}\slashed{P}\slashed{P}\slashed{q} + M_T^4\slashed{P}\slashed{P}\slashed{P}\slashed{P}
		\end{align}
		
		Since loop integrals with odd powers of $q$ in the numerator vanishes, we can discard them. Therefore:
		
		\begin{align}
			\left[\slashed{P} (-\slashed{q} +M_T)\right]^4 &\supset \slashed{P}\slashed{q}\slashed{P}\slashed{q}\slashed{P}\slashed{q}\slashed{P}\slashed{q}  + M_T^2\slashed{P}\slashed{q}\slashed{P}\slashed{q}\slashed{P}\slashed{P}  + M_T^2 \slashed{P}\slashed{q}\slashed{P}\slashed{P}\slashed{q}\slashed{P} + M_T^2\slashed{P}\slashed{q} \slashed{P}\slashed{P}\slashed{P}\slashed{q} \nonumber\\
			&  + M_T^2 \slashed{P}\slashed{P}\slashed{q}\slashed{P}\slashed{q}\slashed{P} + M_T^2\slashed{P} \slashed{P}\slashed{q}\slashed{P}\slashed{P}\slashed{q}  +M_T^2\slashed{P}\slashed{P}\slashed{P}\slashed{q}\slashed{P}\slashed{q}  + M_T^4\slashed{P}\slashed{P}\slashed{P}\slashed{P}
		\end{align}
		
		Using the substitution $q^\mu q^\nu \to \frac{g^{\mu\nu}}{4} q^2$ and $q^\mu q^\nu q^\rho q^\sigma \to  \frac{1}{24} (g^{\mu\nu}g^{\sigma\rho} + g^{\mu\rho} g^{\nu\sigma} + g^{\mu\sigma} g^{\nu\rho})q^4$:
		
		
		\begin{align}
				&\left[\slashed{P} (-\slashed{q} +M_T)\right]^4 \supset \bigg[\gamma^\alpha \gamma_\mu\gamma^\beta\gamma_\nu\gamma^\lambda\gamma_\rho\gamma^\theta\gamma_\sigma\frac{1}{24} (g^{\mu\nu}g^{\sigma\rho} + g^{\mu\rho} g^{\nu\sigma} + g^{\mu\sigma} g^{\nu\rho})q^4 \nonumber\\
				&\frac{1}{4}M_T^2 \gamma^\alpha \gamma^\mu\gamma^\beta\gamma_\mu \gamma^\lambda\gamma^\theta q^2 + \frac{1}{4}M_T^2 \gamma^\alpha\gamma^\mu \gamma^\beta\gamma^\lambda\gamma_\mu \gamma^\theta q^2
				+\frac{1}{4}M_T^2 \gamma^\alpha \gamma^\mu \gamma^\beta \gamma^\lambda \gamma^\theta \gamma_\mu q^2\nonumber\\
				&+\frac{1}{4}M_T^2 \gamma^\alpha\gamma^\beta\gamma^\mu\gamma^\lambda\gamma_\mu\gamma^\theta q^2
				+\frac{1}{4}M_T^2 \gamma^\alpha\gamma^\beta\gamma^\mu\gamma^\lambda\gamma^\theta\gamma_\mu q^2
				+ \frac{1}{4}M_T^2 \gamma^\alpha\gamma^\beta\gamma^\lambda\gamma^\mu\gamma^\theta\gamma_\mu q^2 + M_T^4 \gamma^\alpha\gamma^\beta\gamma^\lambda\gamma^\theta
				\bigg]P_\alpha P_\beta P_\lambda P_\theta
		\end{align}
		
		
		Using the contraction of gamma matrices identities:
		
		\begin{align}
			&\left[\slashed{P} (-\slashed{q} +M_T)\right]^4 \supset \bigg[\frac{1}{24}\gamma^\alpha \gamma^\nu\gamma^\beta\gamma_\nu\gamma^\lambda\gamma^\sigma\gamma^\theta\gamma_\sigma q^4
			+\frac{1}{24} \gamma^\alpha \gamma^\mu\gamma^\beta\gamma^\nu\gamma^\lambda\gamma_\mu\gamma^\theta\gamma_\nu q^4
			+ \frac{1}{24}\gamma^\alpha \gamma^\mu\gamma^\beta\gamma^\nu\gamma^\lambda\gamma_\nu\gamma^\theta\gamma_\mu q^4 \nonumber\\
			&-\frac{1}{2}M_T^2 \gamma^\alpha \gamma^\beta \gamma^\lambda\gamma^\theta q^2+ M_T^2g^{\beta\lambda} \gamma^\alpha\gamma^\theta q^2
			-\frac{1}{2}M_T^2 \gamma^\alpha \gamma^\theta \gamma^\lambda \gamma^\beta q^2 \nonumber\\
			&-\frac{1}{2}M_T^2 \gamma^\alpha\gamma^\beta\gamma^\lambda\gamma^\theta
			+M_T^2g^{\lambda\theta} \gamma^\alpha\gamma^\beta q^2
			- \frac{1}{2}M_T^2 \gamma^\alpha\gamma^\beta\gamma^\lambda\gamma^\theta q^2 + M_T^4 \gamma^\alpha\gamma^\beta\gamma^\lambda\gamma^\theta 
			\bigg]P_\alpha P_\beta P_\lambda P_\theta
		\end{align}
		
		\begin{align}
			&\left[\slashed{P} (-\slashed{q} +M_T)\right]^4 \supset \bigg[\frac{1}{6}\gamma^\alpha \gamma^\beta\gamma^\lambda\gamma^\theta q^4
			-\frac{1}{12} \gamma^\alpha \gamma^\mu\gamma^\beta\gamma^\theta\gamma_\mu\gamma^\lambda q^4
			- \frac{1}{12}\gamma^\alpha \gamma^\mu\gamma^\beta\gamma^\lambda\gamma^\theta\gamma_\mu q^4 \nonumber\\
			&-\frac{1}{2}M_T^2 \gamma^\alpha \gamma^\beta \gamma^\lambda\gamma^\theta  q^2 + M_T^2g^{\beta\lambda} \gamma^\alpha\gamma^\theta q^2
			-\frac{1}{2}M_T^2 \gamma^\alpha \gamma^\theta \gamma^\lambda \gamma^\beta q^2 \nonumber\\
			&-\frac{1}{2}M_T^2 \gamma^\alpha\gamma^\beta\gamma^\lambda\gamma^\theta
			+M_T^2g^{\lambda\theta} \gamma^\alpha\gamma^\beta q^2
			- \frac{1}{2}M_T^2 \gamma^\alpha\gamma^\beta\gamma^\lambda\gamma^\theta q^2+ M_T^4 \gamma^\alpha\gamma^\beta\gamma^\lambda\gamma^\theta
			\bigg]P_\alpha P_\beta P_\lambda P_\theta
		\end{align}
		
		\begin{align}
			&\left[\slashed{P} (-\slashed{q} +M_T)\right]^4 \supset \bigg[\frac{1}{6}\gamma^\alpha \gamma^\beta\gamma^\lambda\gamma^\theta q^4
			-\frac{1}{3}g^{\beta\theta} \gamma^\alpha \gamma^\lambda  q^4
			+ \frac{1}{6}\gamma^\alpha \gamma^\theta\gamma^\lambda\gamma^\beta q^4 \nonumber\\
			&-\frac{1}{2}M_T^2 \gamma^\alpha \gamma^\beta \gamma^\lambda\gamma^\theta q^2 + M_T^2g^{\beta\lambda} \gamma^\alpha\gamma^\theta q^2
			-\frac{1}{2}M_T^2 \gamma^\alpha \gamma^\theta \gamma^\lambda \gamma^\beta q^2 \nonumber\\
			&-\frac{1}{2}M_T^2 \gamma^\alpha\gamma^\beta\gamma^\lambda\gamma^\theta q^2
			+M_T^2g^{\lambda\theta} \gamma^\alpha\gamma^\beta q^2
			- \frac{1}{2}M_T^2 \gamma^\alpha\gamma^\beta\gamma^\lambda\gamma^\theta q^2 + M_T^4 \gamma^\alpha\gamma^\beta\gamma^\lambda\gamma^\theta
			\bigg]P_\alpha P_\beta P_\lambda P_\theta
		\end{align}
		
		Now we need to organize the terms with four gamma matrices that are not in the correct order by using:
		
		\begin{align}
			\gamma^\theta\gamma^\lambda\gamma^\beta = (2 g^{\theta\lambda}\gamma^\beta - 2g^{\theta\beta}\gamma^\lambda + 2g^{\beta\lambda} \gamma^{\theta}- \gamma^\beta\gamma^\lambda \gamma^\theta)
		\end{align}
		
		we get:
		\begin{align}
			&\left[\slashed{P} (-\slashed{q} +M_T)\right]^4 \supset \bigg[\frac{1}{6}\gamma^\alpha \gamma^\beta\gamma^\lambda\gamma^\theta q^4
			-\frac{1}{3}g^{\beta\theta} \gamma^\alpha \gamma^\lambda q^4
			+ \frac{1}{6}\gamma^\alpha (2 g^{\theta\lambda}\gamma^\beta - 2g^{\theta\beta}\gamma^\lambda + 2g^{\beta\lambda} \gamma^{\theta}- \gamma^\beta\gamma^\lambda \gamma^\theta) q^4\nonumber\\
			&-\frac{1}{2}M_T^2 \gamma^\alpha \gamma^\beta \gamma^\lambda\gamma^\theta q^2 + M_T^2g^{\beta\lambda} \gamma^\alpha\gamma^\theta q^2
			-\frac{1}{2}M_T^2 \gamma^\alpha(2 g^{\theta\lambda}\gamma^\beta - 2g^{\theta\beta}\gamma^\lambda + 2g^{\beta\lambda} \gamma^{\theta} - \gamma^\beta\gamma^\lambda \gamma^\theta) q^2\nonumber\\
			&-\frac{1}{2}M_T^2 \gamma^\alpha\gamma^\beta\gamma^\lambda\gamma^\theta  q^2
			+M_T^2g^{\lambda\theta} \gamma^\alpha\gamma^\beta q^2
			- \frac{1}{2}M_T^2 \gamma^\alpha\gamma^\beta\gamma^\lambda\gamma^\theta q^2 + M_T^4 \gamma^\alpha\gamma^\beta\gamma^\lambda\gamma^\theta
			\bigg]P_\alpha P_\beta P_\lambda P_\theta
		\end{align}
		
		\begin{align}
			&\left[\slashed{P} (-\slashed{q} +M_T)\right]^4 \supset \bigg[
			-\frac{2}{3}g^{\beta\theta} \gamma^\alpha \gamma^\lambda q^4
			+ \frac{1}{3}g^{\theta\lambda}\gamma^\alpha \gamma^\beta q^4 + \frac{1}{3}g^{\beta\lambda} \gamma^\alpha\gamma^{\theta} q^4 +M_T^2 g^{\theta\beta}\gamma^\alpha\gamma^\lambda q^2 \nonumber\\
			&-M_T^2 \gamma^\alpha\gamma^\beta\gamma^\lambda\gamma^\theta q^2 + M_T^4 \gamma^\alpha\gamma^\beta\gamma^\lambda\gamma^\theta
			\bigg]P_\alpha P_\beta P_\lambda P_\theta
		\end{align}
		
		Here the spinor indices are not fully contracted. Therefore, we need to compute the trace of these gamma matrices, i.e:
		
		\begin{align}
			&\tr(\gamma^\mu\gamma^\nu) = 4g^{\mu\nu}\\
			&\tr(\gamma^\alpha\gamma^\beta\gamma^\lambda\gamma^\theta) = 4(g^{\alpha\beta}g^{\lambda\theta}- g^{\alpha\lambda}g^{\beta\theta}+g^{\alpha\theta} g^{\beta\lambda})
		\end{align}
		
		Using this:
		
		\begin{align}
			&\tr\left[\slashed{P} (-\slashed{q} +M_T)\right]^4 \supset \bigg[
			-\frac{8}{3}g^{\beta\theta} g^{\alpha\lambda} q^4
			+ \frac{4}{3}g^{\theta\lambda}g^{\alpha\beta}  q^4 + \frac{4}{3}g^{\beta\lambda} g^{\alpha\theta} q^4+4M_T^2 g^{\theta\beta}g^{\alpha\lambda} q^2\nonumber\\
			&-4M_T^2 (g^{\alpha\beta}g^{\lambda\theta}- g^{\alpha\lambda}g^{\beta\theta}+g^{\alpha\theta} g^{\beta\lambda}) q^2 + 4M_T^4 (g^{\alpha\beta}g^{\lambda\theta}- g^{\alpha\lambda}g^{\beta\theta}+g^{\alpha\theta} g^{\beta\lambda})
			\bigg]P_\alpha P_\beta P_\lambda P_\theta
		\end{align}
		
		\begin{align}
			\tr\left[\slashed{P} (-\slashed{q} +M_T)\right]^4 &\supset \left(-\frac{8}{3}q^4 +8M_T^2 q^2 + 4M_T^4\right)P^\mu P^\nu P_\mu P_\nu + \left(\frac{4}{3}q^4 -4M_T^2 q^2+ 4M_T^4\right)P^\mu P_\mu P^\nu P_\nu\nonumber\\
			&+ \left(\frac{4}{3}q^4 -4M_T^2 q^2+ 4M_T^4\right)P^\mu P^\nu P_\nu P_\mu
		\end{align}
		
		\begin{align}
			\tr\left[\slashed{P} (-\slashed{q} +M_T)\right]^4 &\supset \left(-\frac{8}{3} +8M_T^2 + 4M_T^4\right)P^\mu P^\nu P_\mu P_\nu + \left(-\frac{8}{3} -4M_T^2 + 4M_T^4\right)P^\mu P_\mu P^\nu P_\nu\nonumber\\
			&+ \left(\frac{4}{3} -4M_T^2 + 4M_T^4\right)P^\mu P^\nu P_\nu P_\mu + 4P^\mu P_\mu P^\nu P_\nu
		\end{align}
		
	\end{itemize}
	
	\newpage
	
	
	
	
	
	
	
	
	
	
	
	
	
	
	
	
	
	
	
	
	
	
	
	
	
	
	
	
	
	
	
	
	
	
	
	
	
	
	
	
	
	
	
	
	
	
	
	
	
	
	
	
	
	
	
	
	
	
	
	
	
	
	
	
	
	
	
	
	
	
	
	
	
	
	
	
	
	
	
	
	
	
	
	
	
	
	
	
	
	
	
	
	
	
	
	
	
	
	
	
	
	
	
	
	
	
	
	
	
	
	
	
	
	
	
	
	
	
	
	
	
	
	
	
	
	
	
	
	
	
	
	
	
	
	
	
	
	
	
	
	
	
	
	
	
	
	
	
	
	
	
	
	
	
	
	
	
	
	
	
	
	
	
	
	
	
	
	
	
	
	
	
	
	
	
	
	
	
	
	
	
	
	
	
	
	
	
	
	
	
	
	
	
	
	
	
	
	
	
	
	
	
	
	
	
	
	
	
	
	
	
	
	
	
	
	
	
	
	
	
	
	
	
	
	
	
	
	
	
	
	
	
	
	
	
	
	
	
	
	
	
	
	
	
	
	
	
	
	
	
	
	
	
	
	
	
	
	
	
	
	
	
	
	
	
	
	
	
	
	
	
	
	
	
	
	
	
	
	
	
	
	
	
	
	
	
	
	
	
	
	
	
	
	
	
	
	
	
	
	
	
	
	
	
	
	
	
	
	
	
	
	
	
	
	
	
	
	
	
	
	
	
	
	
	\appendix
	

	
	
	
	
	
	
	
	
	\bibliographystyle{unsrt}
	\bibliography{refs}
	
	
	
	
	
\end{document}
